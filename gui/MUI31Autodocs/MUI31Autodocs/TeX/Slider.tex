%---------------- Functions ------------

\chapter{Slider.mui/Slider.mui}
\rule{\textwidth}{1mm}
\vspace{0.5cm}
\begin{deflist}{MMMMMMMM}
\item[\bf FUNCTION]
The slider class generates a gui element that allows a
user to adjust a numeric value. The programmer has not
very much influence on the slider's outfit, there are
only very few tags available. Future versions of MUI
will probably include some preferences options to
allow the user (*not* the programmer) to configure
this outfit.

Note that since slider is a subclass of group class,
you can get horizontal or vertical sliders by simply
using the MUIA\_Group\_Horiz attribute. Default is
a horizontal slider.
\end{deflist}


\subsection{Slider.mui/MUIA\_Slider\_Horiz}
\rule{\textwidth}{1mm}
\vspace{0.5cm}
\begin{deflist}{MMMMMMMM}
\item[\bf NAME]
\begin{description}
\item[MUIA\_Slider\_Horiz]  -- (V11) [ISG], BOOL
\end{description}

\item[\bf FUNCTION]
Specify if you want a horizontal or vertical
slider. Also understands MUIA\_Group\_Horiz to be
compatible with previous versions of MUI.
\end{deflist}


\subsection{Slider.mui/MUIA\_Slider\_Level}
\rule{\textwidth}{1mm}
\vspace{0.5cm}
\begin{deflist}{MMMMMMMM}
\item[\bf NAME]
\begin{description}
\item[MUIA\_Slider\_Level]  -- (V4 ) [ISG], LONG (OBSOLETE)
\end{description}

\item[\bf FUNCTION]
The current position of the slider knob. This value
is guaranteed to be between MUIA\_Slider\_Min and
MUIA\_Slider\_Max.

\item[\bf EXAMPLE]
\begin{flushleft}
\begin{verbatim}
/* vertical task priority slider */
SliderObject,
   MUIA_Group_Horiz , FALSE,
   MUIA_Slider_Min  , -20,
   MUIA_Slider_Max  ,  20,
   MUIA_Slider_Level,   0,
   End;
\end{verbatim}
\end{flushleft}
\item[\bf SEE ALSO]
MUIA\_Slider\_Min, MUIA\_Slider\_Max
\end{deflist}


\subsection{Slider.mui/MUIA\_Slider\_Max}
\rule{\textwidth}{1mm}
\vspace{0.5cm}
\begin{deflist}{MMMMMMMM}
\item[\bf NAME]
\begin{description}
\item[MUIA\_Slider\_Max]  -- (V4 ) [ISG], LONG (OBSOLETE)
\end{description}

\item[\bf FUNCTION]
Adjust the maximum value for a slider object.

\item[\bf SEE ALSO]
MUIA\_Slider\_Min, MUIA\_Slider\_Level
\end{deflist}


\subsection{Slider.mui/MUIA\_Slider\_Min}
\rule{\textwidth}{1mm}
\vspace{0.5cm}
\begin{deflist}{MMMMMMMM}
\item[\bf NAME]
\begin{description}
\item[MUIA\_Slider\_Min]  -- (V4 ) [ISG], LONG (OBSOLETE)
\end{description}

\item[\bf FUNCTION]
Adjust the minimum value for a slider object.
Of course you can use negative number, e.g.
for a slider to adjust task priority.

\item[\bf SEE ALSO]
MUIA\_Slider\_Max, MUIA\_Slider\_Level
\end{deflist}


\subsection{Slider.mui/MUIA\_Slider\_Quiet}
\rule{\textwidth}{1mm}
\vspace{0.5cm}
\begin{deflist}{MMMMMMMM}
\item[\bf NAME]
\begin{description}
\item[MUIA\_Slider\_Quiet]  -- (V6 ) [I..], BOOL
\end{description}

\item[\bf FUNCTION]
When set to TRUE, the slider doesn't display it's current
level in a text object.

\item[\bf SEE ALSO]
MUIA\_Slider\_Level
\end{deflist}


\subsection{Slider.mui/MUIA\_Slider\_Reverse}
\rule{\textwidth}{1mm}
\vspace{0.5cm}
\begin{deflist}{MMMMMMMM}
\item[\bf NAME]
\begin{description}
\item[MUIA\_Slider\_Reverse]  -- (V4 ) [ISG], BOOL (OBSOLETE)
\end{description}

\item[\bf FUNCTION]
Setting this attribute to TRUE will reverse the direction
of the slider.

\item[\bf SEE ALSO]
MUIA\_Slider\_Min, MUIA\_Slider\_Max, MUIA\_Slider\_Level
\end{deflist}


%---------------- End of File ----------
