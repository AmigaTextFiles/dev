%---------------- Functions ------------

\chapter{Listview.mui/Listview.mui}
\rule{\textwidth}{1mm}
\vspace{0.5cm}
\begin{deflist}{MMMMMMMM}
\item[\bf FUNCTION]
It's important to know that MUI makes a difference between a list and a listview. A list is just a collection of some entries and is part of a listview, which attaches a scrollbar and input handling to the list.
During object creation time, you have to be careful not specifying listview tags for the list object or list tags for the listview object, both versions won't work. Once the objects are setup, you can of course talk to the listview as if it was the list directly.
\end{deflist}


\subsection{Listview.mui/MUIA\_Listview\_ClickColumn}
\rule{\textwidth}{1mm}
\vspace{0.5cm}
\begin{deflist}{MMMMMMMM}
\item[\bf NAME]
\begin{description}
\item[MUIA\_Listview\_ClickColumn]  -- (V7 ) [..G], LONG
\end{description}

\item[\bf FUNCTION]
When using a multi column list, this attribute contains the number of
the column where the user clicked.

\item[\bf SEE ALSO]
MUIA\_Listview\_DefClickColumn
\end{deflist}


\subsection{Listview.mui/MUIA\_Listview\_DefClickColumn}
\rule{\textwidth}{1mm}
\vspace{0.5cm}
\begin{deflist}{MMMMMMMM}
\item[\bf NAME]
\begin{description}
\item[MUIA\_Listview\_DefClickColumn]  -- (V7 ) [ISG], LONG
\end{description}

\item[\bf FUNCTION]
When the listview is controlled with the keyboard and the user
presses RETURN, the value given here will be used as default
for MUIA\_Listview\_ClickColumn.

\item[\bf SEE ALSO]
MUIA\_Listview\_ClickColumn
\end{deflist}


\subsection{Listview.mui/MUIA\_Listview\_DoubleClick}
\rule{\textwidth}{1mm}
\vspace{0.5cm}
\begin{deflist}{MMMMMMMM}
\item[\bf NAME]
\begin{description}
\item[MUIA\_Listview\_DoubleClick]  -- (V4 ) [I.G], BOOL
\end{description}

\item[\bf FUNCTION]
This attribute is set to TRUE whenever the user
double clicks on an entry in the list.

\item[\bf SEE ALSO]
MUIA\_Listview\_SelectChange
\end{deflist}


\subsection{Listview.mui/MUIA\_Listview\_DragType}
\rule{\textwidth}{1mm}
\vspace{0.5cm}
\begin{deflist}{MMMMMMMM}
\item[\bf NAME]
\begin{description}
\item[MUIA\_Listview\_DragType]  -- (V11) [ISG], LONG
\end{description}

\item[\bf SPECIAL INPUTS]
\begin{flushleft}
\begin{verbatim}
MUIV_Listview_DragType_None
MUIV_Listview_DragType_Immediate
\end{verbatim}
\end{flushleft}
\item[\bf FUNCTION]
If you want the user to be able to drag items out of
your list, you must set this for the listview class.
Currently, only one drag type is defined.
\end{deflist}


\subsection{Listview.mui/MUIA\_Listview\_Input}
\rule{\textwidth}{1mm}
\vspace{0.5cm}
\begin{deflist}{MMMMMMMM}
\item[\bf NAME]
\begin{description}
\item[MUIA\_Listview\_Input]  -- (V4 ) [I..], BOOL
\end{description}

\item[\bf FUNCTION]
Setting this to FALSE will result in a read only
list view. Defaults to TRUE.

\item[\bf SEE ALSO]
MUIA\_Listview\_MultiSelect
\end{deflist}


\subsection{Listview.mui/MUIA\_Listview\_List}
\rule{\textwidth}{1mm}
\vspace{0.5cm}
\begin{deflist}{MMMMMMMM}
\item[\bf NAME]
\begin{description}
\item[MUIA\_Listview\_List]  -- (V4 ) [I.G], Object *
\end{description}

\item[\bf FUNCTION]
Every listview needs a list object as child.
Specify it here.

As every other child, it will get disposes when
its parent object is disposed.

\item[\bf EXAMPLE]
\begin{flushleft}
\begin{verbatim}
ListviewObject,
   MUIA_Listview_Input, FALSE,
   MUIA_Listview_List , ListObject,
      ReadListFrame,
      MUIA_List_Format       , ",,",
      End,
   End;
\end{verbatim}
\end{flushleft}
\item[\bf SEE ALSO]
MUIA\_Listview\_Input
\end{deflist}


\subsection{Listview.mui/MUIA\_Listview\_MultiSelect}
\rule{\textwidth}{1mm}
\vspace{0.5cm}
\begin{deflist}{MMMMMMMM}
\item[\bf NAME]
\begin{description}
\item[MUIA\_Listview\_MultiSelect]  -- (V7 ) [I..], LONG
\end{description}

\item[\bf SPECIAL INPUTS]
\begin{flushleft}
\begin{verbatim}
MUIV_Listview_MultiSelect_None
MUIV_Listview_MultiSelect_Default
MUIV_Listview_MultiSelect_Shifted
MUIV_Listview_MultiSelect_Always
\end{verbatim}
\end{flushleft}
\item[\bf FUNCTION]
Four possibilities exist for a listviews multi select
capabilities:

MUIV\_Listview\_MultiSelect\_None:
   The listview cannot multiselect at all.

MUIV\_Listview\_MultiSelect\_Default:
   The multi select type (with or without shift) depends on
   the users preferences setting.

MUIV\_Listview\_MultiSelect\_Shifted:
   Overrides the users prefs, multi selecting only together
   with shift key.

MUIV\_Listview\_MultiSelect\_Always:
   Overrides the users prefs, multi selecting without shift key.

Please do *not* override the users prefs unless you have a
good reason!

\item[\bf SEE ALSO]
MUIA\_List\_MultiTestHook
\end{deflist}


\subsection{Listview.mui/MUIA\_Listview\_ScrollerPos}
\rule{\textwidth}{1mm}
\vspace{0.5cm}
\begin{deflist}{MMMMMMMM}
\item[\bf NAME]
\begin{description}
\item[MUIA\_Listview\_ScrollerPos]  -- (V10) [I..], BOOL
\end{description}

\item[\bf SPECIAL INPUTS]
\begin{flushleft}
\begin{verbatim}
MUIV_Listview_ScrollerPos_Default
MUIV_Listview_ScrollerPos_Left
MUIV_Listview_ScrollerPos_Right
\end{verbatim}
\end{flushleft}
\item[\bf FUNCTION]
Specifies the position of a listviews scrollbar.
Don't use this tag unless it's absolutely required!
\end{deflist}


\subsection{Listview.mui/MUIA\_Listview\_SelectChange}
\rule{\textwidth}{1mm}
\vspace{0.5cm}
\begin{deflist}{MMMMMMMM}
\item[\bf NAME]
\begin{description}
\item[MUIA\_Listview\_SelectChange]  -- (V4 ) [..G], BOOL
\end{description}

\item[\bf FUNCTION]
This attribute is set to TRUE whenever the selection state
of one or more items in the list is changing. You can
use this e.g. if you want to display the number of
selected items in a status line.

\item[\bf SEE ALSO]
MUIA\_List\_MultiSelect
\end{deflist}


%---------------- End of File ----------
