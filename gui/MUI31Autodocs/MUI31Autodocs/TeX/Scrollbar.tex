%---------------- Functions ------------

\chapter{Scrollbar.mui/Scrollbar.mui}
\rule{\textwidth}{1mm}
\vspace{0.5cm}
\begin{deflist}{MMMMMMMM}
\item[\bf FUNCTION]
The Scrollbar class has no objects and attributes itself.
It just connects a proportional gadget and two button
gadgets with approriate imagery to make up a scrollbar.

Since Scrollbar class is a subclass of Group class,
every attribute and method is passed through to all
of its children. Thus, you can talk and listen to
a scrollbar as if it was just a single prop gadget.

You can use the attribute MUIA\_Group\_Horiz as with
any other group to determine if the scrollbar should
be horizontal or vertical. By default, a vertical
scrollbar is generated.
\end{deflist}


\subsection{Scrollbar.mui/MUIA\_Scrollbar\_Type}
\rule{\textwidth}{1mm}
\vspace{0.5cm}
\begin{deflist}{MMMMMMMM}
\item[\bf NAME]
\begin{description}
\item[MUIA\_Scrollbar\_Type]  -- (V11) [I..], LONG
\end{description}

\item[\bf SPECIAL INPUTS]
\begin{flushleft}
\begin{verbatim}
MUIV_Scrollbar_Type_Default
MUIV_Scrollbar_Type_Bottom
MUIV_Scrollbar_Type_Top
MUIV_Scrollbar_Type_Sym
\end{verbatim}
\end{flushleft}
\item[\bf FUNCTION]
Specify a certain scrollbar type. Normally, you should respect
the users choice and avoid using this attribute.
\end{deflist}


%---------------- End of File ----------
