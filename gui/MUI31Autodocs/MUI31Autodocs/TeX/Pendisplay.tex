%---------------- Functions ------------

\chapter{Pendisplay.mui/Pendisplay.mui}
\rule{\textwidth}{1mm}
\vspace{0.5cm}
\begin{deflist}{MMMMMMMM}
\item[\bf FUNCTION]
Pendisplay class takes a struct MUI\_PenSpec and displays it.
Its main use is to be sub-classed by Poppen class which adds
a popup window to adjust the MUI\_PenSpec. Poppen class should
be used by every application that allows users to configure
custom drawing pens.
\end{deflist}


\subsection{Pendisplay.mui/MUIA\_Pendisplay\_RGBcolor}
\rule{\textwidth}{1mm}
\vspace{0.5cm}
\begin{deflist}{MMMMMMMM}
\item[\bf NAME]
\begin{description}
\item[MUIA\_Pendisplay\_RGBcolor]  -- (V11) [ISG], struct MUI\_RBBcolor *
\end{description}

\item[\bf FUNCTION]
Private attribute, only for PSI.
\end{deflist}


\subsection{Pendisplay.mui/MUIA\_Pendisplay\_Spec}
\rule{\textwidth}{1mm}
\vspace{0.5cm}
\begin{deflist}{MMMMMMMM}
\item[\bf NAME]
\begin{description}
\item[MUIA\_Pendisplay\_Spec]  -- (V11) [ISG], struct MUI\_PenSpec  *
\end{description}

\item[\bf FUNCTION]
The black box structure MUI\_PenSpec specifies a drawing
pen which should be displayed by Pendisplay class. If you
use Poppen class to allow your users to configure custom
drawing pens, its this attribute that you need to get()
and save in your preferences.

Use the functions MUI\_ObtainPen() and MUI\_ReleasePen()
from muimaster.library and the MUIPEN() macro to get
a usable value for SetAPen() from a struct MUI\_PenSpec.

NOTE: In allmost all cases you will use Poppen class which
is a subclass of Pendisplay class.

\item[\bf SEE ALSO]
muimaster.library/MUI\_ObtainPen(),
muimaster.library/MUI\_ReleasePen()
\end{deflist}


%---------------- End of File ----------
