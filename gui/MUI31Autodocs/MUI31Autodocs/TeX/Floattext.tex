%---------------- Functions ------------

\chapter{Floattext.mui/Floattext.mui}
\rule{\textwidth}{1mm}
\vspace{0.5cm}
\begin{deflist}{MMMMMMMM}
\item[\bf FUNCTION]
Floattext class is a subclass of list class that takes
a big text string as input and splits it up into several
lines to be dislayed. Formatting capabilities include
paragraphs an justified text with word wrap.
\end{deflist}


\subsection{Floattext.mui/MUIA\_Floattext\_Justify}
\rule{\textwidth}{1mm}
\vspace{0.5cm}
\begin{deflist}{MMMMMMMM}
\item[\bf NAME]
\begin{description}
\item[MUIA\_Floattext\_Justify]  -- (V4 ) [ISG], BOOL
\end{description}

\item[\bf FUNCTION]
Indicate whether you want your the text aligned
to the left and right border. MUI will try to
insert spaces between words to reach this goal.

If you want right aligned or centered text,
use the MUIA\_List\_Format attribute.

\item[\bf SEE ALSO]
MUIA\_Floattext\_Text, MUIA\_List\_Format
\end{deflist}


\subsection{Floattext.mui/MUIA\_Floattext\_SkipChars}
\rule{\textwidth}{1mm}
\vspace{0.5cm}
\begin{deflist}{MMMMMMMM}
\item[\bf NAME]
\begin{description}
\item[MUIA\_Floattext\_SkipChars]  -- (V4 ) [IS.], STRPTR
\end{description}

\item[\bf FUNCTION]
Defines an array of characters that shall be skipped
when displaying the text. If you e.g. want to
display a fido message and know it has some CTRL-A
control characters in it, you could set this
attrinbute to "'$\backslash$1"' to prevent floattext class
from displaying unreadable crap.

\item[\bf SEE ALSO]
MUIA\_Floattext\_Text
\end{deflist}


\subsection{Floattext.mui/MUIA\_Floattext\_TabSize}
\rule{\textwidth}{1mm}
\vspace{0.5cm}
\begin{deflist}{MMMMMMMM}
\item[\bf NAME]
\begin{description}
\item[MUIA\_Floattext\_TabSize]  -- (V4 ) [IS.], LONG
\end{description}

\item[\bf FUNCTION]
Adjust the tab size for a text. The tab size is measured
in spaces, so if you plan to use tabs not only at the
beginning of a paragraph, you should consider using
the fixed width font.

Tab size defaults to 8.

\item[\bf SEE ALSO]
MUIA\_Floattext\_Text
\end{deflist}


\subsection{Floattext.mui/MUIA\_Floattext\_Text}
\rule{\textwidth}{1mm}
\vspace{0.5cm}
\begin{deflist}{MMMMMMMM}
\item[\bf NAME]
\begin{description}
\item[MUIA\_Floattext\_Text]  -- (V4 ) [ISG], STRPTR
\end{description}

\item[\bf FUNCTION]
String of characters to be displayed as floattext.
This string may contain linefeeds to mark the end
of paragraphs or tab characters for indention.

MUI will automatically format the text according
to the width of the floattext object. If a word
won't fit into the current line, it will be wrapped.

If you plan to use tabs not only at the beginning
of a line you should consider using the configured
fixed width font.

MUI copies the complete string into a private buffer,
you won't need to keep your text in memory. If memory
is low, nothing will be displayed. Thats why you always
have to be prepared for handling a NULL pointer when
getting back MUIA\_Floattext\_Text.

Setting MUIA\_Floattext\_Text to NULL means to clear
the current text.

Please note that justification and word wrap with
proportional fonts is a complicated operation and
may take a considerable amount of time, especially
with long texts on slow machines.

\item[\bf EXAMPLE]
\begin{flushleft}
\begin{verbatim}
char *text = AllocVec(filesize,MEMF_ANY);

Read(file,text,filesize);

fto = FloattextObject,
   MUIA_Floattext_Text,text,
   End;

FreeVec(text);

/* ... if you need your text later, you can get it   */
/* with a simple get(fto,MUIA_Floattext_Text,&text); */
\end{verbatim}
\end{flushleft}
\item[\bf SEE ALSO]
MUIA\_Floattext\_Justify, MUIA\_Floattext\_TabSize,
MUIA\_Floattext\_SkipChars
\end{deflist}


%---------------- End of File ----------
