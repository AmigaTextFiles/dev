%---------------- Functions ------------

\chapter{Image.mui/Image.mui}
\rule{\textwidth}{1mm}
\vspace{0.5cm}
\begin{deflist}{MMMMMMMM}
\item[\bf FUNCTION]
Image class is used to display one of MUI's standard
images or some selfmade image data.
\end{deflist}


\subsection{Image.mui/MUIA\_Image\_FontMatch}
\rule{\textwidth}{1mm}
\vspace{0.5cm}
\begin{deflist}{MMMMMMMM}
\item[\bf NAME]
\begin{description}
\item[MUIA\_Image\_FontMatch]  -- (V4 ) [I..], BOOL
\end{description}

\item[\bf FUNCTION]
If TRUE, width and height of the given image will be
scaled to match the current font. Images are always
defined with a reference font of topaz/8, bigger fonts
will make the image grow (as long as its maximum size
is big enough).

\item[\bf EXAMPLE]
\begin{flushleft}
\begin{verbatim}
The arrows of a scroll bar are e.g. defined with
MUIA_Image_FontMatch.
\end{verbatim}
\end{flushleft}
\item[\bf SEE ALSO]
MUIA\_Image\_FontMatch, MUIA\_Image\_FontMatchWidth
\end{deflist}


\subsection{Image.mui/MUIA\_Image\_FontMatchHeight}
\rule{\textwidth}{1mm}
\vspace{0.5cm}
\begin{deflist}{MMMMMMMM}
\item[\bf NAME]
\begin{description}
\item[MUIA\_Image\_FontMatchHeight]  -- (V4 ) [I..], BOOL
\end{description}

\item[\bf FUNCTION]
If TRUE, the height of the given image will be scaled
to match the current font. Images are always defined
with a reference font of topaz/8, bigger fonts will
make the image grow (as long as its maximum size
is big enough).

\item[\bf SEE ALSO]
MUIA\_Image\_FontMatch, MUIA\_Image\_FontMatchWidth
\end{deflist}


\subsection{Image.mui/MUIA\_Image\_FontMatchWidth}
\rule{\textwidth}{1mm}
\vspace{0.5cm}
\begin{deflist}{MMMMMMMM}
\item[\bf NAME]
\begin{description}
\item[MUIA\_Image\_FontMatchWidth]  -- (V4 ) [I..], BOOL
\end{description}

\item[\bf FUNCTION]
If TRUE, the width of the given image will be scaled
to match the current font. Images are always defined
with a reference font of topaz/8, bigger fonts will
make the image grow (as long as its maximum size
is big enough).

\item[\bf SEE ALSO]
MUIA\_Image\_FontMatch, MUIA\_Image\_FontMatchHeight
\end{deflist}


\subsection{Image.mui/MUIA\_Image\_FreeHoriz}
\rule{\textwidth}{1mm}
\vspace{0.5cm}
\begin{deflist}{MMMMMMMM}
\item[\bf NAME]
\begin{description}
\item[MUIA\_Image\_FreeHoriz]  -- (V4 ) [I..], BOOL
\end{description}

\item[\bf FUNCTION]
Tell the image if its allowed to get scaled horizontally.
Defaults to FALSE.

\item[\bf SEE ALSO]
MUIA\_Image\_FreeVert, MUIA\_Image\_FontMatch
\end{deflist}


\subsection{Image.mui/MUIA\_Image\_FreeVert}
\rule{\textwidth}{1mm}
\vspace{0.5cm}
\begin{deflist}{MMMMMMMM}
\item[\bf NAME]
\begin{description}
\item[MUIA\_Image\_FreeVert]  -- (V4 ) [I..], BOOL
\end{description}

\item[\bf FUNCTION]
Tell the image if its allowed to get scaled vertically.
Defaults to FALSE.

\item[\bf SEE ALSO]
MUIA\_Image\_FreeHoriz, MUIA\_Image\_FontMatch
\end{deflist}


\subsection{Image.mui/MUIA\_Image\_OldImage}
\rule{\textwidth}{1mm}
\vspace{0.5cm}
\begin{deflist}{MMMMMMMM}
\item[\bf NAME]
\begin{description}
\item[MUIA\_Image\_OldImage]  -- (V4 ) [I..], struct Image *
\end{description}

\item[\bf FUNCTION]
Allows you to use any conventional image structure
within a MUI window. The resulting object is always
as big as the image and not resizable.
\end{deflist}


\subsection{Image.mui/MUIA\_Image\_Spec}
\rule{\textwidth}{1mm}
\vspace{0.5cm}
\begin{deflist}{MMMMMMMM}
\item[\bf NAME]
\begin{description}
\item[MUIA\_Image\_Spec]  -- (V4 ) [I..], char *
\end{description}

\item[\bf FUNCTION]
Specify the type of your image. Usually, you will
use one of the predefined standard images here,
(one of the MUII\_xxx definitions from mui.h),
but you also can supply a string containing
 a MUI image specification. Image specifications
always starts with a digit, followed by a ':',
followed by some parameters. Currently, the
following things are defined (all numeric
parameters need to be ascii values!):

"'0:$<$x$>$"' where $<$x$>$ is between MUII\_BACKGROUND and
        MUII\_FILLBACK2 identifying a builtin pattern.

"'1:$<$x$>$"' where $<$x$>$ identifies a builtin standard image.
        Don't use this, use "'6:$<$x$>$"' instead.

"'2:$<$r$>$,$<$g$>$,$<$b$>$"' where $<$r$>$, $<$g$>$ and $<$b$>$ are 32-bit RGB
                color values specified as 8-digit hex
                string (e.g. 00000000 or ffffffff).
                Kick 2.x users will get an empty image.

"'3:$<$n$>$"' where $<$n$>$ is the name of an external boopsi
        image class.

"'4:$<$n$>$"' where $<$n$>$ is the name of an external MUI brush.

"'5:$<$n$>$"' where $<$n$>$ is the name of an external picture
        file that should be loaded with datatypes.
        Kick 2.x users will get an empty image.

"'6:$<$x$>$"' where $<$x$>$ is between MUII\_WindowBack and
        MUII\_Count-1 identifying a preconfigured
        image/background.

\item[\bf SEE ALSO]
MUIA\_Image\_OldImage
\end{deflist}


\subsection{Image.mui/MUIA\_Image\_State}
\rule{\textwidth}{1mm}
\vspace{0.5cm}
\begin{deflist}{MMMMMMMM}
\item[\bf NAME]
\begin{description}
\item[MUIA\_Image\_State]  -- (V4 ) [IS.], LONG
\end{description}

\item[\bf FUNCTION]
Some MUI images offer different states, you can select
one of the by setting this attribute. Simply use
one of the IDS\_NORMAL, IDS\_SELECTED, ... values
defined in "'intuition/imageclass.h"'.

Note: Objects that respond to user input will
      automatically toggle their state between
      IDS\_NORMAL to IDS\_SELECTED depending on
      their MUIA\_Selected attribute.

\item[\bf SEE ALSO]
MUIA\_Image\_Spec
\end{deflist}


%---------------- End of File ----------
