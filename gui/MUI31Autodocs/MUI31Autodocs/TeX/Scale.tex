%---------------- Functions ------------

\chapter{Scale.mui/Scale.mui}
\rule{\textwidth}{1mm}
\vspace{0.5cm}
\begin{deflist}{MMMMMMMM}
\item[\bf FUNCTION]
A Scale object generates a percentage scale
running from 0\% to 100\%. A good place for
such an object is e.g. below a fuel gauge.

Depending on how much space is available,
the scale will be more or less detailed.

Due to MUI's automatic layout system, you
don't need to worry about it's size. When
placed in a vertical group just below the
object you want to scale, everything is
fine.
\end{deflist}


\subsection{Scale.mui/MUIA\_Scale\_Horiz}
\rule{\textwidth}{1mm}
\vspace{0.5cm}
\begin{deflist}{MMMMMMMM}
\item[\bf NAME]
\begin{description}
\item[MUIA\_Scale\_Horiz]  -- (V4 ) [ISG], BOOL
\end{description}

\item[\bf FUNCTION]
Indicate whether you want a horizontal or a
vertical scale.

Defaults to horizontal.

\item[\bf BUGS]
Currently, only the horizontal scale is implemented.

\item[\bf EXAMPLE]
\begin{flushleft}
\begin{verbatim}
...
VGroup,
   Child, GaugeObject, End,
   Child, Scaleobject, End,
   End,
...
/* and everythins is fine... */
\end{verbatim}
\end{flushleft}
\item[\bf SEE ALSO]
\end{deflist}


%---------------- End of File ----------
