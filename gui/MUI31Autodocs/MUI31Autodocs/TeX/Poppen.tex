%---------------- Functions ------------

\chapter{Poppen.mui/Poppen.mui}
\rule{\textwidth}{1mm}
\vspace{0.5cm}
\begin{deflist}{MMMMMMMM}
\item[\bf FUNCTION]
Poppen class adds input capabilities to its super class
Pendisplay. It should be used if your application allows
users to configure some custom pens for rendering.

A Poppen object will appear as kind of a button which
displays the currently selected color. When the user hits
the button, a Popup window containing a Penadjust object
opens up and lets the user choose change the color.

You can control the window title of the popup window using
the MUIA\_Window\_Title on the Poppen object. It will remember
its value and use it when creating the popup window.

As most MUI popups, the Penadjust popup window runs
asynchronously and stays there until the user terminates
it with "'OK"' or "'Cancel"'. Furthermore, if the popup window
is automatically cancelled if the pop button receives a
MUIM\_Cleanup method.

Anyway, you dont have to care about the internals of this class.
Just create it like

obj = MUI\_NewObject(MUIC\_Poppen,
        MUIA\_CycleChain  , 1,
        MUIA\_Window\_Title, "'Followed Links Color"',
        TAG\_DONE);

somewhere in your prefs window and everything will be fine. You
can get/set the current color from a Poppen object by using
the MUIA\_Pendisplay\_Spec attribute. The resulting struct
MUI\_SpenSpec may then be saved somewhere in your preferences
and used as parameter for MUI\_ObtainPen() and MUI\_ReleasePen().

You can find some example code on using this class in the
Class2 demo of the MUI distribution.
\end{deflist}


%---------------- End of File ----------
