%---------------- Functions ------------

\chapter{Radio.mui/Radio.mui}
\rule{\textwidth}{1mm}
\vspace{0.5cm}
\begin{deflist}{MMMMMMMM}
\item[\bf FUNCTION]
Radio class generates radio button gadgets.
They do the same job as cycle gadgets and
eat up more window space, maybe that's the reason
why so few of them can be found in existing
applications.
\end{deflist}


\subsection{Radio.mui/MUIA\_Radio\_Active}
\rule{\textwidth}{1mm}
\vspace{0.5cm}
\begin{deflist}{MMMMMMMM}
\item[\bf NAME]
\begin{description}
\item[MUIA\_Radio\_Active]  -- (V4 ) [ISG], LONG
\end{description}

\item[\bf FUNCTION]
This attributes defines the number of the active
entry in the radio gadgets. Valid range is from
0 for the first entry to NumEntries-1 for the last.

Setting MUIA\_Radio\_Active causes the gadget to
be updated. On the other hand, when the user
plays around with the gadget, MUIA\_Radio\_Active
will always reflects the current state.

\item[\bf EXAMPLE]
\begin{flushleft}
\begin{verbatim}
set(radioobj,MUIA_Radio_Active,3);
\end{verbatim}
\end{flushleft}
\item[\bf SEE ALSO]
MUIA\_Radio\_Entries
\end{deflist}


\subsection{Radio.mui/MUIA\_Radio\_Entries}
\rule{\textwidth}{1mm}
\vspace{0.5cm}
\begin{deflist}{MMMMMMMM}
\item[\bf NAME]
\begin{description}
\item[MUIA\_Radio\_Entries]  -- (V4 ) [I..], STRPTR *
\end{description}

\item[\bf FUNCTION]
Here you can define what entries shall be displayed
in your radio gadget. You must supply a pointer to
a string array, containing one entry for each item
and terminated with a NULL.

Remember that radio gadget entries may contain any
text formatting code such as bold, italic or
underlined characters.

\item[\bf EXAMPLES]
\begin{flushleft}
\begin{verbatim}
static const char *RA_GroupTitleColor[] =
{
   "normal",
   "highlight",
   "3-dimensional",
\end{verbatim}
\end{flushleft}
\item[\bf NULL]
\begin{flushleft}
\begin{verbatim}
};

 CY_Title = RadioObject,
   MUIA_Radio_Entries, RA_GroupTitleColor,
   End;
\end{verbatim}
\end{flushleft}
\item[\bf SEE ALSO]
MUIA\_Radio\_Active, MUIA\_Text\_Contents
\end{deflist}


%---------------- End of File ----------
