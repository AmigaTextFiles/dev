%---------------- Functions ------------

\chapter{Cycle.mui/Cycle.mui}
\rule{\textwidth}{1mm}
\vspace{0.5cm}
\begin{deflist}{MMMMMMMM}
\item[\bf FUNCTION]
Cycle class generates the well known cycle gadgets.
However, MUI cycle gadgets feature a (configurable)
popup menu to avoid clicking through many entries.
\end{deflist}


\subsection{Cycle.mui/MUIA\_Cycle\_Active}
\rule{\textwidth}{1mm}
\vspace{0.5cm}
\begin{deflist}{MMMMMMMM}
\item[\bf NAME]
\begin{description}
\item[MUIA\_Cycle\_Active]  -- (V4 ) [ISG], LONG
\end{description}

\item[\bf SPECIAL INPUTS]
\begin{flushleft}
\begin{verbatim}
MUIV_Cycle_Active_Next
MUIV_Cycle_Active_Prev
\end{verbatim}
\end{flushleft}
\item[\bf FUNCTION]
This attributes defines the number of the active
entry in the cycle gadgets. Valid range is from
0 for the first entry to NumEntries-1 for the last.

Setting MUIA\_Cycle\_Active causes the gadget to
be updated. On the other hand, when the user
plays around with the gadget, MUIA\_Cycle\_Active
will always reflects the current state.

Using MUIV\_Cycle\_Active\_Next and MUIV\_Cycle\_Active\_Prev
as attribute value during set causes the gadget to
cycle through its entries in the given direction.

\item[\bf EXAMPLE]
\begin{flushleft}
\begin{verbatim}
set(cycleobj,MUIA_Cycle_Active,3);
\end{verbatim}
\end{flushleft}
\item[\bf SEE ALSO]
MUIA\_Cycle\_Entries
\end{deflist}


\subsection{Cycle.mui/MUIA\_Cycle\_Entries}
\rule{\textwidth}{1mm}
\vspace{0.5cm}
\begin{deflist}{MMMMMMMM}
\item[\bf NAME]
\begin{description}
\item[MUIA\_Cycle\_Entries]  -- (V4 ) [I..], STRPTR *
\end{description}

\item[\bf FUNCTION]
Here you can define what entries shall be displayed
in your cycle gadget. You must supply a pointer to
a string array, containing one entry for each item
and terminated with a NULL.

Remember that cycle gadget entries may contain any
text formatting code such as bold, italic or
underlined characters.

Cycle gadgets set the preparse string for all entries
to "'$\backslash$33c"', this means that they will automatically
appear centered. Of course you can override this by
simply preceding your entries with own formatting code.

\item[\bf EXAMPLES]
\begin{flushleft}
\begin{verbatim}
static const char *CYA_GroupTitleColor[] =
{
   "normal",
   "highlight",
   "3-dimensional",
\end{verbatim}
\end{flushleft}
\item[\bf NULL]
\begin{flushleft}
\begin{verbatim}
};

 CY_Title = CycleObject,
   MUIA_Cycle_Entries, CYA_GroupTitleColor,
   End;
\end{verbatim}
\end{flushleft}
\item[\bf SEE ALSO]
MUIA\_Cycle\_Active, MUIA\_Text\_Contents
\end{deflist}


%---------------- End of File ----------
