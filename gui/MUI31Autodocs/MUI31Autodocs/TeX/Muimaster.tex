%---------------- Functions ------------

\chapter{muimaster.library/--background--}
\rule{\textwidth}{1mm}
\vspace{0.5cm}
\begin{deflist}{MMMMMMMM}
\item[\bf PURPOSE]
\begin{flushleft}
\begin{verbatim}
muimaster.library contains functions for creating and diposing
objects, for requester handling and for controlling custom
classes. Additionally, several of the standard MUI classes are
built into muimaster.library. For you as a programmer, there
is no difference between using a builtin class or an external
class coming as "mui:libs/mui/<foobar>.mui". The MUI object
generation call takes care of this situation and loads
external classes automatically when they are needed.
\end{verbatim}
\end{flushleft}
\end{deflist}


\subsection{muimaster.library/MUI\_AllocAslRequest}
\rule{\textwidth}{1mm}
\vspace{0.5cm}
\begin{deflist}{MMMMMMMM}
\item[\bf NAME]
MUI\_AllocAslRequest

\item[\bf FUNCTION]
Provide an interface to asl.library. Using this ensures
your application will benefit from future expansions
 to MUI's window and iconification handling.

\item[\bf SEE ALSO]
asl.library/AllocAslRequest

\end{deflist}


\subsection{muimaster.library/MUI\_AslRequest}
\rule{\textwidth}{1mm}
\vspace{0.5cm}
\begin{deflist}{MMMMMMMM}
\item[\bf NAME]
MUI\_AslRequest

\item[\bf FUNCTION]
Provide an interface to asl.library. Using this ensures
your application will benefit from future expansions
 to MUI's window and iconification handling.

\item[\bf SEE ALSO]
asl.library/AslRequest

\end{deflist}


\subsection{muimaster.library/MUI\_DisposeObject}
\rule{\textwidth}{1mm}
\vspace{0.5cm}
\begin{deflist}{MMMMMMMM}
\item[\bf NAME]
\begin{description}
\item[MUI\_DisposeObject]  -- Delete a MUI object.
\end{description}

\item[\bf SYNOPSIS]
\begin{flushleft}
\begin{verbatim}
MUI_DisposeObject( object )
                   A0

VOID�MUI_DisposeObject( APTR );
\end{verbatim}
\end{flushleft}
\item[\bf �  FUNCTION]
\begin{flushleft}
\begin{verbatim}
Deletes a MUI object and all of it's auxiliary data.
These objects are all created by MUI_NewObject() or NewObject().
Objects of certain classes "own" other objects, which will also
be deleted when the object is passed to MUI_DisposeObject().
Read the per-class documentation carefully to be aware
of these instances.
\end{verbatim}
\end{flushleft}
\item[\bf INPUTS]
object = abstract pointer to a MUI object returned by
         MUI\_NewObject(). The pointer may be NULL, in which case
         this function has no effect.

\item[\bf RESULT]
None.

\item[\bf SEE ALSO]
MUI\_NewObjectA(), SetAttrs(), GetAttr().

\end{deflist}


\subsection{muimaster.library/MUI\_Error}
\rule{\textwidth}{1mm}
\vspace{0.5cm}
\begin{deflist}{MMMMMMMM}
\item[\bf NAME]
\begin{description}
\item[MUI\_Error]  -- Return extra information from the MUI system.
\end{description}

\item[\bf SYNOPSIS]
\begin{flushleft}
\begin{verbatim}
LONG MUI_Error(VOID);
\end{verbatim}
\end{flushleft}
\item[\bf �  FUNCTION]
\begin{flushleft}
\begin{verbatim}
Obsolete. Better use SetIoErr()/IoErr().
\end{verbatim}
\end{flushleft}
\item[\bf SEE ALSO]
MUI\_SetError()

\end{deflist}


\subsection{muimaster.library/MUI\_FreeAslRequest}
\rule{\textwidth}{1mm}
\vspace{0.5cm}
\begin{deflist}{MMMMMMMM}
\item[\bf NAME]
MUI\_FreeAslRequest

\item[\bf FUNCTION]
Provide an interface to asl.library. Using this ensures
your application will benefit from future expansions
 to MUI's window and iconification handling.

\item[\bf SEE ALSO]
asl.library/FreeAslRequest

\end{deflist}


\subsection{muimaster.library/MUI\_FreeClass}
\rule{\textwidth}{1mm}
\vspace{0.5cm}
\begin{deflist}{MMMMMMMM}
\item[\bf NAME]
\begin{description}
\item[MUI\_FreeClass]  -- Free class.
\end{description}

\item[\bf SYNOPSIS]
\begin{flushleft}
\begin{verbatim}
MUI_FreeClass( classptr )
               A0

VOID MUI_FreeClass(struct IClass *classptr);
\end{verbatim}
\end{flushleft}
\item[\bf �  FUNCTION]
\begin{flushleft}
\begin{verbatim}
This function is obsolete since MUI V8.
Use MUI_DeleteCustomClass() instead.
\end{verbatim}
\end{flushleft}
\item[\bf SEE ALSO]
MUI\_CreateCustomClass(), MUI\_DeleteCustomClass()

\end{deflist}


\subsection{muimaster.library/MUI\_GetClass}
\rule{\textwidth}{1mm}
\vspace{0.5cm}
\begin{deflist}{MMMMMMMM}
\item[\bf NAME]
\begin{description}
\item[MUI\_GetClass]  -- Get a pointer to a MUI class.
\end{description}

\item[\bf SYNOPSIS]
\begin{flushleft}
\begin{verbatim}
class = MUI_GetClass( classid )
D0                    A0

struct IClass * MUI_GetClass(char *classid);
\end{verbatim}
\end{flushleft}
\item[\bf �  FUNCTION]
\begin{flushleft}
\begin{verbatim}
This function is obsolete since MUI V8.
Use MUI_CreateCustomClass instead.
\end{verbatim}
\end{flushleft}
\item[\bf SEE ALSO]
MUI\_CreateCustomClass(), MUI\_DeleteCustomClass()

\end{deflist}


\subsection{muimaster.library/MUI\_MakeObjectA}
\rule{\textwidth}{1mm}
\vspace{0.5cm}
\begin{deflist}{MMMMMMMM}
\item[\bf NAME]
\begin{description}
\item[MUI\_MakeObjectA]  -- create an object from the builtin object collection.
\item[MUI\_MakeObject]  -- Varargs stub for MUI\_MakeObjectA
\end{description}

\item[\bf SYNOPSIS]
\begin{flushleft}
\begin{verbatim}
object = MUI_MakeObjectA( objtype, params )
D0                        D0       A0

Object * MUI_MakeObjectA(ULONG type, ULONG *params);

Object * MUI_MakeObject(ULONG type, ...);
\end{verbatim}
\end{flushleft}
\item[\bf �  FUNCTION]
\begin{flushleft}
\begin{verbatim}
Prior to muimaster.library V8, MUI was distributed with several macros
to help creating often used objects. This practice was easy, but using
lots of these macros often resulted in big programs. Now, muimaster
library contains an object library with several often used objects
already built in.

MUI_MakeObject() takes the type of the object as first parameter and
a list of additional (type specific) parameters. Note that these
additional values are *not* a taglist!

See the header file mui.h for documentation on object types and the
required parameters.
\end{verbatim}
\end{flushleft}
\item[\bf SEE ALSO]
MUI\_CreateCustomClass(), MUI\_DeleteCustomClass()

\end{deflist}


\subsection{muimaster.library/MUI\_NewObjectA}
\rule{\textwidth}{1mm}
\vspace{0.5cm}
\begin{deflist}{MMMMMMMM}
\item[\bf NAME]
\begin{description}
\item[MUI\_NewObjectA]  -- Create an object from a class.
\item[MUI\_NewObject]  -- Varargs stub for MUI\_NewObjectA().
\end{description}

\item[\bf SYNOPSIS]
\begin{flushleft}
\begin{verbatim}
object = MUI_NewObjectA( class, tags )
D0                       A0     A1

APTR�MUI_NewObjectA( char *, struct TagItem * );

object = MUI_NewObject( classID, Tag1, ... )

APTR MUI_NewObject( classID, ULONG, ... );
\end{verbatim}
\end{flushleft}
\item[\bf �  FUNCTION]
\begin{flushleft}
\begin{verbatim}
This is the general method of creating objects from MUI classes.
You specify a class by its ID string. If the class is not
already in memory or built into muimaster.library, it will be
loaded using OpenLibrary("mui/%s",0).

You further specify initial "create-time" attributes for the
object via a TagItem list, and they are applied to the resulting
generic data object that is returned. The attributes, their meanings,
attributes applied only at create-time, and required attributes
are all defined and documented on a class-by-class basis.
\end{verbatim}
\end{flushleft}
\item[\bf INPUTS]
classID = the name/ID string of a MUI class, e.g. "'Image.mui"'.
          Class names are case sensitive!

tagList = pointer to array of TagItems containing attribute/value
          pairs to be applied to the object being created.

\item[\bf RESULT]
A MUI object, which may be used in different contexts such
as an application, window or gadget, and may be manipulated
by generic functions. You eventually free the object using
MUI\_DisposeObject().
NULL indicates failure, more information on the error can be
obtained with MUI\_Error().

\item[\bf BUGS]
\item[\bf SEE ALSO]
MUI\_DisposeObject(), MUI\_Error(), SetAttrs(), GetAttr().

\end{deflist}


\subsection{muimaster.library/MUI\_ObtainPen}
\rule{\textwidth}{1mm}
\vspace{0.5cm}
\begin{deflist}{MMMMMMMM}
\item[\bf NAME]
\begin{description}
\item[MUI\_ObtainPen]  -- Obtain a drawing pen.
\end{description}

\item[\bf SYNOPSIS]
\begin{flushleft}
\begin{verbatim}
MUI_ObtainPen ( mri , spec , flags )
                A0    A1     D0

LONG MUI_ObtainPen
(
        struct MUI_RenderInfo *mri,
        struct MUI_PenSpec *spec,
        ULONG flags
);
\end{verbatim}
\end{flushleft}
\item[\bf FUNCTION]
Whenever your application needs custom drawing pens, you should allow
your user to adjust them with a Poppen class. The result from this
Poppen class is a struct MUI\_PenSpec which you can save somewhere in
your preferences and use together with MUI\_ObtainPen(),
MUI\_ReleasePen() and the MUIPEN() macro to transform the spec into a
pen number useful for SetAPen().

\item[\bf NOTE]
This function will work under 2.x but will generally perform better
under 3.x and above.

\item[\bf EXAMPLE]
\begin{flushleft}
\begin{verbatim}
See demo program Class2.c
\end{verbatim}
\end{flushleft}
\item[\bf SEE ALSO]
MUI\_ReleasePen(), Poppen.mui



\end{deflist}


\subsection{muimaster.library/MUI\_RejectIDCMP}
\rule{\textwidth}{1mm}
\vspace{0.5cm}
\begin{deflist}{MMMMMMMM}
\item[\bf NAME]
\begin{description}
\item[MUI\_RejectIDCMP]  -- Reject previously requested input events.
\end{description}

\item[\bf SYNOPSIS]
\begin{flushleft}
\begin{verbatim}
MUI_RejectIDCMP( obj, flags )
                  A0   D0

VOID MUI_RejectIDCMP( Object *obj, ULONG flags );
\end{verbatim}
\end{flushleft}
\item[\bf �  FUNCTION]
\begin{flushleft}
\begin{verbatim}
Reject previously requested input events. You should
ensure that you reject all input events you requested
for an object before it gets disposed. Rejecting
flags that you never requested has no effect.

Critical flags such as IDCMP_MOUSEMOVE and IDCMP_INTUITICKS
should be rejected as soon as possible. See MUI_RequestIDCMP()
for details.
\end{verbatim}
\end{flushleft}
\item[\bf INPUTS]
\begin{description}
\item[obj  ]  - pointer to yourself as an object.
\item[flags]  - one or more IDCMP\_XXXX flags.
\end{description}

\item[\bf EXAMPLE]
\begin{flushleft}
\begin{verbatim}
LONG CleanupMethod(struct IClass *cl, Object *obj, Msg msg)
{
   MUI_RejectIDCMP( obj, IDCMP_MOUSEBUTTONS|IDCMP_RAWKEY );
   return(DoSuperMethodA(cl,obj,msg));
}
\end{verbatim}
\end{flushleft}
\item[\bf SEE ALSO]
MUI\_RequestIDMCP()

\end{deflist}


\subsection{muimaster.library/MUI\_ReleasePen}
\rule{\textwidth}{1mm}
\vspace{0.5cm}
\begin{deflist}{MMMMMMMM}
\item[\bf NAME]
\begin{description}
\item[MUI\_ObtainPen]  -- Obtain a drawing pen.
\end{description}

\item[\bf SYNOPSIS]
\begin{flushleft}
\begin{verbatim}
MUI_ReleasePen ( mri , pen )
                 A0    D0

LONG MUI_ReleasePen
(
        struct MUI_RenderInfo *mri,
        ULONG pen
);
\end{verbatim}
\end{flushleft}
\item[\bf FUNCTION]
Release a pen obtained with MUI\_ObtainPen(). Usually placed
in the Cleanup method of custom classes.

\item[\bf NOTE]
Do *not* use the MUIPEN() maros when releasing pens!

\item[\bf EXAMPLE]
\begin{flushleft}
\begin{verbatim}
See demo program Class2.c
\end{verbatim}
\end{flushleft}
\item[\bf SEE ALSO]
MUI\_ObtainPen(), Poppen.mui



\end{deflist}


\subsection{muimaster.library/MUI\_RequestIDCMP}
\rule{\textwidth}{1mm}
\vspace{0.5cm}
\begin{deflist}{MMMMMMMM}
\item[\bf NAME]
\begin{description}
\item[MUI\_RequestIDCMP]  -- Request input events for your custom class.
\end{description}

\item[\bf SYNOPSIS]
\begin{flushleft}
\begin{verbatim}
MUI_RequestIDCMP( obj, flags )
                  A0   D0

VOID MUI_RequestIDCMP( Object *obj, ULONG flags );
\end{verbatim}
\end{flushleft}
\item[\bf �  FUNCTION]
\begin{flushleft}
\begin{verbatim}
If your custom class needs to do some input handling, you must
explicitly request the events you want to receive. You can
request (and reject) events at any time.

Whenever an input event you requested arrives at your parent
windows message port, your object will receive a
MUIM_HandleInput method.

Note: Time consuming IDCMP flags such as IDCMP_INTUITICKS and
      IDCMP_MOUSEMOVE should be handled with care. Too many
      objects receiving them will degrade performance With
      the following paragraph in mind, this isn't really
      a problem:

      You should try to request critical events only when you
      really need them and reject them with MUI_RejectIDCMP()
      as soon as possible. Usually, mouse controlled objects
      only need MOUSEMOVES and INTUITICKS when a button
      is pressed. You should request these flags only
      on demand, i.e. after receiving a mouse down event
      and reject them immediately after the button has been
      released.
\end{verbatim}
\end{flushleft}
\item[\bf INPUTS]
\begin{description}
\item[obj  ]  - pointer to yourself as an object.
\item[flags]  - one or more IDCMP\_XXXX flags.
\end{description}

\item[\bf EXAMPLE]
\begin{flushleft}
\begin{verbatim}
LONG SetupMethod(struct IClass *cl, Object *obj, Msg msg)
{
   if (!DoSuperMethodA(cl,obj,msg))
      return(FALSE);

   /* do some setup here... */
   ...;

   /* i need mousebutton events and keyboard */
   MUI_RequestIDCMP( obj, IDCMP_MOUSEBUTTONS|IDCMP_RAWKEY );

   return(TRUE);
}
\end{verbatim}
\end{flushleft}
\item[\bf SEE ALSO]
MUI\_RejectIDMCP()

\end{deflist}


\subsection{muimaster.library/MUI\_SetError}
\rule{\textwidth}{1mm}
\vspace{0.5cm}
\begin{deflist}{MMMMMMMM}
\item[\bf NAME]
\begin{description}
\item[MUI\_SetError]  -- Set an error value.
\end{description}

\item[\bf SYNOPSIS]
\begin{flushleft}
\begin{verbatim}
VOID MUI_SetError(LONG);
\end{verbatim}
\end{flushleft}
\item[\bf �  FUNCTION]
\begin{flushleft}
\begin{verbatim}
Obsolete. Better use SetIoErr()/IoErr().
\end{verbatim}
\end{flushleft}
\item[\bf SEE ALSO]
MUI\_Error()
\end{deflist}


%---------------- End of File ----------
