%---------------- Functions ------------

\chapter{Bodychunk.mui/Bodychunk.mui}
\rule{\textwidth}{1mm}
\vspace{0.5cm}
\begin{deflist}{MMMMMMMM}
\item[\bf FUNCTION]
Big and colorful images (e.g. About-Logos) usually take lots
of space when stored in a traditional BitMap structure. To
save memory, you can decide to have the picture compressed
in your code and use the Bodychunk class instead of the
Bitmap class for displaying. MUI will then automatically
decompress your image when its about to appear in a
window.

Since Bodychunk class is a subclass of Bitmap class, you can
of course use all the Bitmaps remapping and transparency
features.
\end{deflist}


\subsection{Bodychunk.mui/MUIA\_Bodychunk\_Body}
\rule{\textwidth}{1mm}
\vspace{0.5cm}
\begin{deflist}{MMMMMMMM}
\item[\bf NAME]
\begin{description}
\item[MUIA\_Bodychunk\_Body]  -- (V8 ) [ISG], UBYTE *
\end{description}

\item[\bf FUNCTION]
Specify a pointer to the BODY data of your picture. This BODY
data must follow normal IFF/ILBM conventions.

You have to supply MUIA\_Bitmap\_Width, MUIA\_Bitmap\_Height and
MUIA\_Bodychunk\_Depth to describe the contents of the BODY
data, otherwise MUI will fail to decompress it.

\item[\bf SEE ALSO]
MUIA\_Bodychunk\_Depth, MUIA\_Bodychunk\_Compression,
MUIA\_Bodychunk\_Masking
\end{deflist}


\subsection{Bodychunk.mui/MUIA\_Bodychunk\_Compression}
\rule{\textwidth}{1mm}
\vspace{0.5cm}
\begin{deflist}{MMMMMMMM}
\item[\bf NAME]
\begin{description}
\item[MUIA\_Bodychunk\_Compression]  -- (V8 ) [ISG], UBYTE
\end{description}

\item[\bf FUNCTION]
MUI is able to uncompress byte\&run compressed BODY chunks
automatically. If your data is compressed, you must
supply a value of cmpByteRun1 (==1) for this tag. Other
compression techniques are not supported.

Omitting this tag or setting it to 0 indicates that the
BODY data is uncompressed. Using the Bodychunk class
doesn't make much sense in this case since its main
purpose is to save memory for big images.

\item[\bf SEE ALSO]
MUIA\_Bodychunk\_Masking, MUIA\_Bodychunk\_Body
\end{deflist}


\subsection{Bodychunk.mui/MUIA\_Bodychunk\_Depth}
\rule{\textwidth}{1mm}
\vspace{0.5cm}
\begin{deflist}{MMMMMMMM}
\item[\bf NAME]
\begin{description}
\item[MUIA\_Bodychunk\_Depth]  -- (V8 ) [ISG], LONG
\end{description}

\item[\bf FUNCTION]
Specify the depth of your picture here. This tag is
required for correct BODY chunk parsing. Also remember
to use MUIA\_Bodychunk\_Masking if your BODY data contains
a masking bitplane.

\item[\bf SEE ALSO]
MUIA\_Bodychunk\_Body, MUIA\_Bodychunk\_Masking
\end{deflist}


\subsection{Bodychunk.mui/MUIA\_Bodychunk\_Masking}
\rule{\textwidth}{1mm}
\vspace{0.5cm}
\begin{deflist}{MMMMMMMM}
\item[\bf NAME]
\begin{description}
\item[MUIA\_Bodychunk\_Masking]  -- (V8 ) [ISG], UBYTE
\end{description}

\item[\bf FUNCTION]
You must indicate if your BODY data contains a masking
plane. Currently, MUI does not use this masking plane
for any purpose, but this attribute is required to
allow correct parsing of the BODY data.

\item[\bf SEE ALSO]
MUIA\_Bodychunk\_Body, MUIA\_Bodychunk\_Compression
\end{deflist}


%---------------- End of File ----------
