%---------------- Functions ------------

\chapter{Numeric.mui/Numeric.mui}
\rule{\textwidth}{1mm}
\vspace{0.5cm}
\begin{deflist}{MMMMMMMM}
\item[\bf FUNCTION]
Numeric class is the base class for everything that
deals with the input (and display) of integer numbers.
Numeric class itself does not feature any GUI elements, it
just offers some basic attributes and methods which are
common to all types of sliders. Creating direct instances
of this class usually doesn't make any sense. Instead, use
one of the included subclasses like Slider.mui,
Numericbutton.mui or Knob.mui to select the type of gadget
you need.

Numeric class and the supplied subclasses communicate with a
set of methods. By writing subclasses which override some of
them, you can change the behaviour of all sliders to fit
your requirements. You could e.g. enhance the builtin value
formatting code which is limited to simple printf-style
strings by replacing the MUIM\_Numeric\_Stringify method with
something more complicated. Or you turn your sliders to
logarythmic scales by replacing MUIM\_Numeric\_ValueToScale
and MUIM\_Numeric\_ScaleToValue.

Imagine you would like a slider which doesn't only display a
users age but also makes comments depending on the current
value, e.g. "'13 years (Teenie)"' ... "'25 years (Twen)"' ...
All you have to do is to write a subclass of any of MUI's
builtin slider types which does nothing but replace
MUIM\_Stringify with your code. See the supplied "'Slidorama"'
demo program to see how this might work.

MUI features several different subclasses of Numeric.mui:
Slider.mui creates an ordinary slider like the ones of
previous MUI releases. Numericbutton.mui creates a
space-saving slider, only the value is shown in the user
interface as kind of a button. When the user clicks on this
button, a slider pops up to adjust the value. Knob.mui
displays a very nice designed turning wheel but also offers
popup possibilities for those who don't like turning a knob
with the mouse. "'Slidorama"' demo shows everything that
is available.

All slider gadgets offer configuration options and make it
easy for the user to enable things he likes and disable
things he dislikes. However, it's your choice to decide
which basic type of gadget shall be used.

If really none of the supplied subclasses of Numeric.mui
suits your requirements, you may of course write custom
classes for numeric data input. If you use Numeric.mui as
base class, you won't need to think about the basic stuff
like min and max values and formatting.

Keyboard control (TAB, cursor keys, MUIA\_ControlChar) is
handled my Numeric.mui automatically, subclasses will not
have to care about it.
\end{deflist}


\subsection{Numeric.mui/MUIA\_Numeric\_Default}
\rule{\textwidth}{1mm}
\vspace{0.5cm}
\begin{deflist}{MMMMMMMM}
\item[\bf NAME]
\begin{description}
\item[MUIA\_Numeric\_Default]  -- (V11) [ISG], LONG
\end{description}

\item[\bf FUNCTION]
Adjust the default value for a numeric input/display gadget.
When the object receives a MUIM\_Numeric\_SetDefault method,
it sets its value to the one given here.

Each type of slider can have a default value to which the
user can always return immediately by some action depending
on the implementation of the subclass. Knob.mui e.g. resets
to defaults after a double click in the knob area.

The default value can also be reached by pressing the
toggle key (usually SPACE) on an active numeric gadget.

MUIA\_Numeric\_Default defaults to 0.

\item[\bf SEE ALSO]
MUIA\_Numeric\_Max, MUIA\_Numeric\_Value, MUIA\_Numeric\_Min,
MUIM\_Numeric\_SetDefault
\end{deflist}


\subsection{Numeric.mui/MUIA\_Numeric\_Format}
\rule{\textwidth}{1mm}
\vspace{0.5cm}
\begin{deflist}{MMMMMMMM}
\item[\bf NAME]
\begin{description}
\item[MUIA\_Numeric\_Format]  -- (V11) [ISG], STRPTR
\end{description}

\item[\bf FUNCTION]
printf-style string to describe the format of the slider
display.

Whenever a subclass of Numeric.mui thinks its time to render
a new value, it doesn't simply write it to a string but
instead calls MUIM\_Numeric\_Stringify. This method of Numeric
class looks for the specified MUIA\_Numeric\_Format in its data
structures and fills a string with the current value.
In detail, things work like his:

- Some slider object (e.g. a knob) receives a MUIM\_Draw method.
- The MUIM\_Draw implementation of the knob object reads
  the current value of the Numeric class and calls
  MUIM\_Numeric\_Stringify with this valuse.
- MUIM\_Numeric\_Stringify of numeric class reads the current
  format and sprintf()s the given value to a buffer. The
  buffer is returned the caller.
- After all this stuff, the MUIM\_Draw implementation
  receives a nice string as result code and finally puts
  it somewhere into the window.

All this method stuff might sound a bit crazy, but in fact its
quite powerful. If you write a subclass of any of MUI's slider
classes which simply replaces MUIM\_Numeric\_Stringify with
your own code, you can create any string you like for display
in these sliders. You might e.g. want to display a nice formatted
time string (hh:mm:ss) in a slider knob which adjusts a number
of seconds. Or you need to adjust a baudrate from a hand of
predefined values. Just overrided MUIM\_Numeric\_Stringify and
you have the choice how the slider value translates into
a string.

If you dont override MUIM\_Numeric\_Stringify, the method
reaches Numeric class which simply does a sprintf() with
the defined MUIA\_Numeric\_Format.

Note well: The maximum length of the result string for
MUIA\_Numeric\_Format is limited to 32 characters. If you
need more, you *must* override the method.

MUIA\_Numeric\_Format defaults to "'\%ld"'.

\item[\bf SEE ALSO]
MUIA\_Numeric\_Max, MUIA\_Numeric\_Value, MUIA\_Numeric\_Min,
MUIM\_Numeric\_Stringify, MUIM\_Numeric\_StringifyValue
\end{deflist}


\subsection{Numeric.mui/MUIA\_Numeric\_Max}
\rule{\textwidth}{1mm}
\vspace{0.5cm}
\begin{deflist}{MMMMMMMM}
\item[\bf NAME]
\begin{description}
\item[MUIA\_Numeric\_Max]  -- (V11) [ISG], LONG
\end{description}

\item[\bf FUNCTION]
Adjust the maximum value for a numeric input/display
gadget.

Numeric class will automatically clip its value to make
it fit between MUIA\_Numeric\_Min and MUI\_Numeric\_Max. Also,
minimum and maximum values are used for several internal
calculations such as the maximum space required to display
a numeric value.

You may change MUIA\_Numeric\_Min and MUIA\_Numeric\_Max with
SetAttrs(), but current MUI versions will *not* update
the objects and with the windows width or height if your
change causes the value display to change it's minimum
and/or maximum pixel sizes. The slider position itself
will be updated though.

MUI treats all values in numeric class as signed
longwords, so that's the limit for all tags.

MUIA\_Numeric\_Max defaults to 100.

\item[\bf NOTES]
This attribute replaces the MUIA\_Slider\_Max tag of
previous MUI releases. For compatibility reasons,
both have the same hex value. Nevertheless, always
use MUIA\_Numeric\_Max in new code.

\item[\bf SEE ALSO]
MUIA\_Numeric\_Min, MUIA\_Numeric\_Value, MUIA\_Numeric\_Default,
MUIM\_Numeric\_GetPixelSize, MUIM\_Numeric\_ValueToScale,
MUIM\_Numeric\_ScaleToValue
\end{deflist}


\subsection{Numeric.mui/MUIA\_Numeric\_Min}
\rule{\textwidth}{1mm}
\vspace{0.5cm}
\begin{deflist}{MMMMMMMM}
\item[\bf NAME]
\begin{description}
\item[MUIA\_Numeric\_Min]  -- (V11) [ISG], LONG
\end{description}

\item[\bf FUNCTION]
Adjust the minimum value for a numeric input/display
gadget.

Numeric class will automatically clip its value to make
it fit between MUIA\_Numeric\_Min and MUI\_Numeric\_Max. Also,
minimum and maximum values are used for several internal
calculations such as the maximum space required to display
a numeric value.

You may change MUIA\_Numeric\_Min and MUIA\_Numeric\_Max with
SetAttrs(), but current MUI versions will *not* update
the objects and with the windows width or height if your
change causes the value display to change it's minimum
and/or maximum pixel sizes. The slider position itself
will be updated though.

MUI treats all values in numeric class as signed
longwords, so that's the limit for all tags.

MUIA\_Numeric\_Min defaults to 0.

\item[\bf NOTES]
This attribute replaces the MUIA\_Slider\_Min tag of
previous MUI releases. For compatibility reasons,
both have the same hex value. Nevertheless, always
use MUIA\_Numeric\_Min in new code.

\item[\bf SEE ALSO]
MUIA\_Numeric\_Max, MUIA\_Numeric\_Value, MUIA\_Numeric\_Default,
MUIM\_Numeric\_GetPixelSize, MUIM\_Numeric\_ValueToScale,
MUIM\_Numeric\_ScaleToValue
\end{deflist}


\subsection{Numeric.mui/MUIA\_Numeric\_RevLeftRight}
\rule{\textwidth}{1mm}
\vspace{0.5cm}
\begin{deflist}{MMMMMMMM}
\item[\bf NAME]
\begin{description}
\item[MUIA\_Numeric\_RevLeftRight]  -- (V11) [ISG], BOOL
\end{description}

\item[\bf FUNCTION]
Reverse the function of left/right keys.

Under some circumstances it might be desirable to
reverse the keyboard control for a slider gadget.
This tag might help.

MUIA\_Numeric\_RevLeftRight defaults to FALSE.

\item[\bf SEE ALSO]
MUIA\_Numeric\_Max, MUIA\_Numeric\_Value, MUIA\_Numeric\_Default,
MUIA\_Numeric\_Reverse, MUIA\_Numeric\_RevUpDown
\end{deflist}


\subsection{Numeric.mui/MUIA\_Numeric\_RevUpDown}
\rule{\textwidth}{1mm}
\vspace{0.5cm}
\begin{deflist}{MMMMMMMM}
\item[\bf NAME]
\begin{description}
\item[MUIA\_Numeric\_RevUpDown]  -- (V11) [ISG], BOOL
\end{description}

\item[\bf FUNCTION]
Reverse the function of up/down keys.

Under some circumstances it might be desirable to
reverse the keyboard control for a slider gadget.
This tag might help.

MUIA\_Numeric\_RevUpDown defaults to FALSE.

\item[\bf SEE ALSO]
MUIA\_Numeric\_Max, MUIA\_Numeric\_Value, MUIA\_Numeric\_Default,
MUIA\_Numeric\_Reverse, MUIA\_Numeric\_RevUpDown
\end{deflist}


\subsection{Numeric.mui/MUIA\_Numeric\_Reverse}
\rule{\textwidth}{1mm}
\vspace{0.5cm}
\begin{deflist}{MMMMMMMM}
\item[\bf NAME]
\begin{description}
\item[MUIA\_Numeric\_Reverse]  -- (V11) [ISG], BOOL
\end{description}

\item[\bf FUNCTION]
Reverse the display of a numeric gadget.

When set to TRUE, the MUIM\_Numeric\_ScaleToValue
and MUIM\_Numeric\_ValueToScale methods are effected
in a way that makes your gadget behave "'reverse"'.
Its minimum numeric value will be mapped to the
maximum scale value of the display and vice versa.

MUIA\_Numeric\_Reverse defaults to FALSE.

\item[\bf NOTES]
This attribute replaces the MUIA\_Slider\_Reverse tag
of previous MUI releases. For compatibility reasons,
both have the same hex value. Nevertheless, always
use MUIA\_Numeric\_Reverse in new code.

\item[\bf SEE ALSO]
MUIA\_Numeric\_Max, MUIA\_Numeric\_Value, MUIA\_Numeric\_Default,
MUIM\_Numeric\_ValueToScale, MUIM\_Numeric\_ScaleToValue
\end{deflist}


\subsection{Numeric.mui/MUIA\_Numeric\_Value}
\rule{\textwidth}{1mm}
\vspace{0.5cm}
\begin{deflist}{MMMMMMMM}
\item[\bf NAME]
\begin{description}
\item[MUIA\_Numeric\_Value]  -- (V11) [ISG], LONG
\end{description}

\item[\bf FUNCTION]
Adjust the current value for a numeric input/display
gadget. Numeric class will automatically clip this
value to make it fit between MUIA\_Numeric\_Min and
MUI\_Numeric\_Max.

Whenever a new value is set, the object receices a
new MUIM\_Draw method to get a chance to update its
display.

MUIA\_Numeric\_Default defaults to 0.

\item[\bf NOTES]
This attribute replaces the MUIA\_Slider\_Level tag of
previous MUI releases. For compatibility reasons,
both have the same hex value. Nevertheless, always
use MUIA\_Numeric\_Value in new code.

\item[\bf SEE ALSO]
MUIA\_Numeric\_Min, MUIA\_Numeric\_Max, MUIA\_Numeric\_Default,
MUIM\_Numeric\_Stringify
\end{deflist}


\subsection{Numeric.mui/MUIM\_Numeric\_Decrease}
\rule{\textwidth}{1mm}
\vspace{0.5cm}
\begin{deflist}{MMMMMMMM}
\item[\bf NAME]
MUIM\_Numeric\_Decrease (V11)

\item[\bf SYNOPSIS]
\begin{flushleft}
\begin{verbatim}
DoMethod(obj,MUIM_Numeric_Decrease,LONG amount);
\end{verbatim}
\end{flushleft}
\item[\bf FUNCTION]
Decrease the value of a numeric class object.

\item[\bf SEE ALSO]
MUIM\_Numeric\_Increase, MUIA\_Numeric\_Value
\end{deflist}


\subsection{Numeric.mui/MUIM\_Numeric\_Increase}
\rule{\textwidth}{1mm}
\vspace{0.5cm}
\begin{deflist}{MMMMMMMM}
\item[\bf NAME]
MUIM\_Numeric\_Increase (V11)

\item[\bf SYNOPSIS]
\begin{flushleft}
\begin{verbatim}
DoMethod(obj,MUIM_Numeric_Increase,LONG amount);
\end{verbatim}
\end{flushleft}
\item[\bf FUNCTION]
Increase the value of a numeric class object.

\item[\bf SEE ALSO]
MUIM\_Numeric\_Decrease, MUIA\_Numeric\_Value
\end{deflist}


\subsection{Numeric.mui/MUIM\_Numeric\_ScaleToValue}
\rule{\textwidth}{1mm}
\vspace{0.5cm}
\begin{deflist}{MMMMMMMM}
\item[\bf NAME]
MUIM\_Numeric\_ScaleToValue (V11)

\item[\bf SYNOPSIS]
\begin{flushleft}
\begin{verbatim}
DoMethod(obj,MUIM_Numeric_ScaleToValue,LONG scalemin, LONG scalemax, LONG scale);
\end{verbatim}
\end{flushleft}
\item[\bf FUNCTION]
This method takes the given sale values and transforms them to
something between the numeric objects min and max values.

\item[\bf RESULT]
The transformed value.
\end{deflist}


\subsection{Numeric.mui/MUIM\_Numeric\_SetDefault}
\rule{\textwidth}{1mm}
\vspace{0.5cm}
\begin{deflist}{MMMMMMMM}
\item[\bf NAME]
MUIM\_Numeric\_SetDefault (V11)

\item[\bf SYNOPSIS]
\begin{flushleft}
\begin{verbatim}
DoMethod(obj,MUIM_Numeric_SetDefault,);
\end{verbatim}
\end{flushleft}
\item[\bf FUNCTION]
This method does nothing but reset the value to its
default. Defaults can be adjusted through
MUIA\_Numeric\_Default.

Only implementors of custom slider classes will need
this method. Sending it from applications doesnt make
any sense.

\item[\bf SEE ALSO]
MUIA\_Numeric\_Value, MUIA\_Numeric\_Default
\end{deflist}


\subsection{Numeric.mui/MUIM\_Numeric\_Stringify}
\rule{\textwidth}{1mm}
\vspace{0.5cm}
\begin{deflist}{MMMMMMMM}
\item[\bf NAME]
MUIM\_Numeric\_Stringify (V11)

\item[\bf SYNOPSIS]
\begin{flushleft}
\begin{verbatim}
DoMethod(obj,MUIM_Numeric_Stringify,LONG value);
\end{verbatim}
\end{flushleft}
\item[\bf FUNCTION]
Call this method in your subclass whenever you want to
translate a value into a string. A pointer to a
string buffer is returned.

Only implementors of custom slider classes will need
this method. Sending it from applications doesnt make
any sense.

\item[\bf EXAMPLE]
\begin{flushleft}
\begin{verbatim}
... somewhere in your Draw method ...

get(obj,MUIA_Numeric_Value,&val);
buf = DoMethod(obj,MUIM_Numeric_Stringify,val);
Text(rp,buf,strlen(buf));
\end{verbatim}
\end{flushleft}
\item[\bf SEE ALSO]
MUIA\_Numeric\_Value, MUIA\_Numeric\_Format
\end{deflist}


\subsection{Numeric.mui/MUIM\_Numeric\_ValueToScale}
\rule{\textwidth}{1mm}
\vspace{0.5cm}
\begin{deflist}{MMMMMMMM}
\item[\bf NAME]
MUIM\_Numeric\_ValueToScale (V11)

\item[\bf SYNOPSIS]
\begin{flushleft}
\begin{verbatim}
DoMethod(obj,MUIM_Numeric_ValueToScale,LONG scalemin, LONG scalemax);
\end{verbatim}
\end{flushleft}
\item[\bf FUNCTION]
This method takes the current value of the numeric object and
transforms it to another scale determined by the parameters.

\item[\bf RESULT]
The transformed value.
\end{deflist}


%---------------- End of File ----------
