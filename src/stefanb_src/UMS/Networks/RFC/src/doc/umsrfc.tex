\input amigatexinfo
\input texinfo
@c %**start of header
@setfilename umsrfc.info
@settitle UMS RFC 1.0 documentation
@setchapternewpage on
@c %**end of header


@comment
@comment  umsrfc.tex V1.0
@comment
@comment  Documentation for UMS RFC (Texinfo format)
@comment
@comment  (c) 1992-1999 Stefan Becker
@comment


@comment ----------------------------------------------------------------
@comment --- Title & Copyright Page (for printed manual)              ---
@comment ----------------------------------------------------------------
@titlepage

@title{UMS RFC}
@subtitle{RFC converters for UMS}
@subtitle{}
@subtitle{Version 1.0}
@subtitle{20 Feb 1999}

@page
@vskip 0pt plus 1filll

@comment ----------------------------------------------------------------
@comment Note to translators:
@comment Translate this page and then copy it to the node "Copyright"
@comment Both pages must be identical!
@comment ----------------------------------------------------------------

@heading COPYRIGHT

Copyright @copyright{} 1992-99 Stefan Becker

Permission is granted to make and distribute verbatim copies of this manual
provided the copyright notice and this permission notice are preserved on
all copies.

@ignore
Permission is granted to process this file by TeX and print the results,
provided the printed document carries a copying permission notice identical to
this one except for the removal of this paragraph (this paragraph not being
relevant to the printed manual).
@end ignore

No program, document, data file or source  code  from  this  software  package,
neither in whole nor in part,  may  be  included  or  used  in  other  software
packages unless it is authorized by a written permission from the author.

@heading NO WARRANTY

There is no warranty for this software package. Although the author  has  tried
to prevent errors he can't guarantee that the  software  package  described  in
this document is 100% reliable. You are therefore using this material  at  your
own risk. The author cannot be made responsible for any damage which is  caused
by using this software package.

@heading DISTRIBUTION

This software package is freely distributable. It  may  be  transfered  to  any
media which is used for the distribution of free software  like  Public  Domain
disk collections, CDROMs, FTP servers or bulletin board systems.

In order to ensure the integrity of this software package distributors should
use the original archive file @file{umsrfc1_0.lha} (file name on Aminet:
@file{umsrfc.lha}). The author cannot be made responsible if this software
package has become unusable due to modifications of the archive contents or of
the archive file itself.

There is no limit on the fee taken by distributors, e.g. for the media costs of
floppy disks, streamer tapes or compact discs, or the process  of  duplicating.
Such limits have proven to be harmful  to  the  idea  of  freely  distributable
software, e.g. the software package was removed instead of reducing  the  price
of a floppy disk below the limit.

Although the author does not impose any limit on these fees he  would  like  to
express his personal opinions on this matter:

@itemize @bullet

@item
This software package should be made  available  to  everyone  free  of  charge
whenever this is possible.

@item
If you have purchased this software package  under  normal  conditions  from  a
Public Domain dealer on a floppy disk and have paid more than 5DM or US $5 then
you have definitely paid too much. Please don't support  this  improper  profit
making any longer and switch to a cheaper source as soon as possible.

@end itemize

@heading USAGE RESTRICTIONS

No program, document, data file or source  code  from  this  software  package,
neither in whole nor in part, may be used on any machine which is used

@itemize @bullet

@item
for the research, development, construction, testing or production  of  weapons
or other military applications. This also includes any machine which is used in
the education for any of the above mentioned purposes.

@item
by people who accept, support  or  use  violence  against  other  people,  e.g.
citizens from foreign countries.

@end itemize

@end titlepage
@headings double
@comment ----------------------------------------------------------------





@comment ----------------------------------------------------------------
@comment --- Top Node (not printed)                                   ---
@comment ----------------------------------------------------------------
@ifinfo
@node Top, Copyright, (dir), (dir)
@top UMS RFC Documentation

Welcome to UMS RFC a package for @ref{UMS} to handle mail and news messages
which conform to the @ref{RFC} standards..

If you have used the UMS UUCP package you should reread the complete
documentation. Almost everthing has changed!

@menu
* Copyright::        Copyright and other legal stuff

Chapters:

* Introduction::     A short tour through UMS, RFC and UUCP
* Requirements::     Requirements for using UMS RFC
* Directories::      Description of all directories in this distribution
* Configuration::    How to configure UMS RFC
* Commands::         Description of all UMS RFC commands

Appendices:

* Examples::         Some example UMS RFC configurations
* Answers::          Solutions to common UMS RFC problems
* History::          History of UMS RFC
* TODO::             Not yet implemented features
* Authors address::  Where to send bug reports & comments
* Credits::          The author wants to thank
* Index::            Index for this document
@end menu

@end ifinfo
@comment ----------------------------------------------------------------





@comment ----------------------------------------------------------------
@comment --- Chapter: Copyright (for Info/AmigaGuide document)        ---
@comment ----------------------------------------------------------------
@ifinfo
@node Copyright, Introduction, Top, Top
@chapter Copyright and other legal stuff
@cindex Copyright
@cindex Distribution
@cindex Legal stuff
@cindex Permissions
@cindex Probibitions

@heading COPYRIGHT

Copyright @copyright{} 1992-99 Stefan Becker

Permission is granted to make and distribute verbatim copies of this manual
provided the copyright notice and this permission notice are preserved on
all copies.

@ignore
Permission is granted to process this file by TeX and print the results,
provided the printed document carries a copying permission notice identical to
this one except for the removal of this paragraph (this paragraph not being
relevant to the printed manual).
@end ignore

No program, document, data file or source  code  from  this  software  package,
neither in whole nor in part,  may  be  included  or  used  in  other  software
packages unless it is authorized by a written permission from the author.

@heading NO WARRANTY

There is no warranty for this software package. Although the author  has  tried
to prevent errors he can't guarantee that the  software  package  described  in
this document is 100% reliable. You are therefore using this material  at  your
own risk. The author cannot be made responsible for any damage which is  caused
by using this software package.

@heading DISTRIBUTION

This software package is freely distributable. It  may  be  transfered  to  any
media which is used for the distribution of free software  like  Public  Domain
disk collections, CDROMs, FTP servers or bulletin board systems.

In order to ensure the integrity of this software package distributors should
use the original archive file @file{umsrfc1_0.lha} (file name on Aminet:
@file{umsrfc.lha}). The author cannot be made responsible if this software
package has become unusable due to modifications of the archive contents or of
the archive file itself.

There is no limit on the fee taken by distributors, e.g. for the media costs of
floppy disks, streamer tapes or compact discs, or the process  of  duplicating.
Such limits have proven to be harmful  to  the  idea  of  freely  distributable
software, e.g. the software package was removed instead of reducing  the  price
of a floppy disk below the limit.

Although the author does not impose any limit on these fees he  would  like  to
express his personal opinions on this matter:

@itemize @bullet

@item
This software package should be made  available  to  everyone  free  of  charge
whenever this is possible.

@item
If you have purchased this software package  under  normal  conditions  from  a
Public Domain dealer on a floppy disk and have paid more than 5DM or US $5 then
you have definitely paid too much. Please don't support  this  improper  profit
making any longer and switch to a cheaper source as soon as possible.

@end itemize

@heading USAGE RESTRICTIONS

No program, document, data file or source  code  from  this  software  package,
neither in whole nor in part, may be used on any machine which is used

@itemize @bullet

@item
for the research, development, construction, testing or production  of  weapons
or other military applications. This also includes any machine which is used in
the education for any of the above mentioned purposes.

@item
by people who accept, support  or  use  violence  against  other  people,  e.g.
citizens from foreign countries.

@end itemize

@end ifinfo
@comment ----------------------------------------------------------------





@comment ----------------------------------------------------------------
@comment --- Chapter: Introduction                                    ---
@comment ----------------------------------------------------------------
@node Introduction, Requirements, Copyright, Top
@chapter A short tour through UMS, RFC and UUCP
@cindex Introduction

This chapter tries to give you a short overview about all UMS RFC related
topics.

@menu
* UMS::              Universal Message System
* RFC::              Request for Comments
* UUCP::             Unix to Unix CoPy
@end menu

@comment ----------------------------------------------------------------





@comment ----------------------------------------------------------------
@comment --- Section: Introduction/UMS                                ---
@comment ----------------------------------------------------------------
@node UMS, RFC, , Introduction
@section Universal Message System
@cindex UMS

UMS is a database system specifically designed for the storage of mail
and news messages. It has some special features which are explained in
this section.

@menu
* Message Format::   Standard UMS message format
* Message Base::     UMS message base
* Users::            Different types of users
* Modes::            UMS operation modes
@end menu

For further information please read the documentation supplied with UMS.

@comment ----------------------------------------------------------------





@comment ----------------------------------------------------------------
@comment --- Subsection: Introduction/UMS/Message Format              ---
@comment ----------------------------------------------------------------
@node Message Format, Message Base, , UMS
@subsection Standard UMS message format
@cindex UMS Message Format

UMS uses only @strong{one} format for all messages. This makes it
possible to handle different kinds of networks, like FIDO or UUCP, with
one system. The user only needs to use message reader and all tools work
automatically with all messages. Messages can easily be exchanged between
different kinds of networks.

Messages from the outside have to be converted to this format before
handing them to UMS (importing). They also have to be converted from this
format when transferring messages from UMS to a network (exporting). These
conversions are handled by special programs called Importers or
Exporters. In UMS RFC the conversion is handled by the
@code{umsrfc.library} which is used by all programs.

@menu
Advanced information:

* Addresses::        UMS address formats
* Attributes::       Additional message information
* Conversion::       UMS <-> RFC conversion
@end menu

@comment ----------------------------------------------------------------





@comment ----------------------------------------------------------------
@comment --- Subsubsection: Introduction/UMS/Msg Format/Addresses     ---
@comment ----------------------------------------------------------------
@node Addresses, Attributes, , Message Format
@subsubsection UMS address formats
@cindex UMS Address Formats

@ref{RFC} addresses are mapped 1:1 to UMS addresses. Additionally UMS RFC
currently understands the following two UMS address formats. It can
convert them to and from @ref{RFC} addresses:

@table @code
@item <zone>:<hub>/<node>.<point>@@fidonet
FIDO (FTN) address format

@item <box>.maus
Maus address format
@end table

@comment ----------------------------------------------------------------





@comment ----------------------------------------------------------------
@comment --- Subsubsection: Introduction/UMS/Msg Format/Attributes    ---
@comment ----------------------------------------------------------------
@node Attributes, Conversion, Addresses, Message Format
@subsubsection Additional message information
@cindex Attributes

UMS RFC supports several parameters which can be  specified  in  the  UMS
message field @code{Attributes}:

@table @code
@item ALIAS
Use this alias for the address generation instead of @ref{rfc.username}.
UMS RFC checks if this alias is valid for the creator of the message.

@item RECEIPT-REQUEST
Send a receipt message to the specified address. This creates the
@ref{RFC} header @code{Return-Receipt-To}. The address may be specified
in one of the following four different forms:

@example
""                      - Creates address for sender of message
<name>                  - Creates address for local user
<account@@domain>        - Any address may be specified
<name>,<account@@domain> - Any address may be specfied
@end example

@item URGENT
Flag this message as urgent. This creates the following @ref{RFC} header:

@example
Priority: urgent
@end example
@end table

@comment ----------------------------------------------------------------





@comment ----------------------------------------------------------------
@comment --- Subsubsection: Introduction/UMS/Msg Format/Conversion    ---
@comment ----------------------------------------------------------------
@node Conversion, , Attributes, Message Format
@subsubsection UMS <-> RFC conversion
@cindex Conversion
@cindex rfc.control

UMS itself is not a @ref{RFC} system, which means that all messages must
be transformed from the RFC format to the UMS format and vice versa. Not
every piece of RFC information is used or stored by UMS, so the importer
must preserve the original information. All RFC header fields which are
not known to the importer are preserved.

There are two exceptions to this rule:

@itemize @minus
@item
The order of the RFC header fields is not always preserved. Luckily, the
RFC standards do not impose a strict order for most of the header fields.

@item
If the message is a @ref{MIME} message and the importer can decode the
information, the exporter will not reconstruct the original message text.
Instead it will create new MIME headers which specify the new format.
@end itemize

The exporters use the additional information which is stored by the
importers to reconstruct the original messages. Here are some examples
where preserving of the information is needed:

@table @code
@item From
@itemx Reply-To
These RFC fields contain the address and maybe the real name of the user.
Since this information is stored in two separate UMS fields these lines
must be parsed to extract the needed information. This parsing process is
not reversible.

@item Newsgroups
Crosspostings are stored as hard-linked messages, which appear in
different groups. But an importer for a system may not have write access
to all groups in the crossposting, so exporters can't reconstruct this
line by reading all hard-linked messages.

@item References
UMS only uses one message ID for message threading.
@end table

Mapping of RFC headers to UMS fields:

@example
RFC header   | UMS fields                            | Preserved
----------------------------------------------------------------
Control      | - (cancel commands will be processed) |   Yes
Date         | Date & CDate                          |
Distribution | Distribution                          |
Followup-To  | ReplyGroup                            |
From         | FromAddr & FromName (MIME decoded)    |   Yes
In-Reply-To  | ReferID                               |   Yes
Message-ID   | MessageID                             |
Newsgroups   | Group (X messages hard linked)        |   Yes
Organization | Organization (MIME decoded)           |
References   | ReferID (the last id in the header)   |   Yes
Reply-To     | ReplyAddr & ReplyName (MIME decoded)  |   Yes
Subject      | Subject (MIME decoded)                |
X-Mailer     | Newsreader                            |
X-NewsReader | Newsreader                            |
<all others> | Comment                               |   Yes
@end example

All control messages will be written to a special newsgroup called
@samp{rfc.control}. You have to make sure that this newsgroup is included
in the @code{WRITEACCESS} pattern of the importer user entries. You can
allow the distribution of control messages by adding this newsgroup to
the @code{READACCESS} pattern of a exporter user entry.

Additional UMS fields used for creation of RFC headers during export:

@example
UMS field          | RFC header
-----------------------------------------------------------------------
Attributes         | (miscellaneous)
LogicalToAddr/Name | To (mail only, superseeds ToAddr)
RfcAttr            | - (contents directly transferred into RFC message)
ToAddr/Name        | To (mail only, also used as physical address)
@end example

@comment ----------------------------------------------------------------





@comment ----------------------------------------------------------------
@comment --- Subsection: Introduction/UMS/Message Base                ---
@comment ----------------------------------------------------------------
@node Message Base, Users, Message Format, UMS
@subsection UMS message base
@cindex UMS Message Base

All UMS messages are stored in a message base. This is a special database
controlled by the @code{umsserver}. Knowledge about exact format of this
database is not needed by the users because they don't access it
directly. They have to go through the standard Interface
@code{ums.library}.

Multiple message bases can be active at the same time, each controlled by
its own @code{umsserver}. Even a message base on a remote machine can  be
accessed via network. All the user has to do is to specify  the  name  of
the message base:

@table @code
@item <name>
Use the specified message base on the local machine.

@item <name>@@<machine>
Use the specified message base on a remote machine using Envoy as network
software.

@item <machine>:<service>
@itemx <machine>:<port>
Access a message base on a remote machine using @ref{TCP/IP} as network
software. The message base will be specified with the configuration on the
remote machine.
@end table

@comment ----------------------------------------------------------------





@comment ----------------------------------------------------------------
@comment --- Subsection: Introduction/UMS/Users                       ---
@comment ----------------------------------------------------------------
@node Users, Modes, Message Base, UMS
@subsection Different types of users
@cindex Users

UMS differentiates three types of users:

@table @samp
@item Users
They can read and write messages, but they don't have any special rights.

@item System Operators
Users with special rights, also called SysOps. They may access the
headers of messages from other users to perform special tasks. Note that
they @strong{can't} read the message text itself, only the headers!

@item Exporters
Users with special read and write rights. They get read rights to all
messages which should be exported. They may also import messages from
remote systems.
@end table

UMS RFC needs one user with the name @samp{postmaster}. It will send all
error messages as mail to this user. Normally you should add the alias
@samp{postmaster} to a SysOp user entry.

@comment ----------------------------------------------------------------





@comment ----------------------------------------------------------------
@comment --- Subsection: Introduction/UMS/Modes                       ---
@comment ----------------------------------------------------------------
@node Modes, , Users, UMS
@subsection UMS operation modes
@cindex UMS Operation Modes
@cindex Leaf Node
@cindex Full Node
@cindex Gateway

UMS can be operated in three different modes for news distribution:

@table @samp
@item Leaf node
The last system in a news message distribution chain. It imports news
messages from one or more remote systems, but doesn't export those news
messages to other systems. Only local news messages are exported to the
remote systems. Most users should use this mode, because it's very easy
to maintain.

To make UMS behave like a leaf node use

@example
( NodeMode "%")
@end example

@xref{rfc.noownfqdn}, for more information.

@item Full node
This node is located inside a news message distribution chain. It imports
news messages from remote systems and exports them to other remote
systems. The maintainer of such a system must control the message flow
between the systems very carefully in order to prevent loops and
duplicate messages. Only experienced users should use this mode.

With the UMS configuration variable @code{NodeMode} you can specify which
newsgroups should be handled by UMS in full node mode, e.g.

@example
( NodeMode "comp.#?" )
@end example

All messages in newsgroups whose names match this pattern may be read by
other exporters. The actual message flow between the systems is
controlled with the variable @code{READACCESS} in the exporter user
entries.

@item Gateway
Such a node behaves like a full node, but additionally it can distribute
news messages between networks with different technologies and newsgroup
names. Such systems are very delicate to handle, can cause many headaches
and create many enemies and flame wars. Only @strong{VERY VERY}
experienced users should use this mode. You have been warned!

To operate UMS as gateway you must have a special key, the @code{GateKey}.
@end table

This scheme is only applied to news messages, because they can generate a
very high amount of traffic in the networks. Mail messages are always
freely distributed between all systems.

@comment ----------------------------------------------------------------





@comment ----------------------------------------------------------------
@comment --- Section: Introduction/RFC                                ---
@comment ----------------------------------------------------------------
@node RFC, UUCP, UMS, Introduction
@section Request for comments
@cindex RFC

RFC are standards which define the Internet technology. Each RFC has a
unique number by which it can be identified, e.g.@: RFC 822. Currently
there around 2000 RFC available, although some of the older ones have
been obsoleted by newer ones.

The following RFC define the standards for mail and news messages and
their transportation across networks.

@menu
Format standards:

* Message Formats::  The standard RFC message formats
* MIME::             Multi-Purpose Message Extensions

Service standards:

* NNTP::             Network News Transfer Protocol
* POP3::             Post Office Protocol (Version 3)
* SMTP::             Simple Mail Transfer Protocol

Other standards:

* TCP/IP::           Internet protocol stack
@end menu

@comment ----------------------------------------------------------------





@comment ----------------------------------------------------------------
@comment --- Subsection: Introduction/RFC/Message Formats             ---
@comment ----------------------------------------------------------------
@node Message Formats, MIME, , RFC
@subsection The standard RFC message formats
@cindex RFC Message Formats
@cindex RFC 822
@cindex RFC 1036

Two standards define the format of messages on the Internet:

@table @samp
@item RFC 822 (1982)
@itemx Standard for the format of ARPA Internet text messages
Defines the format of messages on the Internet. It is therefore the most
important standard for a RFC compliant messages handling system. A
message consists of two parts: the header and the body. They are
seperated by an empty line.

The header consists of non-empty lines with a keyword, followed by a
colon and the parameters. The standard defines many keywords, their
semantics and the syntax of their parameters. Especially it specifies the
format of time and address parameters. An example message header looks
like this:

@example
From: "Joe Dumb User" <jduser@@test.foo.bar>
To: "John Doe" <jd@@dummy.do.main>
Subject: Test message
Date: Fri, 26 May 1995 13:25:07 +0100
Message-ID: <54208434@@test.foo.bar>
@end example

The standard doesn't define any format for the mody of a message. The
only restriction is that only 7 Bit ASCII characters may be used. This
restriction may be circumvented with @ref{MIME}.

@item RFC 1036 (1987)
@itemx Standard for Interchange of USENET Messages

This is the extension of RFC 822 for news messages. It defines some new
keywords, like @code{Newsgroups} and @code{Path}. Also some of the many
allowed variants in RFC 822 are narrowed down to a usable set of options.
@end table

@comment ----------------------------------------------------------------





@comment ----------------------------------------------------------------
@comment --- Subsection: Introduction/RFC/MIME                        ---
@comment ----------------------------------------------------------------
@node MIME, NNTP, Message Formats, RFC
@subsection Multi-Purpose Message Extensions
@cindex Multi-Purpose Message Extensions
@cindex MIME
@cindex RFC 1521
@cindex RFC 1522

RFC 822 concentrates on the format of the message headers and doesn't say
much about the format of the message body. The body may therefore only be
used for plain text. Additionally it allows only 7 Bit ASCII as charset
for messages.

These restrictions, especially when the biggest hype of the  90's  called
Multimedia hit the  message  handling  systems,  led  to  development  of
several extensions to RFC 822 called MIME:

@table @samp
@item RFC 1521 (1993)
@itemx MIME Part I: Mechanisms for Specifying and Describing
@itemx the Format of Internet Message Bodies
This defines a structure for the message body. The body may contain
Non-ASCII text and binary data. The encoding of the message body for
transportation over non-transparent links is specified. Messages may be
composed of several different types of data. A mechanism for splitting up
large messages into several smaller ones is specified.

Additionally it defines some new RFC header fields which identify a MIME
message and specify the contents and the encoding of the message body.

@item RFC 1522 (1993)
@itemx MIME Part II: Message Header Extensions for Non-ASCII Text
Specifies the encoding of Non-ASCII text in RFC message headers. This
allows you to use such texts in subjects.
@end table

UMS RFC currently only supports  the  decoding  of  message  headers  and
bodies on import and the encoding of message bodies on export. So you may
use all  ISO-8859-1  characters  in  the  message  body,  like  @samp{�},
@samp{�} and @samp{�}.

@comment ----------------------------------------------------------------





@comment ----------------------------------------------------------------
@comment --- Subsection: Introduction/RFC/NNTP                        ---
@comment ----------------------------------------------------------------
@node NNTP, POP3, MIME, RFC
@subsection Network News Transfer Protocol
@cindex Network News Transfer Protocol
@cindex NNTP
@cindex RFC 977

USENET, the biggest news distribution system of the world, originally
used @ref{UUCP} as transportation mechanism. But with the advent of the
Internet with its online connections a new mechanism was needed. This led
to the development of the NNTP:

@table @samp
@item RFC 977 (1986)
@itemx Network News Transfer Protocol
This standard specifies a synchronous protocol between two news
exchanging entities, a client and a server. The client plays the active
role and sends commands, like @code{POST} or @code{GROUP}, to the server.
It then waits for a response of the server which contains status or error
codes or the news message. Only after that the client may send the next
command.

A special protocol is defined if the client is also a news server. The
client announces which new messages are available from its system and the
server can request or reject them. This lowers the network traffic
because messages are only transferred once even if there are multiple
redundant distribution paths available for a message.
@end table

During the recent years several inofficial extensions to NNTP have been
developed. Although not officially approved UMS RFC implements some of
most common or useful ones: extended header query (@code{XHDR}), overview
format (@code{XOVER}) and simple authentication
(@code{AUTHINFO USER/PASS}).

@comment ----------------------------------------------------------------





@comment ----------------------------------------------------------------
@comment --- Subsection: Introduction/RFC/POP3                        ---
@comment ----------------------------------------------------------------
@node POP3, SMTP, NNTP, RFC
@subsection Post Office Protocol (Version 3)
@cindex Post Office Protocol
@cindex POP3
@cindex RFC 1081

On systems where user mailboxes are not available network-wide because of
a missing network file system or security reasons the mail messages are
usually received by one host. This central server plays the role of a
post office which handles the mailboxes for its users. The server can be
accessed using POP3.

@table @samp
@item RFC 1081 (1988)
@itemx Post Office Protocol - Version 3
Defines the synchronous protocol between a server and a client. Note that
the client can only retrieve messages. For sending mail messages he has
to use another protocol like @ref{SMTP}.

First the client has to send the user name and the password to gain
access to the mailbox. The client may then retrieve messages from the
mailbox. It may also mark messages for deletion, but the server will
physically delete the messages only after the client has send the command
to close the connection.
@end table

Most systems which use POP3 don't have their own domain name.
@xref{rfc.noownfqdn}.

@comment ----------------------------------------------------------------





@comment ----------------------------------------------------------------
@comment --- Subsection: Introduction/RFC/SMTP                        ---
@comment ----------------------------------------------------------------
@node SMTP, TCP/IP, POP3, RFC
@subsection Simple Mail Transfer Protocol
@cindex Simple Mail Transfer Protocol
@cindex SMTP
@cindex ESMTP
@cindex RFC 821
@cindex RFC 1652
@cindex RFC 1854
@cindex RFC 1869
@cindex RFC 1870

One of the first applications for the Internet was the exchange  of  electronic
mail. Several protocols for transferring mail messages over networks have  been
defined, but the most commonly used one is SMTP and its extension ESMTP:

@table @samp
@item RFC 821 (1982)
@itemx Simple Mail Transfer Protocol
Defines a synchronous protocol between two mail exchanging entities. The
calling party plays the active role which can be changed during the
conversation, but this option seldomly used. For each mail the active
part first transmits the senders address and then a list with the
recipient addresses. These addresses are checked by the receiving side.
After that the mail message is transferred.

@item RFC 1652 (1994)
@itemx SMTP Service Extension for 8bit-MIMEtransport
Defines the extension to exchange information about message body encoding.

@item RFC 1854 (1995)
@itemx SMTP Service Extension for Command Pipelining
Defines the extension how to queue SMTP commands and responses in order to
eliminate turnarounds and thus improving performance.

@item RFC 1869 (1995)
@itemx SMTP Service Extensions
Defines the means to extend SMTP.

@item RFC 1870 (1995)
@itemx SMTP Service Extension for Message Size Declaration
Defines the extension to exchange information about message size and message
size limits.

@end table

@comment ----------------------------------------------------------------





@comment ----------------------------------------------------------------
@comment --- Subsection: Introduction/RFC/TCP-IP                      ---
@comment ----------------------------------------------------------------
@node TCP/IP, , SMTP, RFC
@subsection Internet protocol stack
@cindex Internet protocols
@cindex Protocol stack
@cindex TCP/IP

TCP/IP is the most commonly used abbreviation for protocol suite used on
the Internet. Although there are several other protocols in use the most
important are IP and TCP.

@table @samp
@item IP - Internet Protocol
Defines an unreliable, connectionless datagram service which is the base
for @emph{all} Internet protocols and services. IP datagrams contain a
destination adddress which enables the interconnecting systems to route a
datagram from the sender to the recipient. The addresses, the so called
IP numbers, are 32 Bit wide. The arrival and the order of the datagrams
is not guaranteed with IP.

IP can be used on many different types of network hardware. Examples are
Ethernet, FDDI, serial lines (SLIP, CSLIP or PPP) or X.25.

@item TCP - Transmission Control Protocol
Defines a reliable, connection oriented stream protocol. From the point
of the user the connection looks like a FIFO, that is the bytes arrive in
the same order as he sends them. TCP splits the stream into datagrams and
reassembles them at the receiving side. It also makes sure that the
datagrams are placed correctly when they arrive out of order. Lost or
errorneous datagrams are requested again from the sender.
@end table

@comment ----------------------------------------------------------------





@comment ----------------------------------------------------------------
@comment --- Section: Introduction/UUCP                               ---
@comment ----------------------------------------------------------------
@node UUCP, , RFC, Introduction
@section Unix to Unix CoPy
@cindex Unix to Unix CoPy
@cindex UUCP

UUCP was originally written to transfer files to a remote UNIX machine.
The syntax of the main command @code{uucp} is the same as the UNIX
@code{cp} command hence the name. Each file transfer request creates a
job which is stored in the so called spool directory for the remote
system.

At certain times the local machine checks if there are any files waiting
to be transferred to the remote system. It then starts the command
@code{uucico} (Unix to Unix Copy In Copy Out) which automatically
contacts the remote system via a modem line or a @ref{TCP/IP} connection.
The remote system then starts its own @code{uucico} and both programs
process their jobs and send the files forth and back.

Even remote command execution is possible with UUCP. To achieve this a
job with three files is created. Two of them are data files with the
prefix @file{D.}. One of these contains the data and the other the
commands how to process the data file on the remote system. The third
file with the prefix @file{C.} contains the commands for @code{uucico} to
send both files to the remote system.

The command file is renamed to a file with the prefix @file{X.} on the
remote system. After the connection has been closed @code{uucico} starts
the command @code{uuxqt} (Unix to Unix eXeCuTion?). It looks into the
spool directory for files with the prefix @file{X.} and processes the
commands.

This mechanism can be also used to transfer mail and news messages (The
news distribution system USENET was originally based on UUCP). The data
file contains a message in the @ref{Message Formats,RFC 822} format
and the command file contains mail and news processing commands, like
@code{rmail}.

Only mail messages are transferred in this way nowadays, because small
files degrade the performance of the file transfer. This can be avoided
if the data file contains several messages (batching). News messages can
simply be concatenated into one file, because they already contain the
message length in the header. For mail messages an adaption of the
@ref{SMTP} format is used, called Batched SMTP (BSMTP).

To reduce the amount of data which needs to be transferred batched files
are usually compressed. Currently in use are three compression methods:
compress, freeze and gzip. Normally only the plain output of those
programs is stored in the data file, but compressed news batches usually
contain a small text header which identifies the compression method.

@comment ----------------------------------------------------------------





@comment ----------------------------------------------------------------
@comment --- Chapter: Requirements                                    ---
@comment ----------------------------------------------------------------
@node Requirements, Directories, Introduction, Top
@chapter Requirements for using UMS RFC
@cindex Requirements

This version of UMS RFC requires the following things:

@itemize @bullet

@item
AmigaOS 2.04 (V37) or better.

@item
@ref{UMS} V11 or better.

@item
AmiTCP 3.0 or better if you are using @ref{TCP/IP}.

@item
A service provider for Mail and News via @ref{UUCP} or @ref{TCP/IP}
(@ref{NNTP}, @ref{POP3} or @ref{SMTP}).

@end itemize

@comment ----------------------------------------------------------------





@comment ----------------------------------------------------------------
@comment --- Chapter: Directories                                     ---
@comment ----------------------------------------------------------------
@node Directories, Configuration, Requirements, Top
@chapter Description of all directories in this distribution
@cindex Directories

UMS RFC uses the standard @ref{UMS} directory structure, so you can easily
unpack the distribution into your UMS directory. If you don't want to install
UMS RFC by hand you can use the supplied Installer script.

@table @file
@item UMS/Bin
UMS RFC programs and some utility programs. Add this  directory  to  your
command path.

@item UMS/Devs

Contains the @code{telser.device}. Copy this file to @file{DEVS:} or add this
directory to your @file{DEVS:} path with

@example
Assign DEVS: UMS/Devs ADD
@end example

@item UMS/Docs/english
Documentation (ASCII and AmigaGuide) for UMS RFC in english.

@item UMS/Libs
@code{umsrfc.library} and some utility libraries. Copy these files to your
@file{LIBS:} directory or add this directory to your @file{LIBS:} path
with

@example
Assign LIBS: UMS/Libs ADD
@end example

@item UMS/Networks/RFC/convert
Scripts to import existing mail and news messages from other @ref{UUCP}
packages into an @ref{Message Base,UMS message base}.

@item UMS/Networks/RFC/db
Example configuration files for @code{telser}. Copy these to the directory
@file{AMITCP:db}

@item UMS/Networks/RFC/lib
Example configuration files for @ref{UUCP}. Copy these to the directory
@file{UULIB:}

@item UMS/Networks/RFC/src
This directory contains the @file{History} file with the complete
story about the development of UMS RFC and its predecessors.
@end table

@comment ----------------------------------------------------------------





@comment ----------------------------------------------------------------
@comment --- Chapter: Configuration                                   ---
@comment ----------------------------------------------------------------
@node Configuration, Commands, Directories, Top
@chapter How to configure UMS RFC
@cindex Configuration

UMS RFC offers many options and this chapter explains the configuration.
Please read it carefully because a proper configuration is essential for
the correct operation of UMS RFC.

If you don't want to configure UMS RFC by hand you may use the supplied
Installer script. @strong{NOTE:} The Installer script only works with
with umsserver V11.32, ums.library V11.19 or better!

@menu
* Environment::        AmigaDOS environment variables
* UMS Configuration::  UMS configuration variables
@end menu

@comment ----------------------------------------------------------------





@comment ----------------------------------------------------------------
@comment --- Section: Configuration/Environment                       ---
@comment ----------------------------------------------------------------
@node Environment, UMS Configuration, , Configuration
@section AmigaDOS environment variables
@cindex Environment variables

Only one environment variable is currently used by UMS RFC:

@table @code
@item UMSUUCP.mb
This variable tells the command @ref{uuxqt} which @ref{Message Base,UMS
message base} it should use to import @ref{UUCP} batches.
@end table

@comment ----------------------------------------------------------------





@comment ----------------------------------------------------------------
@comment --- Section: Configuration/UMS Configuration                 ---
@comment ----------------------------------------------------------------
@node UMS Configuration, , Environment, Configuration
@section UMS configuration variables
@cindex UMS configuration variables

Most of the options are configured using the @ref{UMS} variables which
are explained in this section.

@menu
Mandantory variables:

* rfc.domainname::   Domain name for your system
* rfc.username::     Account name for a local UMS RFC user

Optional variables (may be omitted):

* rfc.dstime::       Daylight Saving Time on/off
* rfc.export.::      Address conversions during message export
* rfc.import.::      Address conversions during message import
* rfc.mailencoding:: Encoding of mail message texts during export
* rfc.newsencoding:: Encoding of news message texts during export
* rfc.noownfqdn::    The system has no own domain name
* rfc.pathname::     Path name for your system
* nntpd.access::     Access rights for remote NNTP clients
* nntpd.log::        NNTP server log on/off
* nntpget.groups::   Group list for news message import
* smtpd.log::        SMTP server log on/off

Mandantory UUCP variables:

* uucp.mailexport::  Configuration for mail batches
* uucp.newsexport::  Configuration for news batches
* uucp.nodename::    UUCP node name for your system

Optional UUCP variables (may be omitted):

* uucp.envelope::    UUCP envelope type
* uucp.keepdupes::   Keep dupe messages
* uucp.logdupes::    Log dupe messages
* uucp.mailroute::   Mail routing configuration
* uucp.recipients::  Limit number of recipients
@end menu

@comment ----------------------------------------------------------------





@comment ----------------------------------------------------------------
@comment --- Subsection: Configuration/UMS Config/rfc.domainname      ---
@comment ----------------------------------------------------------------
@node rfc.domainname, rfc.username, , UMS Configuration
@subsection Domain name for your system
@cindex rfc.domainname
@cindex Domain Name

@table @code
@item rfc.domainname
The fully qualified domain name (FQDN) for your system. The FQDN consists of
the hostname plus the name of your domain. This variable is @strong{mandantory}
for using UMS RFC.

If your system doesn't have an own domain name then set @ref{rfc.noownfqdn} and
use the domain name of your server host.
@end table

Example:

@example
( rfc.domainname "foobar.earth.sol.galaxy" )
@end example

@comment ----------------------------------------------------------------





@comment ----------------------------------------------------------------
@comment --- Subsection: Configuration/UMS Config/rfc.username        ---
@comment ----------------------------------------------------------------
@node rfc.username, rfc.dstime, rfc.domainname, UMS Configuration
@subsection Account name for a local UMS RFC user
@cindex rfc.username
@cindex Account name

@table @code
@item rfc.username
This is the @emph{account} name for a user. This variable is
@strong{mandantory} for every UMS user whose messages are exported with UMS
RFC. The user entry also must contain an alias for this name.
@end table

Example:

@example
( rfc.domainname "foobar.earth.sol.galaxy" )

( User
  ( Name "Joe Dumb User" )
  ( Alias "jduser" )
  ...
  ( rfc.username "jduser" )
)
@end example

The above example will result in the following Internet address for the user:

@example
jduser@@foobar.earth.sol.galaxy
@end example

@comment ----------------------------------------------------------------





@comment ----------------------------------------------------------------
@comment --- Subsection: Configuration/UMS Config/rfc.dstime          ---
@comment ----------------------------------------------------------------
@node rfc.dstime, rfc.export., rfc.username, UMS Configuration
@subsection Daylight Saving Time on/off
@cindex rfc.dstime
@cindex Daylight Saving Time

@table @code
@item rfc.dstime
This switch enables the Daylight Saving Time. The default is @code{N}.
@end table

Example:

@example
( rfc.dstime "Y" )
@end example

@comment ----------------------------------------------------------------





@comment ----------------------------------------------------------------
@comment --- Subsection: Configuration/UMS Config/rfc.export.         ---
@comment ----------------------------------------------------------------
@node rfc.export., rfc.import., rfc.dstime, UMS Configuration
@subsection Address conversions during message export
@cindex rfc.export.fido
@cindex rfc.export.maus

@table @code
@item rfc.export.*
This set of variables specificies the domain names used in the address
conversion for exporting messages from non RFC compliant nets. Currently
defined are the conversions for Fido (FTN) and Maus, using the following
variables:

@example
rfc.export.fido  (Default: .fidonet.org)
rfc.export.maus  (Default: .maus.de)
@end example
@end table

Example:

@example
( rfc.export.fido ".fido.de" )
( rfc.export.maus ".maus.sub.org" )
@end example

The above example will result in the following address conversion:

@example
Joe User,1:2/3.4@@fidonet  ->  Joe_User@@p4.f3.n2.z1.fido.de
Joe User,HB.maus          ->  Joe_User@@HB.maus.sub.org
@end example

@comment ----------------------------------------------------------------





@comment ----------------------------------------------------------------
@comment --- Subsection: Configuration/UMS Config/rfc.import.         ---
@comment ----------------------------------------------------------------
@node rfc.import., rfc.mailencoding, rfc.export., UMS Configuration
@subsection Address conversions during message import
@cindex rfc.import.fido
@cindex rfc.import.maus

@table @code
@item rfc.import.*

This set of  variables  specificies  the  domain  names  used  in  the  address
conversion for importing messages arriving from a RFC compliant net, but  which
originate in a non RFC compliant net. You may specify several domain names  for
each type of net, separated by commas. Currently defined  are  the  conversions
for Fido (FTN) and Maus, using the following variables:

@example
rfc.import.fido  (Default: .fidonet.org)
rfc.import.maus  (Default: .maus.de)
@end example
@end table

Example:

@example
( rfc.import.fido ".fido.de,.fidonet.org" )
( rfc.import.maus ".maus.sub.org" )
@end example

The above example will result in the following address conversion:

@example
Joe_User@@p4.f3.n2.z1.fidonet.org  ->  Joe User,1:2/3.4@@fidonet
Joe_User@@p4.f3.n2.z1.fido.de      ->  Joe User,1:2/3.4@@fidonet
Joe_User@@HB.maus.sub.org          ->  Joe User,HB.maus
@end example

@comment ----------------------------------------------------------------





@comment ----------------------------------------------------------------
@comment --- Subsection: Configuration/UMS Config/rfc.mailencoding    ---
@comment ----------------------------------------------------------------
@node rfc.mailencoding, rfc.newsencoding, rfc.import., UMS Configuration
@subsection Encoding of mail message texts during export
@cindex rfc.mailencoding

@table @code
@item rfc.mailencoding
When exporting mail messages which contain non-ASCII characters for
transportation over non-transparent (that is 8 bit clean) links, UMS RFC can
encode them according to the @ref{MIME} standard. The following values are
allowed for this variable:

@example
0  Don't encode (Default)
1  Encode as @code{quoted-printable}
@end example

Note that even if you have disabled encoding the messages might be encoded
as @code{quoted-printable} if the transportation link isn't transparent.
@end table

Example:

@example
( rfc.mailencoding 1 )
@end example

@comment ----------------------------------------------------------------





@comment ----------------------------------------------------------------
@comment --- Subsection: Configuration/UMS Config/rfc.newsencoding    ---
@comment ----------------------------------------------------------------
@node rfc.newsencoding, rfc.noownfqdn, rfc.mailencoding, UMS Configuration
@subsection Encoding of news message texts during export
@cindex rfc.newsencoding

@table @code
@item rfc.newsencoding
When exporting news messages which contain non-ASCII characters for
transportation over non-transparent (that is 8 bit clean) links, UMS RFC can
encode them according to the @ref{MIME} standard. The following values are
allowed for this variable:

@example
0  Don't encode (Default)
1  Encode as @code{quoted-printable}
@end example

Note that even if you have disabled encoding the messages might be encoded
as @code{quoted-printable} if the transportation link isn't transparent.
@end table

Example:

@example
( rfc.newsencoding 1 )
@end example

@comment ----------------------------------------------------------------





@comment ----------------------------------------------------------------
@comment --- Subsection: Configuration/UMS Config/rfc.noownfqdn       ---
@comment ----------------------------------------------------------------
@node rfc.noownfqdn, rfc.pathname, rfc.newsencoding, UMS Configuration
@subsection The system has no own domain name
@cindex rfc.noownfqdn

@table @code
@item rfc.noownfqdn

If your system has no own fully qualified domain name (FQDN), e.g.@: when you
receive your mail with @ref{POP3}, then you must set this variable to @code{Y}.
UMS RFC then doesn't generate any @code{Received:} lines nor does it add your
systems' name to the @code{Path:} line.

In this case you must set @ref{rfc.domainname} to the domain name of your
server host.
@end table

Example:

@example
( rfc.noownfqdn Y )
( rfc.domainname "server.sub.domain" )
@end example

The address will then look like:

@example
jduser@@server.sub.domain
@end example

@comment ----------------------------------------------------------------





@comment ----------------------------------------------------------------
@comment --- Subsection: Configuration/UMS Config/rfc.pathname        ---
@comment ----------------------------------------------------------------
@node rfc.pathname, nntpd.access, rfc.noownfqdn, UMS Configuration
@subsection Path name for your system
@cindex rfc.pathname
@cindex Path name

@table @code
@item rfc.pathname
The name of your system in the "Path:" line for news messages. If your system
is a registered UUCP node, then this variable only needs to contain the
hostname of your system. Otherwise it must contain the FQDN of your system.

If you don't specify this variable then the contents of @ref{rfc.domainname}
are used.
@end table

Example for registered UUCP nodes:

@example
( rfc.domainname "foobar.earth.sol.galaxy" )
( rfc.pathname   "foobar" )
@end example

@comment ----------------------------------------------------------------





@comment ----------------------------------------------------------------
@comment --- Subsection: Configuration/UMS Config/nntpd.access        ---
@comment ----------------------------------------------------------------
@node nntpd.access, nntpd.log, rfc.pathname, UMS Configuration
@subsection Access rights for remote NNTP clients
@cindex nntpd.access

@table @code
@item nntpd.access
When a remote client tries to connect the @ref{NNTP} service the @ref{umsnntpd}
server daemon reads the contents of this variable to find out which access
rights the client has. The variable may contain multiple lines of the following
format:

@example
<Name pattern> <UMS user name>,<Post Y/N>,<Server Y/N>
@end example

The check is based on the domain name of the machine on which the client
runs. The patterns from the first line down to the last line will be used
to match the domain name. When a pattern matches then the access rights
of this entry are used for the client. If no pattern matches the access
is denied with an error message.

The account of the specified UMS user name is used to export and import
the news messages. If the second parameter is set to @code{Y} then
posting (that is importing) of news messages with the @code{POST} command
is allowed. If the third parameter is also set to @code{Y} then the
client is a @ref{NNTP} server and is allowed to use the @code{IHAVE}
command for transferring news messages.
@end table

Example:

@ifinfo
@example
( nntpd.access
   "localhost       NNTP1,Y,Y\\n"
   "*.ping.de       NNTP2,Y,N\\n"
   "*               NNTP3,N,N\\n" )
@end example
@end ifinfo
@iftex
@example
( nntpd.access
   "localhost       NNTP1,Y,Y\n"
   "*.ping.de       NNTP2,Y,N\n"
   "*               NNTP3,N,N\n" )
@end example
@end iftex

The above example allows full access for a client running on the local
machine. Clients running on machines in the domain @code{ping.de} are
allowed to post news messages. All other machines are only allowed to
read news messages. Using three different UMS users you can control which
newsgroups are visible for each of the three categories using the
@code{READACCESS} variable.

@comment ----------------------------------------------------------------





@comment ----------------------------------------------------------------
@comment --- Subsection: Configuration/UMS Config/nntpd.log           ---
@comment ----------------------------------------------------------------
@node nntpd.log, nntpget.groups, nntpd.access, UMS Configuration
@subsection NNTP server log on/off
@cindex nntpd.log

@table @code
@item nntpd.log
This variable enables the command logging of the NNTP server to the UMS
log file. The default is @code{N}.
@end table

Example:

@ifinfo
@example
( nntpd.log "Y" )
@end example

@comment ----------------------------------------------------------------





@comment ----------------------------------------------------------------
@comment --- Subsection: Configuration/UMS Config/nntpget.groups      ---
@comment ----------------------------------------------------------------
@node nntpget.groups, smtpd.log, nntpd.log, UMS Configuration
@subsection Group list for news message import
@cindex nntpget.groups

@table @code
@item nntpget.groups
The @ref{NNTP} importer @ref{umsnntpget} looks into this variable to find
out which newsgroups it should request from the remote server. It may
contain multiple lines. If you are using the option @samp{-g} each line
must contain one newsgroup name. If you don't use this option then each
line may contain multiple group name patterns seperated by commas and
even negations.
@end table

Example (use with option @samp{-g}):

@ifinfo
@example
( nntpget.groups "group.a\\n"
                 "group.b\\n"
                 "group.d\\n"
                 "de.test.a\\n"
                 "foo.bar.a\\n"
                 "foo.bar.b\\n" )
@end example
@end ifinfo
@iftex
@example
( nntpget.groups "group.a\n"
                 "group.b\n"
                 "group.d\n"
                 "de.test.a\n"
                 "foo.bar.a\n"
                 "foo.bar.b\n" )
@end example
@end iftex

Example (use without option @samp{-g}):

@ifinfo
@example
( nntpget.groups "group.*,!group.c,de.test.a\\n"
                 "foo.bar.*\\n" )
@end example
@end ifinfo
@iftex
@example
( nntpget.groups "group.*,!group.c,de.test.a\n"
                 "foo.bar.*\n" )
@end example
@end iftex

@comment ----------------------------------------------------------------





@comment ----------------------------------------------------------------
@comment --- Subsection: Configuration/UMS Config/smtpd.log           ---
@comment ----------------------------------------------------------------
@node smtpd.log, uucp.mailexport, nntpget.groups, UMS Configuration
@subsection SMTP server log on/off
@cindex smtpd.log

@table @code
@item smtpd.log
This variable enables the command logging of the SMTP server to the UMS
log file. The default is @code{N}.
@end table

Example:

@ifinfo
@example
( smtpd.log "Y" )
@end example

@comment ----------------------------------------------------------------





@comment ----------------------------------------------------------------
@comment --- Subsection: Configuration/UMS Config/uucp.mailexport     ---
@comment ----------------------------------------------------------------
@node uucp.mailexport, uucp.newsexport, smtpd.log, UMS Configuration
@subsection Configuration for mail batches
@cindex uucp.mailexport

@table @code
@item uucp.mailexport
This variable tells the @ref{UUCP} exporter @ref{ums2uucp} in which
format mail messages should be exported. The syntax is as follows:

@example
<command>[,<grade>[,<batch>[,<compress command>[,<header>[,<max size>]]]]]
@end example

@table @code
@item command
Command name for the @code{C} line in the @ref{UUCP} command file. You
have to talk to the administrator of your host which commands his
software accepts and how it interprets them. The following table shows
some common names for mail batches:

@example
Name    | Meaning
-----------------------------------------------
rmail   | unbatched, uncompressed mail
rbsmtp  | batched, uncompressed mail
rcbsmtp | batched, compressed (compress) mail
rbcsmtp | batched, compressed (compress) mail
rfbsmtp | batched, compressed (freeze) mail
rbfsmtp | batched, compressed (freeze) mail
rgbsmtp | batched, compressed (gzip) mail
rbgsmtp | batched, compressed (gzip) mail
@end example

@item grade
Every @ref{UUCP} job has a grade which is one letter out of @code{0..9},
@code{A..Z} or @code{a..z}. It specifies the transfer priority for the job.
@code{0} has the highest priority and @code{z} the lowest. The default is
@code{A}.

@item batch
This optional parameter specifies if several mail message should be
merged into one file (batching).

@item compress command
Name of a command for the optional compression of the data files. Legal
values are @code{compress}, @code{freeze} and @code{gzip}. If you leave
this field empty the data files won't be compressed.

@item header
If this field is not empty, then the following header line will be
created in the data file:

@example
#! <header text>
@end example

Mail batches normally don't contain any headers, so you may leave this
field empty.

@item max size
Maximum size for one batch file. When it reaches this size it will be
closed and a new batch file will be used. The default size is 65536
bytes.
@end table
@end table

Example for unbatched, uncompressed mail:

@example
( uucp.mailexport "rmail" )
@end example

Example for batched, uncompressed mail, no header, batch size 65536 Bytes:

@example
( uucp.mailexport "rbsmtp,A,Y" )
@end example

Example for batched, compressed mail, no header, batch size 300000 Bytes:

@example
( uucp.mailexport "rcbsmtp,0,Y,compress,,300000" )
@end example

@comment ----------------------------------------------------------------





@comment ----------------------------------------------------------------
@comment --- Subsection: Configuration/UMS Config/uucp.newsexport     ---
@comment ----------------------------------------------------------------
@node uucp.newsexport, uucp.nodename, uucp.mailexport, UMS Configuration
@subsection Configuration for news batches
@cindex uucp.newsexport

@table @code
@item uucp.newsexport
This variable tells the @ref{UUCP} exporter @ref{ums2uucp} in which
format news messages should be exported. The syntax is as follows:

@example
<command>[,<grade>[,<batch>[,<compress command>[,<header>[,<max size>]]]]]
@end example

@table @code
@item command
Command name for the @code{C} line in the @ref{UUCP} command file. You
have to talk to the administrator of your host which commands his
software accepts and how it interprets them. Normally you should set this
to @code{rnews}.

@item grade
Every @ref{UUCP} job has a grade which is one letter out of @code{0..9},
@code{A..Z} or @code{a..z}. It specifies the transfer priority for the job.
@code{0} has the highest priority and @code{z} the lowest. The default is
@code{A}.

@item batch
This optional parameter specifies if several news message should be
merged into one file (batching). Usually non-batched news transfer is a
bad idea.

@item compress command
Name of a command for the optional compression of the data files. Legal
values are @code{compress}, @code{freeze} and @code{gzip}. If you leave
this field empty the data files won't be compressed.

@item header
If this field is not empty, then the following header line will be
created in the data file:

@example
#! <header text>
@end example

The following table shows some common header texts:

@example
Name     | Meaning
------------------------------------------------
cunbatch | batched, compressed (compress) news
funbatch | batched, compressed (freeze) news
gunbatch | batched, compressed (gzip) news
@end example

You should talk to the administrator of your host which headers and
compression types are supported by the software of the host system.

@item max size
Maximum size for one batch file. When it reaches this size it will be
closed and a new batch file will be used. The default size is 65536
bytes.
@end table
@end table

Example for batched, uncompressed news, no header, batch size 300000 Bytes:

@example
( uucp.newsexport "rnews,A,Y,,,300000" )
@end example

Example for batched, compressed news, with header (#! cunbatch), batch size
65536 Bytes:

@example
( uucp.newsexport "rnews,B,Y,compress,cunbatch" )
@end example

@comment ----------------------------------------------------------------





@comment ----------------------------------------------------------------
@comment --- Subsection: Configuration/UMS Config/uucp.nodename       ---
@comment ----------------------------------------------------------------
@node uucp.nodename, uucp.envelope, uucp.newsexport, UMS Configuration
@subsection UUCP node name for your system
@cindex uucp.nodename

@table @code
@item uucp.nodename
@ref{UUCP} node name for your system. This name should not be longer than
7 characters.
@end table

Example:

@example
( uucp.nodename "foobar" )
@end example

@comment ----------------------------------------------------------------





@comment ----------------------------------------------------------------
@comment --- Subsection: Configuration/UMS Config/uucp.envelope       ---
@comment ----------------------------------------------------------------
@node uucp.envelope, uucp.keepdupes, uucp.nodename, UMS Configuration
@subsection UUCP envelope type
@cindex uucp.envelope

@table @code
@item uucp.envelope
When exporting mail messages over @ref{UUCP} a so called ``envelope'' around
the message is sometimes required, e.g. if your mail is sent unbatched via
@code{rmail}. The following values are allowed for this variable:

@example
0  Don't generate an envelope
1  Generate a RFC 976 style UUCP envelope
2  Generate a normal UUCP envelope        (Default)
@end example

With a given address and date, a normal envelope looks like this:

@example
From <user>@@<domain> <date>
@end example

A RFC 976 envelope looks like this:

@example
From <user> <date> remote from <domain>
@end example

Ask the administrator of your host what envelope type is expected by the
software used on the host system.
@end table

Example:

@example
( uucp.envelope 1 )
@end example

@comment ----------------------------------------------------------------





@comment ----------------------------------------------------------------
@comment --- Subsection: Configuration/UMS Config/uucp.keepdupes      ---
@comment ----------------------------------------------------------------
@node uucp.keepdupes, uucp.logdupes, uucp.envelope, UMS Configuration
@subsection Keep dupe messages
@cindex uucp.keepdupes

@table @code
@item uucp.keepdupes
Normally duplicate messages are flagged as error and the  @ref{UUCP}  job
is not deleted. Sometimes dupes can't be avoided, because @ref{UUCP} is
an offline protocol, that means there is no way to detect dupes on the
sending hosts. On a system with several hosts and high traffic this can
result in a high number of bad job files lying around, which do not
contain real errors. If you set this variable to @code{N} then bad jobs,
which had only dupe errors, are automatically deleted.
@end table

Example:

@example
( uucp.keepdupes "N" )
@end example

@comment ----------------------------------------------------------------





@comment ----------------------------------------------------------------
@comment --- Subsection: Configuration/UMS Config/uucp.logdupes       ---
@comment ----------------------------------------------------------------
@node uucp.logdupes, uucp.mailroute, uucp.keepdupes, UMS Configuration
@subsection Log dupe messages
@cindex uucp.logdupes

@table @code
@item uucp.logdupes
Normally duplicate messages are also logged in the UMS RFC error log. If
you don't want to bother with dupes then you should set this to @code{N}.
@xref{uucp.keepdupes}, for more information.
@end table

Example:

@example
( uucp.logdupes "N" )
@end example

@comment ----------------------------------------------------------------





@comment ----------------------------------------------------------------
@comment --- Subsection: Configuration/UMS Config/uucp.mailroute      ---
@comment ----------------------------------------------------------------
@node uucp.mailroute, uucp.recipients, uucp.logdupes, UMS Configuration
@subsection Mail routing configuration
@cindex uucp.mailroute

@table @code
@item uucp.mailroute
Sometimes it's important to specify static routes for outgoing mails.
Normally this variable is empty so NO routing information will be
generated. Each line in this variable consists of an address pattern and
a list of hosts:

@example
<address pattern> [<host1>[,<host2>....]]
@end example

When the recipient address matches the address pattern and the host list
is not empty, then the @ref{UUCP} exporter @ref{ums2uucp} will create an
explicit mail route. The route information will be added AFTER the
address conversion for non RFC compliant nets has been applied.

If the variable does contain several lines, then the patterns will be
tried from the first line down to the last line until one pattern
matches. If no pattern matches, then no route information will be
generated.
@end table

Example:

@ifinfo
@example
( uucp.mailroute "#?.maus.de        host1\\n"
                 "#?.zer.sub.org    host2,host3\\n"
                 "#?.fidonet.org\\n"
                 "#?.de             host4\\n" )
@end example
@end ifinfo
@iftex
@example
( uucp.mailroute "#?.maus.de        host1\n"
                 "#?.zer.sub.org    host2,host3\n"
                 "#?.fidonet.org\n"
                 "#?.de             host4\n" )
@end example
@end iftex

The following test cases:

@example
user@@box.maus.de
user@@box.zer.sub.org
user@@f7.n242.z2.fidonet.org
user@@zauber.nase.de
user@@dummy.blubb.edu
@end example

are translated to the following mail routes for non-batched mail tranfer:

@example
C rmail host1!box.maus.de!user
C rmail host2!host3!box.zer.sub.org!user
C rmail user@@f7.n242.z2.fidonet.org
C rmail host4!zauber.nase.de!user
C rmail user@@dummy.blubb.edu
@end example

and to the following mail routes for batched mail transfer:

@example
RCPT TO: <@@host1:user@@box.maus.de>
RCPT TO: <@@host2,@@host3:user@@box.zer.sub.org>
RCPT TO: <user@@f7.n242.z2.fidonet.org>
RCPT TO: <@@host4:user@@zauber.nase.de>
RCPT TO: <user@@dummy.blubb.edu>
@end example

@comment ----------------------------------------------------------------





@comment ----------------------------------------------------------------
@comment --- Subsection: Configuration/UMS Config/uucp.recipients     ---
@comment ----------------------------------------------------------------
@node uucp.recipients, , uucp.mailroute, UMS Configuration
@subsection Limit number of recipients
@cindex uucp.recipients

@table @code
@item uucp.recipients
Limits the number of recipients for one mail file. If there are more
recipients then the @ref{UUCP} exporter @ref{ums2uucp} will create
several mail files for one mail. The default is to put all recipients in
one mail file.
@end table

Example:

@example
( uucp.recipients 2 )
@end example

Using the above example on a mail with three recipients will result in
two separate mails to be send. If unbatched mail transfer is used then
the following two @code{C} lines are generated in the command files:

@example
C rmail usera@@test1.foo.bar,userb@@test2.foo.bar
C rmail userc@@test3.foo.bar
@end example

If batched  mail  transfer  is  used  then  the  following  commands  are
generated in the batch file:

@example
MAIL FROM: <user@@test.foo.bar>
RCPT TO: <usera@@test1.foo.bar>
RCPT TO: <userb@@test2.foo.bar>
DATA
...
MAIL FROM: <user@@test.foo.bar>
RCPT TO: <userc@@test3.foo.bar>
DATA
@end example

@comment ----------------------------------------------------------------





@comment ----------------------------------------------------------------
@comment --- Chapter: Commands                                        ---
@comment ----------------------------------------------------------------
@node Commands, Examples, Configuration, Top
@chapter Description of all UMS RFC commands
@cindex Command Syntax
@cindex Command Line Options

This section contains the description of all UMS RFC commands and their
command line options. The following syntax will be used:

@example
        -a          Mandantory command line option. You may not omit this
                    option when using a command.

        -b <...>    Command line option with parameter. The space between
                    the option and the parameter is optional.

        [-c <...>]  Optional command line option. You may omit this option
                    when using a command. The default value will be used
                    instead.
@end example

The commands usually return 0 on success and 20 on error. Exceptions to this
rule will be mentioned in the description for the tool.

The following UMS RFC commands are available:
@menu
Commands for NNTP:

* umsnntp::          NNTP exporter
* umsnntpd::         NNTP importer (server daemon)
* umsnntpget::       NNTP importer
* setnntpdate::      Set date of the last NNTP access

Commands for POP3:

* umspop3::          POP3 importer
* umspop3d::         POP3 exporter (server daemon)

Commands for SMTP:

* umssmtp::          SMTP exporter
* umssmtpd::         SMTP importer (server daemon)

Commands for UUCP:

* ums2uucp::         UUCP exporter
* uuxqt::            UUCP importer

Tools:

* umsrfcprint::      Print UMS messages as RFC messages
* umsrfcwrite::      Import RFC messages as UMS messages

Common command line options:

* UMS options::      Common UMS related command line options
* Protocol options:: Common protocol related command line options
@end menu

@comment ----------------------------------------------------------------





@comment ----------------------------------------------------------------
@comment --- Section: Commands/umsnntp                                ---
@comment ----------------------------------------------------------------
@node umsnntp, umsnntpd, , Commands
@section NNTP exporter
@cindex NNTP exporter
@cindex umsnntp

The command @code{umsnntp} exports news messages via @ref{NNTP} to a
remote host.

@example
Syntax:

        umsnntp  -h <remote host>
                [-S <Service name or port number>]

                [-p <UMS password>]
                [-s <UMS message base>]
                [-u <UMS user name>]

                [-n <authentication name>]
                [-i <authentication password>]

                [-b <select bit>]
                [-I]

                [-c]
                [-r <n>]
                [-w <n>]

Defaults:

                 -S  Service @samp{nntp} (port 119)
                 -p  No password
                 -s  Default message base
                 -u  User @samp{NNTP}
                 -n  jduser
                 -i  secret
                 -b  No select bit
                 -r  360
                 -w  60
@end example

Explanation of the command line options:

@table @samp
@item -n
This name is send to the remote server if it requests user
authentication. Note that the name is send in clear text!

@item -i
This string is send to the remote server if the user authentication also
requires a password. Note that the password is send in clear text!

@item -b
Bit number of an additional select bit. Only messages with this bit set
in the user flags will be exported. This can be used to run a special
filter program before exporting with @code{umsnntp}.

@item -I
Use the @code{IHAVE} command to send the news messages to the remote
host. If you don't specify this option, then the @code{POST} command is
used. Most @ref{NNTP} servers don't allow the @code{IHAVE} command for
clients, so you may not be able to use this option.

@item -c
Export news messages continously. After exporting all new messages
@code{umsnntp} will wait a short while and then will start to export all new
messages which arrived in the meantime.

@item -r
Causes @code{umsnntp} to rescan the whole message base for new messages
after @var{n} export runs. All messages which couldn't be sent until now
will be sent again. This option is only useful in conjunction with
@samp{-c}.

@item -w
Causes @code{umsnntp} to wait @var{n} seconds between two export runs.
This option is only useful inconjunction with @samp{-c}.
@end table

See also @ref{UMS options} and @ref{Protocol options}.

@comment ----------------------------------------------------------------





@comment ----------------------------------------------------------------
@comment --- Section: Commands/umsnntpd                               ---
@comment ----------------------------------------------------------------
@node umsnntpd, umsnntpget, umsnntp, Commands
@section NNTP importer (server daemon)
@cindex NNTP server daemon
@cindex umsnntpd

The command @code{umsnntpd} is a server daemon which offers a transfer
service for news messages via @ref{NNTP} to remote clients. It can only
be executed by the Internet Super Daemon @ref{Inetd}.

@example
Syntax:

        umsnntpd [<UMS user name>]
                 [<UMS password>]
                 [<UMS message base>]

Defaults:

                 User @samp{NNTPD}
                 No password
                 Default message base
@end example

@comment ----------------------------------------------------------------





@comment ----------------------------------------------------------------
@comment --- Section: Commands/umsnntpget                             ---
@comment ----------------------------------------------------------------
@node umsnntpget, setnntpdate, umsnntpd, Commands
@section NNTP importer
@cindex NNTP importer
@cindex umsnntpget

The command @code{umsnntpget} imports news messages via @ref{NNTP} from a
remote host.

@example
Syntax:

        umsnntpget  -h <remote host>
                   [-S <Service name or port number>]

                   [-p <UMS password>]
                   [-s <UMS message base>]
                   [-u <UMS user name>]

                   [-n <authentication name>]
                   [-i <authentication password>]

                   [-g]
                   [-P <n>]
                   [-c]
                   [-q]

Defaults:

                    -S  Service @samp{nntp} (port 119)
                    -p  No password
                    -s  Default message base
                    -u  User @samp{NNTP}
                    -n  jduser
                    -i  secret
                    -P  1
@end example

Explanation of the command line options:

@table @samp
@item -n
This name is send to the remote server if it requests user
authentication. Note that the name is send in clear text!

@item -i
This string is send to the remote server if the user authentication also
requires a password. Note that the password is send in clear text!

@item -g
Use the commands @code{GROUP}, @code{STAT} and @code{NEXT} to scan for
new news messages. This may lower the load on the remote @ref{NNTP}
server but the transfer overhead may be significantely higher, especially
if the server uses long expiration periods for news messages. If you
don't specify this option the command @code{NEWNEWS} is used which only
transfers information about new messages (@pxref{setnntpdate}).

@item -P
Use @var{n} (up to 20) parallel processes to transfer news messages. This
is useful to make the transfer of news messages over a slow serial link
more efficient by using most of the available bandwidth.

@item -c
Use parameters on the command line after the last valid command line
option as input parameters. You may request news messages by message ID:

@example
umsnntpget -h nntphost -c 1234@@dummy.do.main 6789@@test.foo.bar
@end example

or by group names:

@example
umsnntpget -h nntphost -c -g comp.sys.amiga.misc comp.sys.amiga.datacomm
@end example

@item -q
Don't print message ids.
@end table

See also @ref{nntpget.groups}, @ref{UMS options} and @ref{Protocol options}.

@comment ----------------------------------------------------------------





@comment ----------------------------------------------------------------
@comment --- Section: Commands/setnntpdate                            ---
@comment ----------------------------------------------------------------
@node setnntpdate, umspop3, umsnntpget, Commands
@section Set date of the last NNTP access
@cindex Set date of the last NNTP access
@cindex setnntpdate

The command @code{setnntpdate} sets the date of the last @ref{NNTP} access for
@ref{umsnntpget}.

@example
Syntax:

        setnntpdate  <UMS user name>
                     <UMS password>
                    [<UMS message base>]
                    [DATE <date string>]

Defaults:

                     Default message base
@end example

Explanation of the command line options:

@table @samp
@item DATE
You have to specify the date as defined in RFC 822. @code{setnntpdate} will
transform it to the format which is needed for @ref{umsnntpget}. If you don't
specify this option the last access date will be printed in RFC 822 format.
@end table

Example:

@example
setnntpdate NNTP secret DATE "18 Nov 1994 20:05:30"
@end example

@comment ----------------------------------------------------------------





@comment ----------------------------------------------------------------
@comment --- Section: Commands/umspop3                                ---
@comment ----------------------------------------------------------------
@node umspop3, umspop3d, setnntpdate, Commands
@section POP3 importer
@cindex POP3 importer
@cindex umspop3

The command @code{umspop3} imports mail messages via @ref{POP3} from a
remote host.

@example
Syntax:

        umspop3  -h <remote host>
                [-S <Service name or port number>]

                [-p <UMS password>]
                [-s <UMS message base>]
                [-u <UMS user name>]

                 -n <mailbox name>
                 -i <mailbox password>
                [-k]

                [-c]
                [-w <n>]

Defaults:

                 -S  Service @samp{pop3} (port 110)
                 -p  No password
                 -s  Default message base
                 -u  User @samp{POP3}
                 -w  60
@end example

Explanation of the command line options:

@table @samp
@item -n
Name of the mailbox on the remote host.

@item -i
Password for the mailbox on the remote host.

@item -k
If this option is used @code{umspop3} won't delete the messages in the
mailbox on the remote host. Note that these messages will show up as dupes
next time you use @code{umspop3}!

@item -c
Import mail messages continously. After importing all new messages
@code{umspop3} will wait a short while and then will start to import all new
messages which arrived in the meantime.

@item -w
Causes @code{umspop3} to wait @var{n} seconds between two export runs.
This option is only useful inconjunction with @samp{-c}.
@end table

@code{umspop3} returns 5 if there were no messages in the mailbox.

See also @ref{UMS options} and @ref{Protocol options}.

@comment ----------------------------------------------------------------





@comment ----------------------------------------------------------------
@comment --- Section: Commands/umspop3d                               ---
@comment ----------------------------------------------------------------
@node umspop3d, umssmtp, umspop3, Commands
@section POP3 exporter (server daemon)
@cindex POP3 server daemon
@cindex umspop3d

The command @code{umspop3d} is a server daemon which offers a transfer
service for mail messages via @ref{POP3} to remote clients. It can only
be executed by the Internet Super Daemon @ref{Inetd}.

@example
Syntax:

        umspop3d [<UMS user name>]
                 [<UMS password>]
                 [<UMS message base>]

Defaults:

                 User @samp{POP3D}
                 No password
                 Default message base
@end example

@comment ----------------------------------------------------------------





@comment ----------------------------------------------------------------
@comment --- Section: Commands/umssmtp                                ---
@comment ----------------------------------------------------------------
@node umssmtp, umssmtpd, umspop3d, Commands
@section SMTP exporter
@cindex SMTP exporter
@cindex umssmtp

The command @code{umssmtp} exports mail messages via @ref{SMTP} to a
remote host.

@example
Syntax:

        umssmtp  -h <remote host>
                [-S <Service name or port number>]

                [-p <UMS password>]
                [-s <UMS message base>]
                [-u <UMS user name>]

                [-b <select bit>]

                [-c]
                [-r <n>]
                [-w <n>]

Defaults:

                 -S  Service @samp{smtp} (port 25)
                 -p  No password
                 -s  Default message base
                 -u  User @samp{SMTP}
                 -b  No select bit
                 -r  360
                 -w  60
@end example

Explanation of the command line options:

@table @samp
@item -b
Bit number of an additional select bit. Only messages with this bit set
in the user flags will be exported. This can be used to run a special
filter program before exporting with @code{umssmtp}.

@item -c
Export mail messages continously. After exporting all new messages
@code{umssmtp} will wait a short time and then will start to export all new
messages which arrived in the meantime.

@item -r
Causes @code{umssmtp} to rescan the whole message base for new messages
after @var{n} export runs. All messages which couldn't be sent until now
will be sent again. This option is only useful in conjunction with
@samp{-c}.

@item -w
Causes @code{umssmtp} to wait @var{n} seconds between two export runs.
This option is only useful inconjunction with @samp{-c}.
@end table

See also @ref{UMS options} and @ref{Protocol options}.

When @code{umssmtp} connects to a @ref{SMTP} server it tries to use
@ref{SMTP,ESMTP} first. If this succeeds then @code{umssmtp} controls its
actions according to the options offered by the server:

@table @samp
@item SIZE
It will tell the server how big the messages are before transferring them.

@item 8BITMIME
It will tell the server which encoding type the message body has before
transferring it.

@item PIPELINING
Commands and responses will be queued to improve performance.
@end table

@comment ----------------------------------------------------------------





@comment ----------------------------------------------------------------
@comment --- Section: Commands/umssmtpd                               ---
@comment ----------------------------------------------------------------
@node umssmtpd, ums2uucp, umssmtp, Commands
@section SMTP importer (server daemon)
@cindex SMTP server daemon
@cindex umssmtpd

The command @code{umssmtpd} is a server daemon which offers a transfer
service for mail messages via @ref{SMTP} to remote clients. It can only
be executed by the Internet Super Daemon @ref{Inetd}.

@example
Syntax:

        umssmtpd [<UMS user name>]
                 [<UMS password>]
                 [<UMS message base>]

Defaults:

                 User @samp{SMTPD}
                 No password
                 Default message base
@end example

This server offers the @ref{SMTP,ESMTP} extensions SIZE, 8BITMIME and
PIPELINING. It uses the @ref{UMS} configuration variable @code{MaxMsgSize} to
tell the client what is the maximum message size which it will accept.

@comment ----------------------------------------------------------------





@comment ----------------------------------------------------------------
@comment --- Section: Commands/ums2uucp                               ---
@comment ----------------------------------------------------------------
@node ums2uucp, uuxqt, umssmtpd, Commands
@section UUCP exporter
@cindex UUCP exporter
@cindex ums2uucp

The command @code{ums2uucp} exports mail and news messages and creates
batch files for the transfer via @ref{UUCP} to a remote host.

@example
Syntax:

        ums2uucp  -h <remote host>

                 [-p <UMS password>]
                 [-s <UMS message base>]
                 [-u <UMS user name>]

                 [-b <select bit>]
                 [-d <n>]
                 [-t]

Defaults:

                  -p  No password
                  -s  Default message base
                  -u  User @samp{UUCP}
                  -b  No select bit
                  -d  0
@end example

Explanation of the command line options:

@table @samp
@item -h
UUCP node name of the remote host.

@item -b
Bit number of an additional select bit. Only messages with this bit set
in the user flags will be exported. This can be used to run a special
filter program before exporting with @code{ums2uucp}.

@item -d
UUCP log level: 0, 1, 2 or 3.

@item -t
Use the TaylorUUCP spool directory layout instead of the normal one. The
command files are also optimized. You @emph{must} use the @code{uucico}
from the TaylorUUCP package when you specify this option!
@end table

See also @ref{UMS options}.

@comment ----------------------------------------------------------------





@comment ----------------------------------------------------------------
@comment --- Section: Commands/uuxqt                                  ---
@comment ----------------------------------------------------------------
@node uuxqt, umsrfcprint, ums2uucp, Commands
@section UUCP importer
@cindex UUCP importer
@cindex uuxqt

The command @code{uuxqt} imports mail and news messages from batch files
which were transferred via @ref{UUCP} from a remote host. This command is
normally only executed by the @ref{UUCP} transfer program @code{uucico}.

@example
Syntax:

        uuxqt [-p <UMS password>]
              [-d <n>]
              [<subdirectory>]

Defaults:

               -p  No password
               -d  0
@end example

Explanation of the command line options:

@table @samp
@item -p
Password for the UMS user. This will be used for @strong{ALL} logins.

@item -d
UUCP log level: 0, 1, 2 or 3.

@item subdirectory
Name of the subdirectory in the @ref{UUCP} spool directory. This
parameter is supplied by @code{uucico} when a hierarchical spool
directory structure is used.
@end table

@code{uuxqt} uses the following system for the @ref{Users,UMS user
names}:

@example
uucp.default  Used as the first login to read the configuration.

uucp.<name>   Used when importing a batch from node @samp{name}.
@end example

@comment ----------------------------------------------------------------





@comment ----------------------------------------------------------------
@comment --- Section: Commands/umsrfcprint                            ---
@comment ----------------------------------------------------------------
@node umsrfcprint, umsrfcwrite, uuxqt, Commands
@section Print UMS messages as RFC messages
@cindex Print UMS messages as RFC messages
@cindex umsrfcprint

The command @code{umsrfcprint} prints @ref{UMS} messages as @ref{RFC} messages
to the standard output. You can use the output redirection (@file{>file}) to
write messages into a file. The printed messages will @emph{not} be marked as
read. This tool is very helpful when you want to know how the exported message
will look like.

@example
Syntax:

        umsrfcprint  <UMS user name>
                     <UMS password>
                    [<UMS message base>]
                     MSGNUM <n>
                    [DOTS]

Defaults:

                     Default message base
                     Don't encode lines beginning with a dot
@end example

Explanation of the command line options:

@table @samp
@item MSGNUM
Specifies which @ref{UMS} message should be printed.

@item DOTS
Some protocols like @ref{NNTP}, @ref{POP3} and @ref{SMTP} require a special
encoding of lines beginning with a "." (dot). With this option you can enable
this encoding.
@end table

Example:

@example
umsrfcprint >message.txt exporter secret MSGNUM 1234 DOTS
@end example

@comment ----------------------------------------------------------------





@comment ----------------------------------------------------------------
@comment --- Section: Commands/umsrfcwrite                            ---
@comment ----------------------------------------------------------------
@node umsrfcwrite, UMS options, umsrfcprint, Commands
@section Import RFC messages as UMS messages
@cindex Import RFC messages as UMS messages
@cindex umsrfcwrite

The command @code{umsrfcwrite} reads @ref{RFC} messages and imports them as
@ref{UMS} messages. You can use the input redirection (@file{<file}) or the
@code{FILE} parameter to read messages from a file. This tool is very helpful
when you want to import messages from other RFC tools.

@example
Syntax:

        umsrfcwrite  <UMS user name>
                     <UMS password>
                    [<UMS message base>]
                    [FILE <filename>]
                    [DOTS]
                    [<recipient1> ...]

Defaults:

                     Default message base
                     Read message from standard input
                     Don't decode lines beginning with a dot
                     Import news message
@end example

Explanation of the command line options:

@table @samp
@item FILE
Specifies from which file the @ref{RFC} message should be read instead from the
standard input.

@item DOTS
Some protocols like @ref{NNTP}, @ref{POP3} and @ref{SMTP} require a special
decoding of lines beginning with a "." (dot). With this option you can enable
this decoding. Note that the last line containing only a single dot must have
been removed already!
@end table

If you don't specify any recipient addresses then the message will be imported
as news message. If you specify one or more addresses the message will imported
as mail message with the addresses as recipients.

Note that @code{umsrfcwrite} does @emph{not} transform the @code{CR-LF}
combination to @code{LF}. Thus you have to perform this transformation
yourself before importing the message with this tool!

Examples:

@example
umsrfcwrite importer secret FILE mail.txt DOTS jduser@@dummy.test.com
umsrfcwrite <news.txt importer secret
@end example

@comment ----------------------------------------------------------------





@comment ----------------------------------------------------------------
@comment --- Section: Commands/UMS options                            ---
@comment ----------------------------------------------------------------
@node UMS options, Protocol options, umsrfcwrite, Commands
@section Common UMS related command line options

The following @ref{UMS} related comand line options are understood by
most of the programs:

@table @samp
@item -p
Password for the @ref{Users,UMS user}.

@item -s
Name of the @ref{Message Base,UMS message base} used for importing or
exporting.

@item -u
Name of the @ref{Users,UMS user}.
@end table

@comment ----------------------------------------------------------------





@comment ----------------------------------------------------------------
@comment --- Section: Commands/Protocol options                       ---
@comment ----------------------------------------------------------------
@node Protocol options, , UMS options, Commands
@section Common protocol related command line options

The following protocol related comand line options are understood by most
of the programs:

@table @samp
@item -h
Name or IP number of the remote host.

@item -S
Name or port number of the @ref{TCP/IP} service. All services have
default port numbers so this option should normally not be used.

@end table

@comment ----------------------------------------------------------------





@comment ----------------------------------------------------------------
@comment --- Appendix: Examples                                       ---
@comment ----------------------------------------------------------------
@node Examples, Answers, Commands, Top
@appendix Some example UMS RFC configurations
@cindex Examples

@menu
UMS Configuration:

* User::             Configuration for an UMS RFC user
* Leaf Node::        Configuration for a simple node
* Full Node::        Configuration for a full node
* Gateway::          Configuration for a gateway.

TCP/IP Configuration:

* Access::           Security configuration
* Inetd::            Configuration for the Internet Super Daemon
* Services::         Services configuration

UUCP Configuration:

* UUCPDirectories::  UUCP directories
* UUCPConfig::       Main UUCP configuration
* L.Ports::          UUCP ports description
* L.Sys::            Remote UUCP systems
* UUCPPoll::         UUCP poll script

Configuration for UUCP over TCP/IP:

* Telser::           Telser configuration
* Hosts::            Host descriptions
@end menu

@comment ----------------------------------------------------------------





@comment ----------------------------------------------------------------
@comment --- Section: Examples/User                                   ---
@comment ----------------------------------------------------------------
@node User, Leaf Node, , Examples
@section Configuration for an UMS RFC user
@cindex User Configuration

Excerpt from the file @file{ums.config}:

@example
( User
    ( Alias
        jduser
        postmaster
    )
    ( Readaccess "#?" )
    ( Writeaccess "#?" )
    ( Netaccess "#?" )
    ( Import "%" )
    ( Name "Joe Dumb User" )
    ( Password "top secret" )
    ( rfc.username "jduser" )
)
@end example

@comment ----------------------------------------------------------------





@comment ----------------------------------------------------------------
@comment --- Section: Examples/Leaf Node                              ---
@comment ----------------------------------------------------------------
@node Leaf Node, Full Node, User, Examples
@section Configuration for a simple node
@cindex Leaf Node Configuration

Excerpt from the file @file{ums.config}:

@example
( Aka "#?@@test.(foo.bar|uucp)" )
( NodeMode "%" )

( Exporter
    ( Alias
        uucp.default
        NNTP
        POP3
        SMTP
        UUCP
    )
    ( Readaccess "#?" )
    ( Writeaccess "#?" )
    ( Netaccess "#?" )
    ( Import "#?" )
    ( Export "#?" )
    ( Distribution "#?" )
    ( Name uucp.dummy )
    ( Password  )
    ( rfc.domainname "test.foo.bar" )
    ( rfc.mailencoding 0 )
    ( rfc.newsencoding 0 )
    ( rfc.pathname "test.foo.bar" )
    ( uucp.envelope 1 )
    ( uucp.keepdupes "Y" )
    ( uucp.logdupes "Y" )
    ( uucp.mailexport "rmail,A,N" )
    ( uucp.newsexport "rnews,B,Y,compress,cunbatch,300000" )
    ( uucp.nodename "test" )
)
@end example

This system receives all messages from the node @code{dummy}. It will
send out unbatched mail messages and batched, compressed news messages to
the node @code{dummy}. The transmission link to the remote node is
transparent, so no encoding is needed. The users on this system will have
the address

@example
<user>@@test.foo.bar
@end example

To poll the node @code{dummy} with @ref{UUCP} the user has to use the
following commands:

@example
ums2uucp -h dummy
uucico -sdummy
@end example

To send out mail and news message via @ref{TCP/IP} and to fetch mail
messages the user has to use the following commands:

@example
umssmtp -h dummy.foo.bar
umsnntp -h dummy.foo.bar
umspop3 -h dummy.foo.bar -n jduser -i more_top_secret
@end example

@comment ----------------------------------------------------------------





@comment ----------------------------------------------------------------
@comment --- Section: Examples/Full Node                              ---
@comment ----------------------------------------------------------------
@node Full Node, Gateway, Leaf Node, Examples
@section Configuration for a full node
@cindex Full Node Configuration

Excerpt from the file @file{ums.config}:

@example
( Aka "#?@@test.(foo.bar|uucp)" )
( NodeMode "(comp|sci).#?" )

( rfc.domainname "test.foo.bar" )
( rfc.pathname "test.foo.bar" )

( Exporter
    ( Alias
        uucp.default
        UUCP
    )
    ( Readaccess "(comp|sci).#?" )
    ( Writeaccess "(comp|sci).#?" )
    ( Netaccess "#?" )
    ( Import "#?" )
    ( Export "#?.foo.bar" )
    ( Distribution "#?" )
    ( Name uucp.dummy )
    ( Password  )
    ( rfc.mailencoding 1 )
    ( rfc.newsencoding 1 )
    ( uucp.envelope 0 )
    ( uucp.keepdupes "N" )
    ( uucp.logdupes "Y" )
    ( uucp.mailexport "rbsmtp,A,Y" )
    ( uucp.newsexport "rnews,B,Y,gzip,gunbatch" )
    ( uucp.nodename "test" )
)

( Exporter
    ( Alias
        NNTPD
        POP3
        POP3D
        SMTP
        SMTPD
    )
    ( Readaccess "#?" )
    ( Writeaccess "#?" )
    ( Netaccess "#?" )
    ( Import "#?" )
    ( Export "~(#?.foo.bar)" )
    ( Distribution "#?" )
    ( Name NNTP )
    ( Password  )
    ( rfc.mailencoding 0 )
    ( rfc.newsencoding 0 )
)
@end example

In this setup the newsgroup hierarchies @code{comp} and @code{sci} are
distributed between two systems. Mail messages to the domain
@code{foo.bar} are send out with @ref{UUCP}, all others are send out with
@ref{SMTP}.

@comment ----------------------------------------------------------------





@comment ----------------------------------------------------------------
@comment --- Section: Examples/Gateway                                ---
@comment ----------------------------------------------------------------
@node Gateway, Access, Full Node, Examples
@section Configuration for a gateway
@cindex Gateway Configuration

Excerpt from the file @file{ums.config}:

@example
( Netgroup
    "comp.sys.amiga.misc"
    "fidonet.comp.sys.amiga.misc"
)

( Exporter
    ( Alias
        uucp.default
        NNTP
        NNTPD
        POP3
        POP3D
        SMTP
        SMTPD
        UUCP
    )
    ( Readaccess "comp.#?" )
    ( Writeaccess "comp.#?" )
    ( Netaccess "#?" )
    ( Import "#?" )
    ( Export "~(#?@@fidonet)" )
    ( Distribution "#?" )
    ( Name uucp.dummy )
    ( Password  )
    ( rfc.domainname "test.foo.bar" )
    ( rfc.mailencoding 0 )
    ( rfc.newsencoding 0 )
    ( rfc.pathname "test.foo.bar" )
    ( uucp.envelope 0 )
    ( uucp.keepdupes "N" )
    ( uucp.logdupes "Y" )
    ( uucp.mailexport "rbsmtp,A,Y" )
    ( uucp.newsexport "rnews,B,Y,gzip,gunbatch" )
    ( uucp.nodename "test" )
)
@end example

The example shows the UMS RFC part of a gateway configuration. One
newsgroup is distributed between networks with different technologies
(gating).

@comment ----------------------------------------------------------------





@comment ----------------------------------------------------------------
@comment --- Section: Examples/Access                                 ---
@comment ----------------------------------------------------------------
@node Access, Inetd, Gateway, Examples
@section Security Configuration
@cindex Security Configuration

If you are using the commercial version of AmiTCP you may limit the
access to your system. The file @file{AMITCP:db/inet.access} contains the
rules which services may be used on your system. The following two lines
allow the access to the services @ref{NNTP} and @ref{SMTP} from any host
on the Internet:

@example
smtp    *   allow   LOG
nntp    *   allow   LOG
@end example

@comment ----------------------------------------------------------------





@comment ----------------------------------------------------------------
@comment --- Section: Examples/Inetd                                  ---
@comment ----------------------------------------------------------------
@node Inetd, Services, Access, Examples
@section Configuration for the Internet Super Daemon
@cindex Inetd Configuration

The daemons for the @ref{NNTP}, @ref{POP3} and @ref{SMTP} services are
started automatically when a remote host accesses those services. Add the
following three lines to the @code{inetd} configuration file
@file{AMITCP:db/inetd.conf}:

@example
smtp    stream  tcp nowait root UMS:Bin/umssmtpd umssmtpd
pop3    stream  tcp nowait root UMS:Bin/umspop3d umspop3d
nntp    stream  tcp nowait root UMS:Bin/umsnntpd umsnntpd
@end example

@comment ----------------------------------------------------------------





@comment ----------------------------------------------------------------
@comment --- Section: Examples/Services                               ---
@comment ----------------------------------------------------------------
@node Services, UUCPDirectories, Inetd, Examples
@section Services configuration
@cindex Services configuration

To use the services @ref{NNTP}, @ref{POP3} and @ref{SMTP} the file
@file{AMITCP:db/services} must contain the following lines:

@example
smtp    25/tcp
pop3    110/tcp
nntp    119/tcp
@end example

@comment ----------------------------------------------------------------





@comment ----------------------------------------------------------------
@comment --- Section: Examples/UUCPDirectories                        ---
@comment ----------------------------------------------------------------
@node UUCPDirectories, UUCPConfig, Services, Examples
@section UUCP directories
@cindex UUCP Directories

@ref{UUCP} needs three directories for its work:

@table @file
@item Spool directory
Directory for command, data and log files. You should assign @file{UUSPOOL:} to
this directory. Each remote site has its own subdirectory. In each subdirectory
there should be a directory @file{bad-jobs} for errorneous jobs. Example with
two remote systems named @code{dummy} and @code{test}:

@example
1> Dir UUSPOOL: all
     dummy (dir)
          bad-jobs (dir)
     test (dir)
          bad-jobs (dir)
  logfile                          TimeLog
  XferStat
@end example

@item Configuration directory
UUCP configuration file reside here. You should assign @file{UULIB:} to this
directory.

@example
1> Dir UULIB:
  config                           L.Ports
  L.Sys                            security
  seq
@end example

@item Public directory
Download and upload area for UUCP. You should assign @file{UUPUB:} to this
directory.
@end table

@comment ----------------------------------------------------------------





@comment ----------------------------------------------------------------
@comment --- Section: Examples/UUCPConfig                             ---
@comment ----------------------------------------------------------------
@node UUCPConfig, L.Ports, UUCPDirectories, Examples
@section Main UUCP configuration
@cindex UUCP Configuration

The file @file{UULIB:config} contains the @ref{UUCP} configuration. A minimal
configuration only needs your node name, a dummy user name and a setting for
the debug level:

@example
NodeName    foobar
UserName    root
Debug       0
@end example

@comment ----------------------------------------------------------------





@comment ----------------------------------------------------------------
@comment --- Section: Examples/L.Ports                                ---
@comment ----------------------------------------------------------------
@node L.Ports, L.Sys, UUCPConfig, Examples
@section UUCP ports description
@cindex UUCP ports description

The file @file{UULIB:L.Ports} describes the devices which @code{uucico} uses to
contact remote systems. Normally this is a serial port with a modem, so the
file should look like this:

@example
acu=ser device=serial.device unit=0
@end example

If you want to use @ref{UUCP} over @ref{TCP/IP} you must add another device to
this file:

@example
acu=tcp device=telser.device unit=0
@end example

@comment ----------------------------------------------------------------





@comment ----------------------------------------------------------------
@comment --- Section: Examples/L.Sys                                  ---
@comment ----------------------------------------------------------------
@node L.Sys, UUCPPoll, L.Ports, Examples
@section Remote UUCP Systems
@cindex Remote UUCP Systems

The file @file{UULIB:L.Sys} describes how @code{uucico} can contact remote
systems. This includes the device to use, the serial line speed, the phone
number and the login script:

@example
dummy    Any   ser   19200    ATDP123456  ogin: foobar ssword: hello
@end example

To contact a remote system with @ref{UUCP} over @ref{TCP/IP} you replace the
phone number with the hostname of the remote system. So the line should look
like this:

@example
dummy    Any   tcp   19200    dummy    ogin: foobar ssword: hello
@end example

@comment ----------------------------------------------------------------





@comment ----------------------------------------------------------------
@comment --- Section: Examples/UUCPPoll                               ---
@comment ----------------------------------------------------------------
@node  UUCPPoll, Telser, L.Sys, Examples
@section UUCP poll script
@cindex UUCP poll script

The following script first exports all messages to the remote system and then
contacts it. You can specify the name of the remote system as parameter:

@example
.KEY SYSTEM
.DEF SYSTEM dummy
.BRA @{
.KET @}

ums2uucp -h @{SYSTEM@}
uucico -s@{SYSTEM@}
@end example

@comment ----------------------------------------------------------------





@comment ----------------------------------------------------------------
@comment --- Section: Examples/Telser                                 ---
@comment ----------------------------------------------------------------
@node Telser, Hosts, UUCPPoll, Examples
@section Telser configuration
@cindex Telser Configuration

The file @file{AMITCP:db/telser.conf} describes the configuration for every
unit of the @code{telser.device}. For the unit 0 it should look like this:

@example
0  ""    NoOperation NOOPENWIN NOLINGER NODEBUG T:telser.log
@end example

@comment ----------------------------------------------------------------





@comment ----------------------------------------------------------------
@comment --- Section: Examples/Hosts                                  ---
@comment ----------------------------------------------------------------
@node Hosts, , Telser, Examples
@section Host descriptions
@cindex Host Descripions

The file @file{AMITCP:db/telser.hosts} contains the descriptions of the remote
systems. If you @emph{dial} a remote system using the @code{telser.device} it
looks into this file to find out the hostname and the parameters. For our
example system it looks like this:

@example
dummy    dummy.do.main  540   ""    00000000001
@end example

Please make sure that the 11th bit is set, because it activates the transparent
mode of @code{telser.device} which is a @strong{must} for @ref{UUCP} over
@ref{TCP/IP}.

@comment ----------------------------------------------------------------





@comment ----------------------------------------------------------------
@comment --- Appendix: Answers                                        ---
@comment ----------------------------------------------------------------
@node Answers, History, Examples, Top
@appendix Solutions to common UMS RFC problems
@cindex Common Problems
@cindex Questions & Answers

@table @samp
@item Error: No rfc.username for...
You have forgotten to add the variable @ref{rfc.username} to the user entry
of a lokal user.

@item Error: Can't write article, check WRITEACCESS!
UMS RFC couldn't write the article to @strong{any} newsgroup. You have
requested a newsgroup but forgotten to add it to the @code{WRITEACCESS} patter
of the importer entry. If you are receiving control messages you must also add
the newsgroup @code{rfc.control} to this pattern!

@item Error: Can't get connection data!
You forgot to update the service file of your AmiTCP installation.
@xref{Services}.

@end table

You should also read the FAQ about @ref{UMS}.

@comment ----------------------------------------------------------------





@comment ----------------------------------------------------------------
@comment --- Appendix: History                                        ---
@comment ----------------------------------------------------------------
@node History, TODO, Answers, Top
@appendix History of UMS RFC
@cindex History

The following list gives an overview of all the changes since the first
version of UMS RFC: For more detailed information read the file
@file{UMS/Networks/RFC/src/History}.

@table @emph

@item 1.0 (20-Feb-1999)
@itemize @minus

@item
New version scheme taken into use. Changes in the library interface
should be less error-prone now.

@item
Now all LFs are converted to CR-LF as required by the
@ref{Message Formats,RFC 822} format, except for unbatched UUCP mail
messages or UUCP news messages. All tools convert CR-LF to LF before
importing the messages.

@item
@ref{SMTP,ESMTP} extensions PIPELINING and 8BITMIME implemented.

@item
@ref{NNTP} extensions @code{XOVER} and @code{LIST OVERVIEW.FMT}
implemented in the server. The clients now support the
@code{AUTHINFO USER/PASS} authentication extension.

@item
Implemented some support for TaylorUUCP in The @ref{UUCP} tools.

@item
Added new tool @ref{umsrfcwrite} for importing RFC messages.

@item
The variable @code{uucp.dumbhost} has been replace with the variable
@ref{uucp.envelope}. This might require a change to your UMS configuration.

@item
The variables @code{uucp.debug}, @code{uucp.debugfile} and @code{uucp.filtercr}
have been removed.

@item
@ref{RFC} times is now converted to local time.

@item
Several bug fixes.
@end itemize

@item 0.12 (24-Mar-1996)
@itemize @minus

@item
Library interface changed. Please make sure that you only use the
programs and library of this version together.

@item
Support for systems without own domain name added, @pxref{rfc.noownfqdn}.
The variable @code{rfc.receivedname} has been removed.

@item
The variable @ref{rfc.pathname} is now optional.

@item
Support for @ref{SMTP,ESMTP} added. Currently implemented is the SIZE
extension.

@item
Added new tool @ref{umsrfcprint} for printing RFC messages.

@item
Added Installer script.

@item
Several bug fixes.
@end itemize

@item 0.11 (09-Dec-1995)
@itemize @minus
@item
@ref{POP3} server daemon @ref{umspop3d}.

@item
Support for UMS' binary field.

@item
Implemented UUCP grades. The format of the variables @ref{uucp.mailexport}
and @ref{uucp.newsexport} was changed!

@item
Implemented CTRL-C handling for @ref{umsnntpget}.

@item
Command logging for @ref{NNTP} and @ref{SMTP} server daemons.

@item
Switch for Daylight Saving Time.

@item
Several bug fixes and new command switches.
@end itemize

@item 0.10 (04-Jun-1995)
@itemize @minus
@item
First official release of the UMS RFC package the successor of the UMS UUCP
package.

@item
All RFC from/to UMS conversions are handled by the @file{umsrfc.library} now.

@item
Support for @ref{NNTP}, @ref{POP3}, @ref{SMTP}, @ref{UUCP}.

@item
New documentation.
@end itemize
@end table

For all older versions please read the file @file{History} in the source
directory.

@comment ----------------------------------------------------------------





@comment ----------------------------------------------------------------
@comment --- Appendix: TODO                                           ---
@comment ----------------------------------------------------------------
@node TODO, Authors address, History, Top
@appendix Not yet implemented features
@cindex TODO

The following things have not yet been implemented. The order shown below
doesn't imply any priority for an item:

@itemize @bullet
@item
MX support for the @ref{SMTP} exporter @ref{umssmtp}.

@item
Encoding/Decoding with @code{base64} in @ref{MIME} messages.

@item
@ref{MIME} encoding of messages headers.

@item
The address conversion should be freely configurable. Does anyone have a
good reference for a pattern search & replace algorithm (like in perl)?
@end itemize

@comment ----------------------------------------------------------------





@comment ----------------------------------------------------------------
@comment --- Appendix: Authors address                                ---
@comment ----------------------------------------------------------------
@node Authors address, Credits, TODO, Top
@appendix Where to send bug reports & comments
@cindex Address
@cindex Bug reports
@cindex Comments
@cindex E-Mail
@cindex Internet address
@cindex Postal address

The author can be reached at the following addresses:

@table @asis
@item Postal address:

@example

        Stefan Becker
        Richard-Wagner-Strasse 20
D-44651 Herne
        GERMANY
@end example

@item Internet Electronic Mail:

@example

stefanb@@yello.ping.de
@end example
@end table

@comment ----------------------------------------------------------------





@comment ----------------------------------------------------------------
@comment --- Appendix: Credits                                        ---
@comment ----------------------------------------------------------------
@node Credits, Index, Authors address, Top
@appendix The author wants to thank
@cindex Credits

I want to thank all users of the public beta version for testing this
package. Especially I want to thank Matthias Scheler and Bernhard
M�llemann who tested intermediate versions on their systems. Many thanks
also to the members of the @code{UMS-Dev} mailing list for the
discussions and the ideas.

Thanks go to Kai `wusel' Siering for allowing me to include his version
of @code{uucico} in this package. You may find the latest version of the
program in his @ref{UUCP} package called @code{wUUCP} on Aminet in the
directory @file{comm/uucp}.

Thanks also to Sam Yee for allowing me to include parts of his
@code{telser} package. You may find the latest version of his package on
Aminet in the directory @file{comm/tcp}. @code{telser} is Shareware, so
if you use it regularly @strong{PLEASE} do register it!

Thanks go to Kai Bolay for the original Installer script and the
UMS Installation ARexx tools.

Thanks also to Hartmut Goebel for allowing me to include his
@code{rexxdossupport.library} (Copyright @copyright{} 1994--1996 Hartmut
Goebel) in this package. This library is required for the UMS
Installation ARexx tools. The latest version of this package can be found
on Aminet in @file{util/rexx/rexxdossupport.lha}.

@comment ----------------------------------------------------------------





@comment ----------------------------------------------------------------
@comment --- Unnumbered chapter: Index                                ---
@comment ----------------------------------------------------------------
@node Index, , Credits, Top
@unnumbered Index


@printindex cp

@comment ----------------------------------------------------------------





@comment ----------------------------------------------------------------
@comment --- Table of Contents (appears only in printed manual)       ---
@comment ----------------------------------------------------------------
@contents
@comment ----------------------------------------------------------------

@comment This is REALLY the end :-)

@bye
