\input amigatexinfo
\input texinfo
@c %**start of header
@setfilename ToolManager.info
@settitle ToolManager 3.1 Documentation
@setchapternewpage on
@c %**end of header


@comment
@comment  english.tex V3.1
@comment
@comment  Documentation for ToolManager (Texinfo format, Language: english)
@comment
@comment  Copyright (C) 1990-1998 Stefan Becker
@comment


@comment ----------------------------------------------------------------
@comment --- Title & Copyright Page (for printed manual)              ---
@comment ----------------------------------------------------------------
@titlepage

@title{ToolManager}
@subtitle{An extension for the Amiga Workbench}
@subtitle{}
@subtitle{Version 3.1}
@subtitle{01 Jun 1998}

@page
@vskip 0pt plus 1filll

@comment ----------------------------------------------------------------
@comment Note to translators:
@comment Translate this page and then copy it to the node "Copyright"
@comment Both pages must be identical (except for the TeX commands)!
@comment ----------------------------------------------------------------

Permission is granted to make and distribute verbatim copies of this
manual provided the copyright notice and this permission notice are
preserved on all copies.

@ignore
Permission is granted to process this file by TeX and print the results,
provided the printed document carries a copying permission notice identical to
this one except for the removal of this paragraph (this paragraph not being
relevant to the printed manual).
@end ignore

@noindent COPYRIGHT

Copyright @copyright{} 1990--1998 Stefan Becker

No program, document, data file or source code from this software
package, neither in whole nor in part, may be included or used in other
software packages unless it is authorized by a written permission from
the author.

@noindent NO WARRANTY

There is no warranty for this software package. Although the author
has tried to prevent errors he can't guarantee that the software package
described in this document is 100% reliable. You are therefore using this
material at your own risk. The author cannot be made responsible for any
damage which is caused by using this software package.

@noindent DISTRIBUTION

This software package is freely distributable. It may be transfered to
any media which is used for the distribution of free software like Public
Domain disk collections, CDROMs, FTP servers or bulletin board systems.

In order to ensure the integrity of this software package distributors
should use the original archive files:

@tex
\settabs 2 \columns
\+{\tt ToolManager3\_1Binaries.lha}&  (file name on Aminet: {\tt ToolManagerBin.lha})\cr
\+{\tt ToolManager3\_1Developer.lha}& (file name on Aminet: {\tt ToolManagerDev.lha})\cr
\+{\tt ToolManager3\_1Extras.lha}&    (file name on Aminet: {\tt ToolManagerExt.lha})\cr
\+{\tt ToolManager3\_1Locale.lha}&    (file name on Aminet: {\tt ToolManagerLoc.lha})\cr
\+{\tt ToolManager3\_1Sources.lha}&   (file name on Aminet: {\tt ToolManagerSrc.lha})\cr
@end tex

The author cannot be made responsible if this software package has
become unusable due to modifications of the archive contents or of the
archive files itself.

There is no limit on the fee taken by distributors, e.g.@: for the media
costs of floppy disks, streamer tapes or compact discs, or the process of
duplicating. Such limits have proven to be harmful to the idea of freely
distributable software, e.g.@: the software package was removed instead of
reducing the price of a floppy disk below the limit.

Although the author does not impose any limit on these fees he would
like to express his personal opinions on this matter:

@itemize @bullet
@item This software package should be made available to everyone free of
charge whenever this is possible.

@item If you have purchased this software package under normal conditions
from a Public Domain dealer on a floppy disk and have paid more than
5 DM or US $5 then you have @emph{definitely} paid too much. Please don't
support this improper profit making any longer and switch to a
cheaper source as soon as possible.
@end itemize

@noindent USAGE RESTRICTIONS

No program, document, data file or source code from this software
package, neither in whole nor in part, may be used on any machine which
is used

@itemize @bullet
@item for the research, development, construction, testing or production
of weapons or other military applications. This also includes any
machine which is used in the education for any of the above
mentioned purposes.

@item by people who accept, support or use violence against other people,
e.g.@: citizens from foreign countries.
@end itemize
@page
(This page is intentionally left blank)
@end titlepage
@headings double
@comment ----------------------------------------------------------------




@comment ----------------------------------------------------------------
@comment --- Top Node (not printed)                                   ---
@comment ----------------------------------------------------------------
@ifinfo
@node Top, Copyright, (dir), (dir)
@top ToolManager 3.1 Documentation

@menu
Important information:

* Copyright::          Copyright and other legal stuff
* GiftWare::           If you like ToolManager@dots{}
* Future::             About the future of ToolManager
* Author::             Where to send bug reports, comments & donations

Usage:

* Requirements::       What is required to run ToolManager?
* Installation::       How to install ToolManager
* Concepts::           Concepts behind ToolManager
* Preferences::        How to configure ToolManager

Appendices:

* Hotkeys::            How to define a Hotkey
* Questions::          Frequently asked questions
* History::            History of ToolManager
* Credits::            The author would like to thank@dots{}
* MUI::                Information about MUI
* Index::              Index for this document
@end menu

@end ifinfo
@comment ----------------------------------------------------------------




@comment ----------------------------------------------------------------
@comment --- Chapter: Copyright (for Info/AmigaGuide document)        ---
@comment ----------------------------------------------------------------
@ifinfo
@node Copyright, GiftWare, Top, Top
@chapter Copyright and other legal stuff
@cindex Copyright
@cindex Distribution
@cindex Legal stuff
@cindex Permissions
@cindex Probibitions

Permission is granted to make and distribute verbatim copies of this
manual provided the copyright notice and this permission notice are
preserved on all copies.

@noindent COPYRIGHT

Copyright @copyright{} 1990--1998 Stefan Becker

No program, document, data file or source code from this software
package, neither in whole nor in part, may be included or used in other
software packages unless it is authorized by a written permission from
the author.

@noindent NO WARRANTY

There is no warranty for this software package. Although the author
has tried to prevent errors he can't guarantee that the software package
described in this document is 100% reliable. You are therefore using this
material at your own risk. The author cannot be made responsible for any
damage which is caused by using this software package.

@noindent DISTRIBUTION

This software package is freely distributable. It may be transfered to
any media which is used for the distribution of free software like Public
Domain disk collections, CDROMs, FTP servers or bulletin board systems.

In order to ensure the integrity of this software package distributors
should use the original archive files:

@noindent
ToolManager3_1Binaries.lha  (file name on Aminet: ToolManagerBin.lha)
ToolManager3_1Developer.lha (file name on Aminet: ToolManagerDev.lha)
ToolManager3_1Extras.lha    (file name on Aminet: ToolManagerExt.lha)
ToolManager3_1Locale.lha    (file name on Aminet: ToolManagerLoc.lha)
ToolManager3_1Sources.lha   (file name on Aminet: ToolManagerSrc.lha)

The author cannot be made responsible if this software package has
become unusable due to modifications of the archive contents or of the
archive files itself.

There is no limit on the fee taken by distributors, e.g.@: for the media
costs of floppy disks, streamer tapes or compact discs, or the process of
duplicating. Such limits have proven to be harmful to the idea of freely
distributable software, e.g.@: the software package was removed instead of
reducing the price of a floppy disk below the limit.

Although the author does not impose any limit on these fees he would
like to express his personal opinions on this matter:

@itemize @bullet
@item This software package should be made available to everyone free of
charge whenever this is possible.

@item If you have purchased this software package under normal conditions
from a Public Domain dealer on a floppy disk and have paid more than
5 DM or US $5 then you have @emph{definitely} paid too much. Please don't
support this improper profit making any longer and switch to a
cheaper source as soon as possible.
@end itemize

@noindent USAGE RESTRICTIONS

No program, document, data file or source code from this software
package, neither in whole nor in part, may be used on any machine which
is used

@itemize @bullet
@item for the research, development, construction, testing or production
of weapons or other military applications. This also includes any
machine which is used in the education for any of the above
mentioned purposes.

@item by people who accept, support or use violence against other people,
e.g.@: citizens from foreign countries.
@end itemize
@end ifinfo
@comment ----------------------------------------------------------------




@comment ----------------------------------------------------------------
@comment --- Chapter: GiftWare                                        ---
@comment ----------------------------------------------------------------
@node GiftWare, Future, Copyright, Top
@chapter If you like ToolManager@dots{}
@cindex GiftWare
@cindex Donations

ToolManager is @emph{GiftWare}, @strong{not} ShareWare!

So if you like the program and use it very often, you should consider to
send a little donation or gift to honor the work that I have put into
this program. I suggest a donation of US $10--$20 or 10--20 DM. Please
don't send cheques or money orders from outside Europe, because most
often cashing those items costs more than what they amount to.

If can't affort to send a donation you don't have to feel bad about it.
But you have to send me at least a postcard or letter saying that you are
using ToolManager (I like to get fan mail :-). @xref{Author}.

@comment ----------------------------------------------------------------




@comment ----------------------------------------------------------------
@comment --- Chapter: Future                                          ---
@comment ----------------------------------------------------------------
@node Future, Author, GiftWare, Top
@chapter About the future of ToolManager
@cindex Future

Since the last major release of ToolManager 2.1 in May 1993 it has been a
troubled time for the Amiga and its user community. At the time of this
writing the future still doesn't look too optimistic. Despite of this I
have decided to develop a new version of ToolManager 3.0, because of
the enormous feedback I got from hundreds of satisfied users.

My trustworthy A3000 is now getting old and with the current situation it
is uncertain which direction the Amiga will take. I can't afford to buy
every upgrade or any of the (eventually) new machines, because this
project is only my hobby. Thus it depends on your feedback and donations
if I'm able to work on future versions of ToolManager.

This is also a call to the companies which are working on the future Amigas
(Amiga Technologies, Phase5, ProDAD, VisCorp or whoever is currently involved).
I'm only a FD author and can't afford to buy every possible system or to pay
the developer material for every system. So I need your support if you wan't to
see ToolManager running on your system. Remember that ToolManager is one of (if
not @strong{the}) most-used tools on the Amiga and therefore it will be a bonus
for your system.

The future of ToolManager depends on @emph{YOUR} support!

@comment ----------------------------------------------------------------




@comment ----------------------------------------------------------------
@comment --- Chapter: Author                                          ---
@comment ----------------------------------------------------------------
@node Author, Requirements, Future, Top
@chapter Where to send bug reports, comments & donations
@cindex Addresses
@cindex Author
@cindex Contact addresses
@cindex EMail
@cindex Homepage
@cindex Post address
@cindex Word Wide Web

The author can be reached at the following addresses:

@table @asis
@item Post address:

@example

        Stefan Becker
        Bonner Ring 68
D-50374 Erfstadt
        GERMANY
@end example

@item Electronic Mail:

@example

stefanb@@yello.ping.de
@end example
@end table

There is also a ToolManager homepage available in the World Wide Web:

@example
http://www.ping.de/sites/yello/toolmanager.html
@end example

@comment ----------------------------------------------------------------




@comment ----------------------------------------------------------------
@comment --- Chapter: Requirements                                    ---
@comment ----------------------------------------------------------------
@node Requirements, Installation, Author, Top
@chapter What is required to run ToolManager?
@cindex Requirements
@cindex OS 3.0
@cindex V39
@cindex DataTypes
@cindex PictDT V43
@cindex Pophotkey
@cindex Popport
@cindex Popposition
@cindex DOSPath
@cindex WBStart
@cindex ScreenNotify

ToolManager needs at least:

@table @asis
@item AmigaOS 3.0 (V39)
or better for memory pools and the picture.datatype.

@item WBStart 2.2
This enables ToolManager to start Workbench programs.

@item DOSPath 1.0
This enables ToolManager to handle AmigaDOS paths.
@end table

Additionally it supports:

@table @asis
@item ScreenNotify 1.0
This package enables ToolManager to open and close its dock windows when
public screens open or close.

@item picture.datatype V43
The extensions of this enhanced picture.datatype are automatically supported if
it is installed on your system. @xref{Credits}.
@end table

The Preferences editor requires:

@table @asis
@item AmigaOS 3.0 (V39)
or better for memory pools.

@item MUI 3.7
The object-oriented GUI system. @xref{MUI}.

@item Pophotkey, Popport, Popposition
MUI custom classes for popups. @xref{Credits}.
@end table

@comment ----------------------------------------------------------------




@comment ----------------------------------------------------------------
@comment --- Chapter: Installation                                    ---
@comment ----------------------------------------------------------------
@node Installation, Concepts, Requirements, Top
@chapter How to install ToolManager
@cindex Installation
@cindex Installer V43.3

Please use the supplied Installer script to install ToolManager. It
requires the AmigaOS Installer V43.3. This version can be found
on the Aminet.

Make sure that you quit the old ToolManager first before installing the
new version!

@comment ----------------------------------------------------------------




@comment ----------------------------------------------------------------
@comment --- Chapter: Concepts                                        ---
@comment ----------------------------------------------------------------
@node Concepts, Preferences, Installation, Top
@chapter Concepts behind ToolManager
@cindex Concepts

ToolManager is a program which lets you start your tools in a very easy way.
You can start programs by using keyboard shortcuts (@pxref{Hotkeys}), by
selecting an entry from the Workbench's Tools menu or by clicking an icon
either on the Workbench or in special dock windows. You can even drag icons
from Workbench drawers on those icons to supply files to the programs.
Additionally you can attach a sound to each of these actions.

All these things are controlled by ToolManager objects. Each object has a
specific task and contains information how to accomplish this task. F.ex.
programs are represented as Exec objects and contain information about the
program name and the stack size.

There are two types of objects. The object types Exec, Image and Sound are
called basic objects, because they only contain information about one program,
one image and one sound.

The object types Menu, Icon and Dock are called compound objects, because they
bind several basic objects together to perform their tasks. F.ex. an icon on
the Workbench is represented by an Icon object which has a link to an Image
object for the icon imagery and a link to an Exec object which will start a
program when the icon is activated.

@menu
* Exec objects::    Programs
* Image objects::   Images
* Sound objects::   Sounds
* Menu objects::    Entries in the Workbench Tools menu
* Icon objects::    Icons in the Workbench window
* Dock objects::    Button windows
@end menu

@comment ----------------------------------------------------------------




@comment ----------------------------------------------------------------
@comment --- Section: Concepts/Exec objects                           ---
@comment ----------------------------------------------------------------
@node Exec objects, Image objects, , Concepts
@section Programs
@cindex Exec objects

An Exec object contains information about a program which is started when the
object is activated. The object can be activated directly by the user using a
@ref{Hotkeys, hotkey} or through a compund object. On activation a set of files
can be supplied which are forwarded to the program as startup parameters. Exec
objects are configured with the @ref{ExecWindow, Exec object edit window}.

ToolManager can execute different kinds of programs:

@table @asis
@item Shell
The program is executed as shell program just as the user had typed in the
command line into the shell. Shell scripts have to be started in this mode. You
can use the usual @file{[]} place holders to specify where the file arguments
should be placed on the command line.

@item Workbench
A start from the Workbench is simulated. All files are supplied as Workbench
arguments. Shell-only programs and shell scripts will not work when started as
Workbench program.

@item ARexx
An ARexx script or command is executed.

@item Dock
The command specifies a ToolManager Dock object which should be activated. You
can use this program type to create docks which are embedded in other docks.

@item Hotkey
A hotkey is generated. This might be used to control another program from
ToolManager.
@end table

@comment ----------------------------------------------------------------




@comment ----------------------------------------------------------------
@comment --- Section: Concepts/Image objects                          ---
@comment ----------------------------------------------------------------
@node Image objects, Sound objects, Exec objects, Concepts
@section Images
@cindex Image objects

An Image object contains information about an image which is used by a compund
object. ToolManager can load Workbench icons which are used by @ref{Icon
objects}. When a @ref{Dock objects, Dock object} uses an Image object the data
is loaded using the DataTypes system. Thus ToolManager can use every image for
which you have a valid datatype installed on your system. Image objects are
configured with the @ref{ImageWindow, Image object edit window}.

@comment ----------------------------------------------------------------




@comment ----------------------------------------------------------------
@comment --- Section: Concepts/Sound objects                          ---
@comment ----------------------------------------------------------------
@node Sound objects, Menu objects, Image objects, Concepts
@section Sounds
@cindex Sound objects

A Sound object contains information about a sound command. This sound command
is sent as an ARexx command to an external sound player. Sound objects are
configured with the @ref{SoundWindow, Sound object edit window}.

@comment ----------------------------------------------------------------




@comment ----------------------------------------------------------------
@comment --- Section: Concepts/Menu objects                           ---
@comment ----------------------------------------------------------------
@node Menu objects, Icon objects, Sound objects, Concepts
@section Entries in the Workbench Tools menu
@cindex Menu objects

A Menu object is a compound object which binds an @ref{Exec objects, Exec
object} and a @ref{Sound objects, Sound object} together to create an entry in
the Workbench Tools menu. Whenever this menu entry is selected the Exec object
and the Sound object are activated. Every selected icon on the Workbench is
used as startup parameter for the program. Menu objects are configured with the
@ref{MenuWindow, Menu object edit window}.

@comment ----------------------------------------------------------------




@comment ----------------------------------------------------------------
@comment --- Section: Concepts/Icon objects                           ---
@comment ----------------------------------------------------------------
@node Icon objects, Dock objects, Menu objects, Concepts
@section Icons in the Workbench window
@cindex Icon objects

An Icon object is a compound object which binds an @ref{Exec objects, Exec
object}, an @ref{Image objects, Image object} and a @ref{Sound objects, Sound
object} together to create an icon in the Workbench window. The Image object is
used to create the icon image. Icon objects are configured with the
@ref{IconWindow, Icon object edit window}.

Icons can be activated in two ways. The user can double-click the icon or he
can select icons on the Workbench and drop them on the icon. Whenever the icon
is activated the Exec object and the Sound object are activated. The icons
which have been dropped onto the icon are used as startup parameter for the
program.

@comment ----------------------------------------------------------------




@comment ----------------------------------------------------------------
@comment --- Section: Concepts/Dock objects                           ---
@comment ----------------------------------------------------------------
@node Dock objects, , Icon objects, Concepts
@section Button windows
@cindex Dock objects

A Dock object is a compound object which presents a window with button rows to
the user. Each button binds an @ref{Exec objects, Exec object}, an @ref{Image
objects, Image object} and a @ref{Sound objects, Sound object} together. Each
button can display either a text, an image or both. The name of the Exec object
is used for the text. The Image object is used to for the image. Dock objects
are configured with the @ref{DockWindow, Dock object edit window}.

Each button can be activated in two ways. The user can click on the button or
he can select icons on the Workbench and drop them on the button. Whenever the
icon is activated the Exec object and the Sound object are activated. The
selected icons are used as startup parameter for the program.

A Dock object can be ``activated'' with a @ref{Hotkeys, hotkey}. When the dock
window is closed and the user enters the hotkey then the dock window is opened
and vice versa.

If the @file{screennotify.library} is installed then ToolManager can open and
close dock windows automatically. Every time a screen is going to be closed all
dock windows on this screen are closed first. When the public screen opens
again all dock windows for this screen are opened again.

@comment ----------------------------------------------------------------




@comment ----------------------------------------------------------------
@comment --- Chapter: Preferences                                     ---
@comment ----------------------------------------------------------------
@node Preferences, Hotkeys, Concepts, Top
@chapter How to configure ToolManager
@cindex Preferences

The ToolManager preferences editor is used to configure ToolManager.

@menu
* MainWindow::         The main window

Object edit windows:

* ExecWindow::         How to configure Exec objects
* ImageWindow::        How to configure Image objects
* SoundWindow::        How to configure Sound objects
* MenuWindow::         How to configure Menu objects
* IconWindow::         How to configure Icon objects
* DockWindow::         How to configure Dock objects

Miscellaneous windows:

* GroupWindow::        How to rename an object group
* ClipWindow::         Clipboard for objects
* GlobalWindow::       Global ToolManager options

@end menu

The preferences editor understands the standard Workbench tool types and Shell
command parameters:

@table @asis
@item @code{FROM} (Shell only)
Specifies the file name from which the editor should load the configuration.

@item @code{EDIT} (default)
Edit the configuration.

@item @code{USE}
Use the specified configuration temporarily.

@item @code{SAVE}
Use the specified configuration permanently.

@item @code{CREATEICONS}
Create icons for the configuration files when they are saved. When started from
Workbench the preferences editor creates icons by default. When started from
the shell no icons are created by default.
@end table

@comment ----------------------------------------------------------------




@comment ----------------------------------------------------------------
@comment --- Section: Preferences/MainWindow                          ---
@comment ----------------------------------------------------------------
@node MainWindow, ExecWindow, , Preferences
@section The main window

The main window contains the object lists. By clicking on the object type you
can select which list is currently visible. Each list can contain several
groups. Each group can hold several objects.

A double-click on the name of a group opens the @ref{GroupWindow, group edit
window}. Clicking on the symbol left to the group name opens and closes the
group. If a group is open you can see all the objects in this group. A
double-click on the name of an object opens the object edit window.

To move a group you first select one, drag it to the new position
while holding the left mouse button and then release the mouse button.
You can also move objects between groups with this method.

Attached to the list are four buttons:

@table @asis
@item New Group
Creates a new empty group. The @ref{GroupWindow, group edit window} will open
so that you can set the name of the new group.

@item New Object
Create a new object in the selected group. The object edit window will open so
that you can edit the properties of the new object.

@item Delete
Deletes the selected group or object. If a group is selected also all objects
in this group will be deleted.

@item Sort
If an object or an open group is selected then the contents of this group are
sorted alphabetically. Otherwise the groups are sorted alphabetically.
@end table

With the buttons at the bottom of the main window you can tell the preferences
editor where to store the configuration. The name of the configuration file is
@file{ToolManager.prefs}.

@table @asis
@item Save
Store the configuration to @file{ENVARC:} and @file{ENV:}. The new
configuration will be taken into use automatically and survive a reboot. After
storing the preferences editor exits.

@item Use
Store the configuration to @file{ENV:}. The new configuration will be taken
into use automatically but it will @emph{not} survive a reboot. After storing
the preferences editor exits.

@item Test
Store the configuration to @file{ENV:}. The new configuration will be taken
into use automatically but it will @emph{not} survive a reboot. The preferences
editor does not exit.

@item Cancel
The preferences editor exits. All changes which have not been saved will be
discarded.
@end table

@comment ----------------------------------------------------------------




@comment ----------------------------------------------------------------
@comment --- Section: Preferences/ExecWindow                          ---
@comment ----------------------------------------------------------------
@node ExecWindow, ImageWindow, MainWindow, Preferences
@section How to configure Exec objects

@ref{Exec objects} contain information about programs. The edit window has the
following gadgets:

@table @asis
@item Name
Name of the object.

@item Exec Type
Type of the program. You can choose between Shell, Workbench, ARexx, Dock,
Hotkey and Network. The type Network is currently not supported.

@item Command
The name of the program. This is either the file name, the name of a dock
objects or a hotkey description depending on the Exec Type.

@item Hotkey
A @ref{Hotkeys, hotkey} desciption string which activates this Exec object.

@item Stack
The stack size for the program. ToolManager will enforce a minimum size of 4096
bytes.

@item Priority
The priority for the program. Usually you should only use the default value 0.

@item Arguments
If this gadget is selected then files are forwarded to the program as startup
arguments. Otherwise the files are ignored.

@item To Front
If this gadget is selected then the specified public screen is moved to front
before starting the program.

@item Current Directory
The program is started from this directory.

@item Path
You can here supply a list of directories (separated with semicolons) which is
used by shell programs to search for other programs.

@item Output File
The output of shell programs is redirected to this file. If you specify a
console window here then the ouput and the input of the shell program is
redirected to this window.

@item Public Screen
Specifies the public screen which will be moved to the front before starting
the programm.
@end table

@comment ----------------------------------------------------------------




@comment ----------------------------------------------------------------
@comment --- Section: Preferences/ImageWindow                         ---
@comment ----------------------------------------------------------------
@node ImageWindow, SoundWindow, ExecWindow, Preferences
@section How to configure Image objects

@ref{Image objects} contain information about images. The edit window has the
following gadgets:

@table @asis
@item Name
Name of the object.

@item File
The name of the file from which the image data should be loaded. Usually
you have to remove the ending @file{.info} if you want to load an icon file.
@end table

@comment ----------------------------------------------------------------




@comment ----------------------------------------------------------------
@comment --- Section: Preferences/SoundWindow                         ---
@comment ----------------------------------------------------------------
@node SoundWindow, MenuWindow, ImageWindow, Preferences
@section How to configure Sound objects

@ref{Sound objects} contain information about sounds. The edit window has the
following gadgets:

@table @asis
@item Name
Name of the object.

@item Command
The ARexx command which should be send to the external sound player.

@item ARexx Port
The ARexx port name of the external sound player. The default is @code{PLAY}
which is used by the program @file{upd}.
@end table

@comment ----------------------------------------------------------------




@comment ----------------------------------------------------------------
@comment --- Section: Preferences/MenuWindow                          ---
@comment ----------------------------------------------------------------
@node MenuWindow, IconWindow, SoundWindow, Preferences
@section How to configure Menu objects

@ref{Menu objects} contain information about entries in the Workbench Tools
menu. The edit window has the following gadgets:

@table @asis
@item Name
Name of the object. This is also used to create the menu entry.

@item Exec Object
Link to the attached @ref{Exec objects, Exec object}. Use Drag&Drop from the
@ref{MainWindow, main window} or a @ref{ClipWindow, clipboard} to attach an
object. You can edit the attached object by clicking on it.

@item Sound Object
Link to the attached @ref{Sound objects, Sound object}. Use Drag&Drop from the
@ref{MainWindow, main window} or a @ref{ClipWindow, clipboard} to attach an
object. You can edit the attached object by clicking on it.
@end table

@comment ----------------------------------------------------------------




@comment ----------------------------------------------------------------
@comment --- Section: Preferences/IconWindow                          ---
@comment ----------------------------------------------------------------
@node IconWindow, DockWindow, MenuWindow, Preferences
@section How to configure Icon objects

@ref{Icon objects} contain information about icons in the Workbench window.
The edit window has the following gadgets:

@table @asis
@item Name
Name of the object.

@item Exec Object
Link to the attached @ref{Exec objects, Exec object}. Use Drag&Drop from the
@ref{MainWindow, main window} or a @ref{ClipWindow, clipboard} to attach an
object. You can edit the attached object by clicking on it.

@item Image Object
Link to the attached @ref{Image objects, Image object}. Use Drag&Drop from the
@ref{MainWindow, main window} or a @ref{ClipWindow, clipboard} to attach an
object. You can edit the attached object by clicking on it.

@item Sound Object
Link to the attached @ref{Sound objects, Sound object}. Use Drag&Drop from the
@ref{MainWindow, main window} or a @ref{ClipWindow, clipboard} to attach an
object. You can edit the attached object by clicking on it.

@item Position
Specifies the X and Y coordinates of the icon, e.g.@: for @code{X = 100} and
@code{Y = 55} you would enter @code{100/55}. The coordinates are relative to
the top left corner of the Workbench window.

@item Show Name
If this gadget is selected then the name of the object is shown below the icon.
@end table

@comment ----------------------------------------------------------------




@comment ----------------------------------------------------------------
@comment --- Section: Preferences/DockWindow                          ---
@comment ----------------------------------------------------------------
@node DockWindow, GroupWindow, IconWindow, Preferences
@section How to configure Dock objects

@ref{Dock objects} contain information about dock windows. The edit window has
the following gadgets:

@table @asis
@item Name
Name of the object. This also used as window title.

@item Hotkey
A @ref{Hotkeys, hotkey} desciption string which open and closes the dock
window.

@item Public Screen
Specifies the public screen on which the dock window appears. You @strong{must}
specify a valid public screen name if you want to use the automatic open and
close feature for dock windows.

@item Font
Use this font for the button texts.

@item Columns
Number of button columns in the dock window. Columns will be filled with
buttons from left to right. If the last column in a row is filled then a new
button row is added. All buttons will have the same width and height.

@item Position
Specifies the X and Y coordinates of the dock window, e.g.@: for @code{X = 150}
and @code{Y = 200} you would enter @code{150/200}. The coordinates are relative
to the top left corner of the screen.

@item Entries
Each entry in this list creates one button. The left column contains the link
to the attached @ref{Exec objects, Exec object}, the middle column the link to
the attached @ref{Image objects, image object} and the right column the link to
the attached @ref{Sound objects, sound object}. Use Drag&Drop from the
@ref{MainWindow, main window} or a @ref{ClipWindow, clipboard} to attach an
object. You can edit the attached objects by double-clicking on them. You can
use Drag&Drop to sort the entries in the list. When you press the Delete button
the currently selected entry is removed from the list. The attached objects
itself are @emph{not} deleted.

@item Activated
The dock window will be opened when the configuration is loaded.

@item Backdrop
The dock window is moved to the back after it has been opened.

@item Border
When this gadget is selected the dock window looks like a normal window with a
border and window gadgets. Otherwise it will have no border at all. Note that
you can move the dock window only if it has a border.

@item Menu
A menu is attached to the dock window. The menu allows you to close the dock
window, start the ToolManager @ref{Preferences, preferences editor} or to quit
ToolManager.

@item Frontmost
When this gadget is selected then the dock window will always open on the
frontmost public screen.

@item Pop Up
The dock window closes automatically after a button has been selected.

@item Centered
The dock window opens centered around the current mouse position.

@item Sticky
Usually a dock window remembers its position when you close it. It will open
on this position if you open it again. If this gadget is selected then the dock
window will always open at the same position.

@item Images
The buttons in the dock window will display images. Note that you have to
attach @ref{Image objects} to the dock entries in this case.

@item Text
The buttons in the dock window will display the name of the attached @ref{Exec
objects, Exec object}. Note that you have to attach Exec objects to the dock
entries in this case.
@end table

@comment ----------------------------------------------------------------




@comment ----------------------------------------------------------------
@comment --- Section: Preferences/GroupWindow                         ---
@comment ----------------------------------------------------------------
@node GroupWindow, ClipWindow, DockWindow, Preferences
@section How to rename an object group

You can change the name of the group with the name string gadget.

@comment ----------------------------------------------------------------




@comment ----------------------------------------------------------------
@comment --- Section: Preferences/ClipWindow                          ---
@comment ----------------------------------------------------------------
@node ClipWindow, GlobalWindow, GroupWindow, Preferences
@section Clipboard for objects

This window contains a list with attached @ref{Exec objects}, @ref{Image
objects} and @ref{Sound objects}. You can drag objects from this list and drop
them on edit windows. You can edit the attached objects by double-clicking
them. When you press the Delete button the currently selected object is removed
from the list. The object itself is @emph{not} deleted.

The clipboard can be opened from the menu in the main window. You can have
several clipboards open at once.

@comment ----------------------------------------------------------------




@comment ----------------------------------------------------------------
@comment --- Section: Preferences/GlobalWindow                        ---
@comment ----------------------------------------------------------------
@node GlobalWindow, , ClipWindow, Preferences
@section Global ToolManager options

This window lets you change the global options of ToolManager. It can be opened
from the menu in the main window. It has the following gadgets:

@table @asis
@item Current Directory
Set the current directory for the ToolManager process. All files without an
absolute path name will be opened relative to this directory. The default
directory is the boot volume.

@item Preferences editor
Path of the ToolManager @ref{Preferences, preferences editor} binary. The
default is @file{SYS:Prefs/ToolManager}.

@item Enable Network
Currently not supported.

@item Enable Remap
Enable colour remapping for picture.datatype. Disable this only if you the
images in the dock windows come up with the wrong colours.

@item Remap Precision
Sets the precision of the colour remapping. You might try change this
value if the colour choices of the remap algorithm are unsatisfactory on
your system.
@end table

@comment ----------------------------------------------------------------




@comment ----------------------------------------------------------------
@comment --- Appendix: Hotkeys                                        ---
@comment ----------------------------------------------------------------
@node Hotkeys, Questions, Preferences, Top
@appendix How to define a Hotkey
@cindex Introduction to Hotkeys
@cindex Hotkeys

This chapter describes how to define a Hotkey as an Input Description String,
which is then parsed by Commodities. Each time a Hotkey is activated
Commodities generates an event which is used by ToolManager to activate Exec
objects or to toggle Dock objects. A description string has the following
syntax:

@example
[<class>] @{[-][<qualifier>]@} [-][upstroke] [<key code>]
@end example

All keywords are case insensitive.

@code{class} describes the InputEvent class. This parameter is optional and
if it is missing the default @code{rawkey} is used. @xref{InputEvent classes}.

Qualifiers are ``signals'' that must be set or cleared by the time the Hotkey
is activated; otherwise no event will be generated. For each qualifier that
must be set you supply its keyword. All other qualifiers are expected to be
cleared by default. If you want to ignore a qualifier, just set a @code{-}
before its keyword. @xref{Qualifiers}.

A Hotkey event is usually generated when a key is pressed. If the event
should be generated when the key is released, supply the keyword
@code{upstroke}. When both press and release of the key should generate an
event, use @code{-upstroke}.

The key code is depending on the InputEvent class. @xref{Key codes}.

@menu
* InputEvent classes::
* Qualifiers::
* Key codes::
* Hotkey examples::
@end menu

Note: Choose your Hotkeys @emph{carefully}, because Commodities has a high
priority in the InputEvent handler chain, i.e.@: it will override existing
definitions.

@comment ----------------------------------------------------------------




@comment ----------------------------------------------------------------
@comment --- Appendix Section: Hotkeys/InputEvent classes             ---
@comment ----------------------------------------------------------------
@node InputEvent classes, Qualifiers, , Hotkeys
@appendixsection InputEvent classes
@cindex Diskinserted
@cindex Diskremoved
@cindex InputEvent classes
@cindex Rawkey
@cindex Rawmouse

Commodities supports most of the InputEvent classes that are generated by the
input.device. This section describes those classes that are most useful for
ToolManager Hotkeys.

@table @code
@item rawkey
This is the default class and covers all keyboard events. For example
@code{rawkey a} or @code{a} creates an event every time when the key ``a'' is
pressed. You must specify a key code for this class.
@xref{rawkey key codes,rawkey}.

@item rawmouse
This class describes all mouse button events. You must specify a key code for
this class. @xref{rawmouse key codes,rawmouse}.

@item diskinserted
Events of this class are generated when a disk is inserted in a drive. This
class has no key codes.

@item diskremoved
Events of this class are generated when a disk is removed from a drive. This
class has no key codes.
@end table

@comment ----------------------------------------------------------------




@comment ----------------------------------------------------------------
@comment --- Appendix Section: Hotkeys/Qualifiers                     ---
@comment ----------------------------------------------------------------
@node Qualifiers, Key codes, InputEvent classes, Hotkeys
@appendixsection Qualifiers
@cindex Qualifiers

Commodities supports the following qualifiers:

@table @asis
@item @code{lshift}, @code{left_shift}
Left shift key

@item @code{rshift}, @code{right_shift}
Right shift key

@item @code{shift}
Either shift key

@item @code{capslock}, @code{caps_lock}
Caps lock key

@item @code{caps}
Either shift key or caps lock key

@item @code{control}, @code{ctrl}
Control key

@item @code{lalt}, @code{left_alt}
Left alt key

@item @code{ralt}, @code{right_alt}
Right alt key

@item @code{alt}
Either alt key

@item @code{lcommand}, @code{lamiga}, @code{left_amiga}, @code{left_command}
Left Amiga/Command key

@item @code{rcommand}, @code{ramiga}, @code{right_amiga}, @code{right_command}
Right Amiga/Command key

@item @code{numericpad}, @code{numpad}, @code{num_pad}, @code{numeric_pad}
This keyword @emph{must} be used for any key on the numeric pad.

@item @code{leftbutton}, @code{lbutton}, @code{left_button}
Left mouse button

@item @code{midbutton}, @code{mbutton}, @code{middlebutton}, @code{middle_button}
Middle mouse button

@item @code{rbutton}, @code{rightbutton}, @code{right_button}
Right mouse button

@item @code{repeat}
This qualifier is set when the keyboard repeat is active. This is only
useful for the InputEvent class @code{rawkey}.
@end table

@comment ----------------------------------------------------------------




@comment ----------------------------------------------------------------
@comment --- Appendix Section: Hotkeys/Key codes                      ---
@comment ----------------------------------------------------------------
@node Key codes, Hotkey examples, Qualifiers, Hotkeys
@appendixsection Key codes

Each InputEvent class has its own key codes:

@menu
* rawkey:rawkey key codes.
* rawmouse:rawmouse key codes.
@end menu

@comment ----------------------------------------------------------------




@comment ----------------------------------------------------------------
@comment --- Appendix Subsection: Hotkeys/Key codes/rawkey key codes  ---
@comment ----------------------------------------------------------------
@node rawkey key codes, rawmouse key codes, , Key codes
@appendixsubsec Key codes for InputEvent class @code{rawkey}
@cindex Key codes for @code{rawkey}

@table @asis
@item @code{a}-@code{z}, @code{0}-@code{9}, @dots{}
ASCII characters

@item @code{f1}, @code{f2}, @dots{}, @code{f10}, @code{f11}, @code{f12}
Function keys

@item  @code{up}, @code{cursor_up}, @code{down}, @code{cursor_down}
@itemx @code{left}, @code{cursor_left}, @code{right}, @code{cursor_right}
Cursor keys

@item  @code{esc}, @code{escape}, @code{backspace}, @code{del}, @code{help}
@itemx @code{tab}, @code{comma}, @code{return}, @code{space}, @code{spacebar}
Special keys

@item  @code{enter}, @code{insert}, @code{delete}
@itemx @code{page_up}, @code{page_down}, @code{home}, @code{end}
Numeric Pad keys. Each of these key codes @emph{must} be used with the
@code{numericpad} qualifier keyword!
@end table

@comment ----------------------------------------------------------------




@comment ----------------------------------------------------------------
@comment --- Appendix Subsection: Hotkeys/Key codes/rawmouse key codes --
@comment ----------------------------------------------------------------
@node rawmouse key codes, , rawkey key codes, Key codes
@appendixsubsec Key codes for InputEvent class @code{rawmouse}
@cindex Key codes for @code{rawmouse}

@table @code
@item mouse_leftpress
Press left mouse button

@item mouse_middlepress
Press middle mouse button

@item mouse_rightpress
Press right mouse button
@end table

Note: To use one of these key codes, you must also set the corresponding
qualifier keyword, e.g.@:

@example
rawmouse leftbutton mouse_leftpress
@end example

@comment ----------------------------------------------------------------




@comment ----------------------------------------------------------------
@comment --- Appendix Section: Hotkeys/Hotkey examples                ---
@comment ----------------------------------------------------------------
@node Hotkey examples, , Key codes, Hotkeys
@appendixsection Examples for Hotkeys
@cindex Examples for Hotkeys

@table @code
@item ralt t
Hold right Alt key and press ``t''

@item ralt lalt t
Hold left @emph{and} right Alt key and press ``t''

@item alt t
Hold either Alt key and press ``t''

@item rcommand f2
Hold right Amiga key and press the second function key

@item numericpad enter
Press the Enter key on the numeric pad

@item rawmouse midbutton leftbutton mouse_leftpress
Hold middle mouse button and press the the left mouse button

@item diskinserted
Insert a disk in any drive
@end table

@comment ----------------------------------------------------------------




@comment ----------------------------------------------------------------
@comment --- Appendix: Questions                                      ---
@comment ----------------------------------------------------------------
@node Questions, History, Hotkeys, Top
@appendix Frequently asked questions

Here are the answers to the most asked questions about ToolManager:

@itemize @minus
@item When I start the ToolManager preferences editor only a requester
with the text ``Program initialization failed'' appears. What's wrong?

The preferences editor checks the basic requirements before opening the
first window. Please check that these @ref{Requirements, requirements}
are fulfilled! Your system might also be running out of memory. You may
have to quit some other applications first before enough memory is
available to run the ToolManager preferences editor. Also there can be
only one preferences editor running at one time.

@item How can I run the ToolManager preferences editor on another public
screen than the Workbench screen?

Select the entry @code{MUI...} from the settings menu in the
@ref{MainWindow, main window} menu. Now select the @code{System} settings
page and enter the name of the public screen into the string gadget. For
further details please consult the @ref{MUI} documentation.

@item After converting my old ToolManager 2.x configuration some pictures
are missing in the dock windows and some dock windows don't appear at
all!

ToolManager 3.1 only supports picture files in dock windows which can be
accessed via the picture.datatype. In your old configuration you have
attached some Image objects to the Dock objects which refer to icon
files. In order to use icon files you have to install an icon datatype on
your system. You may find such a datatype e.g.@: on the Aminet. You can
also convert them to e.g. IFF Brushes.

@item After converting my old ToolManager 2.x configuration some of the
icons in the Workbench window are missing!

ToolManager 3.1 only supports icons files for Icon objects. In your old
configuration you have attached some Image objects to the Icon objects
which refer to IFF Brushes. You have to convert them to icon files in
order to use them with ToolManager 3.1 Icon objects.

@item When I use icon files for images in the dock windows then there are
some addtional texts attached to the image or it has a thick border.
What's wrong?

You have installed an icon.datatype on your system which inserts
additional information from the icon into the image. Please check the
documentation of the icon.datatype how this information can be
suppressed. If you don't like the thick border you also have to tell the
icon.datatype not to generate the usual icon borders.

@item Why can't ToolManager create multiple ``Tools'' menus or sub-menus?

Multiple menus or sub-menus are currently not supported by the Workbench.
To create them, you have to @emph{hack} them into the AmigaOS, which can
result in an unstable system. Therefore I won't implement it in ToolManager.


@item How can I create a horizontal dock window?

Just set the number of columns to the number of entries in the dock object.


@item How can I create an output window for shell programs?

Output windows can be created by using the @code{CON:} device. Use the
following file name to create an auto-open window with a close gadget which
doesn't close after the program has quit:

@example
CON:10/10/640/100/Output-Window/AUTO/CLOSE/WAIT
@end example

The @code{CON:} device has many options, please consult your AmigaDOS manual
for further information.


@item How can I put the arguments in the middle of a Shell/Arexx command line?

Usually all arguments are appended to the command line. To insert the
arguments anywhere in the command line, ToolManager uses the same @code{[]}
syntax, which is used by the AmigaShell command @code{alias}. So for example

@example
Dir [] all
@end example

will insert all arguments before the keyword @code{all}.


@item How can I create sub-docks?

You must use Exec objects of the type Dock. Put such objects in the entries of
your main dock and they will open/close the other docks.


@item The dock windows disappear when the Workbench screen is closed and opened
again.

You have forgotten to set the public screen name for the dock window to
@code{Workbench}. ToolManager will close dock windows when the public screen
closes. But it has to know on which public screen the dock windows should
appear in order to open them automatically when the public screen opens again.
@end itemize

@comment ----------------------------------------------------------------




@comment ----------------------------------------------------------------
@comment --- Appendix: History                                        ---
@comment ----------------------------------------------------------------
@node History, Credits, Questions, Top
@appendix History of ToolManager

@comment ----------------------------------------------------------------
@comment Note to translators:
@comment This chapter doesn't need to be translated
@comment ----------------------------------------------------------------

@table @asis
@item 3.1, Release date 01.06.1998
@itemize @minus
@item Clipboard list also shows the object type now
@item Only requires locale.libary V38 now
@item Icon drop corrected for dock windows with borders
@item The __geta4 qualifier was missing for some BOOPSI dispatchers. This might
correct some strange bugs which were reported?
@item Installer script now checks for required libraries
@end itemize

@item 3.0, Release date 23.02.1997
@itemize @minus
@item Again rewritten (almost) from scratch :-)
@item Old object system removed, TM objects are now BOOPSI objects
@item Now uses memory pools
@item Delay parameter removed from Exec Objects
@item Animation support removed from Image Objects
@item Picture.datatype V43 support added to Image Objects
@item Only icon images supported for Icon Objects
@item Only images loadable via picture.datatype are supported in Dock Objects
@item Pattern & Vertical flags and Title parameter removed from Dock Objects
@item Dock Objects can now display text and images
@item Dock Objects can now be completely borderless
@item New preferences file format, hopefully more flexible
@item Converter for the ToolManager 2.x format
@item Events are now checked while the configuration is read
@item Preferences is now a MUI application: resizable window, multiple open
edit windows and Drag&Drop support
@item Changing an object name automatically updates all references to the
object.
@item Support for grouping objects.
@item All dock objects get the screen notifications
@item Added support for DOSPath 1.0
@item CLI command lines are not limited to 4KB anymore
@item Installer script
@end itemize


@item 2.1b, Release date 13.03.1996
@itemize @minus
@item Minor update to 2.1
@item Added support for WBStart 2.0
@end itemize


@item 2.1a, Release date 26.03.1995
@itemize @minus
@item Minor update to 2.1
@item Added support for ScreenNotify 1.0
@item Included newer version of WBStart-Handler
@item Included missing AutoDocs for toolmanager.library
@end itemize


@item 2.1, Release date 16.05.1993, Fish Disks #872 & #873
@itemize @minus
@item New Exec object types: Dock, Hot Key, Network
@item New Dock object flags: Backdrop, Sticky
@item New object type: Access
@item Network support
@item Editor main window is now an AppWindow
@item Gadget keyboard shortcuts in the preferences editor
@item New tooltypes for the preferences editor
@item Several bug fixes
@item Enhanced documentation
@end itemize


@item 2.0, Release date 26.09.1992, Fish Disk #752
@itemize @minus
@item Complete new concept (object oriented)
@item (Almost) Complete rewrite
@item ToolManager is now split up into two parts
@item Main handler is now embedded into a shared library
@item Configuration is now handled by a Preferences program
@item Configuration file format has changed again :-) It is an IFF File now
and resides in ENV:
@item Multiple Docks and multi-column Docks
@item Docks with new window design
@item Dock automatically detects largest image size
@item Sound support
@item Direct ARexx support for Exec objects
@item ToolManager can be used without the Workbench. If the Workbench isn't
running, it won't use any App* features.
@item Locale support
@item Path from Workbench will be used for CLI tools
@item Seperate Handler Task for starting WB processes
@end itemize


@item 1.5, Release date 10.10.1991, Fish Disk #551
@itemize @minus
@item Status Window: New/Open/Append/Save As menu items for config file
@item Edit Window: File requesters for file string gadgets
@item Added a Dock Window (a la NeXT)
@item Added a DeleteTool
@item A list of all active HotKeys can be shown
@item Tools can be moved around in the list
@item Icon positioning in the edit window added
@item Name of the program icon can be set
@item CLI tools can have an output file and a path list
@item Uses UserShell for CLI tools
@item Maximum command line length for CLI tools is now 4096 Bytes
@item AppIcons without a name are supported now
@item Workbench screen will be moved to front if you pop up the Status window
@item Workbench screen can be moved to front before starting a tool via HotKey
@item TM will wait up to 20 seconds for the workbench.library
@item Added a DELAY switch which causes TM to wait <num> seconds before adding
any App* stuff
@item renamed some tooltypes/parameters
@item some visual cues added
@item some internal changes
@end itemize


@item 1.4, Release date 09.07.1991, Fish Disk #527
@itemize @minus
@item Keyboard short cuts for tools
@item AppIcons for tools
@item Menu item can be switched off
@item Configuration file format completely changed (hopefully the last time)
@item CLI commandline parsing is now done by ReadArgs()
@item Status & edit window updated to new features
@item Safety check before program shutdown added
@item Menu item ``Open TM Window'' only appears if the program icon is
disabled
@item WB startup method changed. Now supports project icons
@item several internal changes
@end itemize


@item 1.3, Release date 13.03.1991, Fish Disk #476
@itemize @minus
@item Now supports different configuration files
@item Format of the configuration file slightly changed
@item Tool definitions can be changed at runtime
@item Now supports CLI & Workbench startup method
@item Selected icons are passed as parameters to the tools
@item Now uses the startup icon as program icon if started from Workbench
@item The position of the icon can now be supplied in the configuration file
@item The program icon can now be disabled
@item New menu entry ``Show TM Window''
@item Every new started ToolManager passes its startup parameters to the
already running ToolManager process
@end itemize


@item 1.2, Release date 12.01.1991, Fish Disk #442
@itemize @minus
@item Status window changed to a no-GZZ & simple refresh type
(this should save some bytes)
@item Status window remembers its last position
@item New status window gadget ``Save Configuration'': saves the actual tool
list in the configuration file
@item Small bugs removed in the ListView gadget handling
@item Name of the icon hard-wired to ``ToolManager''
@end itemize


@item 1.1, Release date 01.01.1991
@itemize @minus
@item Icons can be dropped on the status window
@item Status window contains a list of all tool names
@item Tools can be removed from the list
@end itemize


@item 1.0, Release date 04.11.1990
@itemize @minus
@item Initial release
@end itemize

@end table

@comment ----------------------------------------------------------------




@comment ----------------------------------------------------------------
@comment --- Appendix: Credits                                        ---
@comment ----------------------------------------------------------------
@node Credits, MUI, History, Top
@appendix The author would like to thank@dots{}
@cindex Credits
@cindex Thanks

ToolManager has gone through many major evolutionary phases since its first
implementation in mid-1990. This development would have been impossible if I
hadn't received the enormous feedback from various ToolManager users. Many
ideas & features resulted from this source@dots{}

Therefore I would like to thank:

@table @asis
@item For Alpha/Beta testing, ideas & bug reports:
Osma Ahvenlampi, Stephane Barbaray, Olaf Barthel, Fionn Behrens,
Mario Cattaneo, Michael van Elst, Michael Hohmann, Markus Illenseer,
Frank Mariak, Klaus Melchior, Bernhard Moellemann, Matthias Scheler,
Ralph Schmidt, Tobias Walter.

@item For the translations:
The Amiga Translators Organization (ATO). Please check the file
@file{Readme.Locale} for the list of translators.

@item Matthew Dillon
Without your @strong{excellent} C development system DICE and various other
tools, ToolManager wouldn't exist!

@item All users who sent me money:
Your support made this release possible.

@item All users who sent me a note:
I really enjoyed reading your letters and E-Mails!
@end table

ToolManager uses the following packages:

@table @asis
@item picture.datatype V43
Copyright @copyright{} 1995--1996 Ralph Schmidt, Frank Mariak &@*
Matthias Scheler

@item WBStart 2.2
Copyright @copyright{} 1991--1996 Stefan Becker

@item ScreenNotify 1.0
Copyright @copyright{} 1995 Stefan Becker

@item DOSPath 1.0
Copyright @copyright{} 1996 Stefan Becker

@item MUI
Copyright @copyright{} 1993--1997 Stefan Stuntz@*
World Wide Web home page: @code{http://www.sasg.com/}.

@itemx Pophotkey.mcc, Popport.mcc, Popposition.mcc
Copyright @copyright{} 1996--1997 Klaus Melchior

@item Icons
Copyright @copyright{} 1995 Michael W. Hohmann
@end table

@comment ----------------------------------------------------------------




@comment ----------------------------------------------------------------
@comment --- Appendix: MUI                                            ---
@comment ----------------------------------------------------------------
@node MUI, Index, Credits, Top
@appendix Information about MUI
@cindex MUI

@example
                          This application uses


                        MUI - MagicUserInterface

                (c) Copyright 1993-97 by Stefan Stuntz


MUI is a system to generate and maintain graphical user interfaces. With
the  aid  of  a  preferences program, the user of an application has the
ability to customize the outfit according to his personal taste.

MUI is distributed as shareware. To obtain a complete package containing
lots of examples and more information about registration please look for
a  file  called  "muiXXusr.lha"  (XX means the latest version number) on
your local bulletin boards or on public domain disks.

          If you want to register directly, feel free to send


                         DM 30.-  or  US$ 20.-

                                  to

                             Stefan Stuntz
                        Eduard-Spranger-Stra�e 7
                             80935 M�nchen
                                GERMANY



             Support and online registration is available at

                          http://www.sasg.com/
@end example

@comment ----------------------------------------------------------------




@comment ----------------------------------------------------------------
@comment --- Unnumbered chapter: Index                                ---
@comment ----------------------------------------------------------------
@node Index, , MUI, Top
@unnumbered Index

@printindex cp

@comment ----------------------------------------------------------------




@comment ----------------------------------------------------------------
@comment --- Table of Contents (appears only in printed manual)       ---
@comment ----------------------------------------------------------------
@contents
@comment ----------------------------------------------------------------

@comment This is REALLY the end :-)

@bye
